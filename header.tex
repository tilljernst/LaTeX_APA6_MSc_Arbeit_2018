\documentclass[
    doc,
    12pt,
    a4paper,
    biblatex,
    apacite,
    natbib,
    longtable]{apa6}
    
\usepackage[ngerman]{babel}
\usepackage[utf8x]{inputenc}
\usepackage{amsmath}
\usepackage{mathptmx} 
\usepackage{amssymb}
\usepackage{graphicx}
\usepackage[colorinlistoftodos]{todonotes}
\usepackage{pdfpages}
\usepackage[]{ragged2e}
\usepackage{glossaries}
\usepackage{array,makecell}
\usepackage{multirow}
\usepackage{multicol}
\usepackage{xcolor,colortbl}
\usepackage{pgfplots} %balkendiagramme
\pgfplotsset{compat=1.15}
\usepackage[onehalfspacing]{setspace} % Zeilenabstend
\geometry{top=25.4mm, left=25.4mm, right=25.4mm, bottom=25.4mm}
\usepackage{appendix}
\usepackage{nicefrac} % Darstellung von Brüchen im Text e.g. 1/2
\usepackage{listings} % Darstellung von SW-Code


% Für Verweise innerhalb Doc (siehe https://strobelstefan.org/?p=145) -> funktioniert nicht mich package hyperref!
%------------------------------------------------
\usepackage[autostyle=true,german=quotes]{csquotes}
\usepackage{prettyref}
\usepackage{titleref}
%%% Für Abschnitte %%%
\newrefformat{sec}{siehe Abschnitt~\ref{#1} \enquote{\titleref{#1}} \ auf Seite \pageref{#1}}
%%% Für Abbildungen %%%
\newrefformat{fig}{siehe Abb.~\ref{#1} \enquote{\titleref{#1}} \ auf Seite \pageref{#1}}
%%% Für Tabellen %%%
\newrefformat{tab}{siehe Tab.~\ref{#1} \enquote{\titleref{#1}} \ auf Seite \pageref{#1}}    
    
% Flattersatz ohne die Titel zu beeinflussen
\setlength{\RaggedRightParindent}{\parindent}
% Color Def für SW Code
\definecolor{dkgreen}{rgb}{0,0.6,0}
\definecolor{gray}{rgb}{0.5,0.5,0.5}
\definecolor{mauve}{rgb}{0.58,0,0.82}
\lstset{frame=tb,
  language=Java,
  aboveskip=3mm,
  belowskip=3mm,
  showstringspaces=false,
  columns=flexible,
  basicstyle={\small\ttfamily},
  numbers=none,
  numberstyle=\tiny\color{gray},
  keywordstyle=\color{blue},
  commentstyle=\color{dkgreen},
  stringstyle=\color{mauve},
  breaklines=true,
  breakatwhitespace=true,
  tabsize=3,
  frame=single
}