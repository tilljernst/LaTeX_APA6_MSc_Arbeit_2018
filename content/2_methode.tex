% ---------------------------------------
\subsection{Versuchssituation}\label{sec:Versuchssituation}
Die Durchführung dieser Arbeit erfolgte im Frühlingssemester 2018 im Rahmen einer Masterarbeit an der Zürcher Hochschule für Angewandte Wissenschaften (ZHAW) im Studiengang Angewandte Psychologie in der Vertiefung klinische Psychologie. Dieser Arbeit ging die Wahl des Themas und des Dozenten mittels Disposition im Herbstsemester 2017 voraus. Darin wurde anhand aktueller theoretischer Ansätze und der bisherigen Forschung (siehe auch Kap. \titleref{sec:Hintergrund}) das Vorgehen und das Design vordefiniert und zusammen mit dem betreuenden Dozenten besprochen. 

In einem ersten Schritt wurde, basierend auf der theoretischen Vorarbeit, ein standardisierter Fragebogen anhand der für die Beantwortung der Fragestellung notwendigen Konstrukte \textit{Bindung}, \textit{Stress}, \textit{Medienverhalten} und \textit{subjektives Wohlbefinden}, sowie der soziodemografischen Daten zusammengestellt. Dazu wurden soweit als möglich bereits normierte und validierte Instrumente oder Teile aus bereits existierender Umfragen verwendet. Die Umfrage erfolgte elektronisch über das Internet und setze sich aus einer kurzen Erläuterung der Studie, sowie entsprechende Anweisungen zur Beantwortung der Fragen und den ausgewählten Erhebungsinstrumenten zusammen. 

Die Umsetzung des Fragebogens richtete sich an die von der Umfragesoftware \textit{Enterprise Feedback Suite} der Firma \citeA{Questback2018} vorgegebenen Frageoptionen (ein Auszug des Fragebogens ist in angepasster Form im Anhang \titleref{app:Fragebogen} zu finden). Gestartet wurde die Umfrage mit einer kurzen Einführung, mit der Anonymitätserklärung, dem Hinweis auf den Wettbewerb und die Versuchspersonenstunden. Danach folgten die einzelnen Erhebungsinstrumente, beginnend mit der Erfassung der demographischen Daten, gefolgt vom Mediennutzungs-Fragebogen, dem Bindungsfragebogen, dem Stressfragebogen und dem Fragebogen zum subjektiven Wohlbefinden. Am Ende der Befragung konnten sich die Teilnehmer für die Teilnahme an einem Wettbewerb oder dem Erhalt für Versuchspersonenstunden, wenn sie Studierende der ZHAW in Psychologie waren, entscheiden. Zudem mussten die Teilnehmen angeben, ob sie die Fragen seriös ausgefüllt hatten. Der Fragebogen wurde mittels Pretest auf seine Funktionalität hin überprüft und erfolgte in zwei Schritten. Einerseits erlaubte die Software eine elektronische Testung des Fragebogens auf Korrektheit betreffend der möglichen Antwortpfaden. Dabei ging es um die Auswählbarkeit der Frageoptionen und die Erreichbarkeit aller aufgestellter Frageitems. In einem weiteren Schritt wurden fünf willkürlich ausgewählte Testpersonen aus dem Bekanntenkreis des Autoren als Testpersonen rekrutiert, die den kompletten Umfragebogen durchspieltn. Damit sollte sichergestellt werden, dass die Fragen und deren Anweisungen verständlich formuliert wurden und die Beantwortung der Fragen möglich war. Vor der Aktivschaltung des Fragebogens wurden alle bis dahin erfassten Daten der Tester gelöscht und die gesamte Umfrage frisch initialisiert.

Nach der Fertigstellung der Umfrage wurde diese für die Teilnehmer online geschalten. Ab diesem Zeitraum fand die Rekrutierung und die Vorarbeit für die theoretische Auswertung der Resultate statt. Am Ende der Befragung wurde anhand der bereinigten Daten die empirischen Hypothesen (siehe Kap. \titleref{sec:EmpirischeHypothesen}) geprüft und die Beantwortung der Fragestellung vorgenommen.

Die gesamte Arbeit dauerte von der Erstellung der Disposition im Herbstsemester 2017 bis zur Fertigstellung der Masterarbeit Ende Frühlingssemster 2018 zwei Semester und wurde Ende July 2018 zur Bewertung eingereicht.

% ---------------------------------------
\subsection{Design \& Operationalisiserung} \label{sec:Design}
Bei der vorliegenden Studie handelt es sich um eine empirische Querschnitts-Studie mit quantitativem Charakter. Aus ökonomischer Sicht und bezüglich einem hohen Mass an Standardisierung \cite[S.~86ff]{sedlmeier2008} wurde eine schriftliche Befragung via Internet-Fragebogen durchgeführt. Dadurch konnten die Befragten leichter kontaktiert und ein höherer Grad an Anonymität erreicht werden. Dazu wurden die für die Hypothesen relevanten Parameter empirisch erhoben und mittels deskriptiver Statistik ausgewertet. Der Fragebogen wurde in deutscher Sprache verfasst, weshalb die Rekrutierung ausschliesslich in der deutschsprachigen Schweiz erfolgte. Eine Übersetzung der Umfrage in weiter Sprachen hätte den Umfang dieser Arbeit gesprengt, da die einzelnen Fragebögen durch eine der wissenschaft genügenden Hin- und Rückübersetzung durch Personen aus den beiden Muttersprächen zu erfolgen gehabt hätte \cite{Pfetsch2016}.

Im Folgenden werden die einzelnen für die Beantwortung der Fragestellung notwendiger Konstrukte näher erläutert und operationalisiert.

\subsubsection{Soziodemografische Daten}\label{sec:SoziodemografischeDaten}
Bei den soziodemografischen Daten wurden folgende Parameter erfasst: Geschlecht, Jahrgang (Alter), Alter des Kindes in Monaten, Geschlecht des Kindes, brutto Familieneinkommen gemäss \citeA{NZZ2014}, Bildungsabschluss gemäss \citeA{Bfsnd}, Anzahl Personen im gleichen Haushalt, Lebensform, Anzahl Tage, an denen das Kind fremdbetreut ist und der durchschnittliche Betreuungsaufwand pro Woche.

\subsubsection{Medien und Mediennutzung}\label{sec:MedienMediennutzung}
Die Fragebogenitems wurden basierend auf den Vorlagen der Erhebungsinstrumenten der Studien \citeA{Feierabend2017, Blikk2017, Waller2016, Suter2015, Feierabend2015, Kabali2015} neu erstellt. Dabei stand die Mediennutzungszeit während der Betreuung der Kinder, also die Dauer der Mediennutzung in Minuten und das dazu verwendeten Geräte und Medium im Fokus. Es erfolgte ein Spezifizierung des Frageitems hinsichtlich des betreuten Kindes, ob dieses wach war oder geschlafen hat und hinsichtlich des benutzen Mediums, ob dieses privat oder geschäftlich genutzt wurde. Die Erfassung der Mediennutzungszeit wurde anhand der letzten Betreuungstätigkeit vorgenommen, da davon ausgegangen wurde, dass sich die Befragten am ehesten an die absolute Zeit erinnern konnten. Eine Unterteilung in Tage unter der Woche oder am Wochenende wurde als wenig zielführend erachtet. Für die Beantwortung der Fragestellung hätte bereits die reine Mediennutzungszeit gereicht. Es wurde jedoch als hilfreich erachtet, zusätzliche Parameter zu der reinen Mediennutzungsdauer zu erheben, um gegebenenfalls die Fragestellung detaillierter beantworten zu können. Ein weiterer Grund für die zusätzlichen Fragebogenitems ist hinsichtlich der spärlichen Datenlage im Bereich Mediennutzung von Eltern mit Kleinkindern zu nennen. Erfasst wurden neben dem benutzen Medium und der aufgewendeten Zeit, die im Haushalt vorhanden Geräte und auf welches der benutzen Geräte während der Betreuung am wenigsten verzichtet werden konnte. 

\textit{TBD: Operationalisierung - e.g. Gesamtzeit in Minuten, gegliedert nach Medium, etc.}

\subsubsection{Bindung: Adult Attachment Scale (AAS)}\label{sec:AAS}
Für die Erfassung des Bindungsstils (\textit{engl. attachment}) wurde der von \citeA{Schmidt2004} ins Deutsche übersetzte Adult Attachment Scale (AAS) von \citeA{Collins1990} verwendet. Dieser versucht die Bindung über die Selbstbeschreibungsmasse zu erfassen, in Ahnlehnung an die 1-Item-Selbstbeschreibungsmasse von \citeA{Hazan1987}. Dabei gilt dieser gemäss \citeA{Fraley2000} als einer der weit verbreiteten Selbsterfassungsbögen im englischsprachigen Raum.

Der AAS Fragebogen besteht aus insgesamt 18 Items, die auf einer fünfstufigen Likert-Skala von \enquote{stimmt gar nicht} (1), \enquote{stimmt eher nicht} (2), \enquote{stimmt teils / teils} (3), \enquote{stimmt eher} (4) bis \enquote{stimmt genau} (5) eingeschätzt werden. Dabei werden die Skalenwerte als Summenwerte der Itemantworten jeder Skala berechnet. Der Fragebogen bezieht sich auf bindungsrelevante Einstellungen der Befragten und wird mit drei Bindungsskalen erhoben: Die Skala \textit{Nähe} (1) beschreibt das Ausmass, in dem sich eine Person mit Nähe wohl fühlt und diese Nähe nicht mit übermässigen Ängsten verbindet. Die Skala \textit{Vertrauen} (2) beschreibt das Ausmass, in dem eine Person darauf vertraut, dass andere für sie verfügbar sind und dass die Person sich diesen anderen gegebenenfalls tatsächlich anvertrauen kann. Die Skala \textit{Angst} (3) beschreibt in erster Linie Ängste, allein gelassen oder verlassen zu werden. Was sich in einem übermässigen Bedürfnis nach Nähe aus, oder in Befürchtungen, der andere würde diese Bedürnisse zurückweisen. 

Gemäss der Prüfung der Faktorenstruktur des AAS durch \citeA{Schmidt2004} wurden die Elemente 2 und 9 auf Basis der inhaltlichen Ambivalenz der Itemformulierung und der Ergebnisse der konfirmatorischen Faktorenanalyse eliminiert. Diese Reduktion führte zu einer fünf-Item-Version der Skala Nähe und Angst. Die dadurch erhaltene interne Konsistenz (Reliabilität) erreichte in der Untersuchung von \citeA{Schmidt2004} ein $\alpha$ von 0,80 für die Skala Nähe, ein $\alpha$ von 0,72 für die Skala Vertauen und ein $\alpha$ von 0,78 für die Skala Angst. Somit lieferte die überarbeitete Version gemäss \citeA{Buehner2011} niedrige bis mittlere Werte und sind somit zufriedenstellend (Endgültige Version siehe Anhang \titleref{app:AAS}).

\textit{TBD: Operationalisierung}


\subsubsection{Stress: Perceived Stress Questionnaire (PSQ)}\label{sec:PSQ}
Für die Erfassung der aktuellen subjektiv erlebten Belastung, wurde der \textit{Perceived Stress Questionnaire (PSQ)} von \citeA{Levenstein1993} in der deutschen Übersetzung von \citeA{Fliege2001} verwendet.
\citeA{Levenstein1993} kreierte für den anglo-amerikanischen und italienischen Sprachraum ein Instrument, mit dem das Ausmass der subjektiv wahrgenommenen und erlebten aktuellen Belastung erfasst werden soll. Also das Ausmass aktuell wahrgenommener Belastungsfaktoren und das Erleben der eigenen Belastetheit auf der kognitiven und emotionalen Ebene \cite{Fliege2001}. Der Begriff Belastetheit soll verdeutlichen, dass es sich nicht um die Quelle der Belastung, sondern die Reaktion darauf gemeint ist. Die inhaltlichen Kriterien des Fragebogens sind (vgl. ebd.): (1) Stress wird als subjektives Belastungserleben erfasst. Damit wird gemeint, dass sich die Itemformulierung so weit wie möglich an der Perspektive der Wahrnehmung und Bewerung durch die Person orientiert. (2) Belastungsfaktoren und subjektive Belastetheit sollen als übergeordnete Klassen erfasst werden. Dies wird durch eine Vermeiden von Person- und situationsspezifische Formulierungen erreicht (z.B. keine eindeutig berufsbezogene Items). (3) Es wird lediglich das Belastungserleben und nicht der konkrete Umgang mit der Belastung erfragt. Dadurch soll sich der Test von der Erfassung von Bewältigungsbemühungen abgrenzen. Das Belastungserleben wird unabhängig von der Stelle erfasst, an der sich eine Person in einem möglichen Bewältingungsprozess befindet. (4) Es wird die Selbsteinschätzung der Person erhoben und somit nur der bewusste Anteil des Belastungserlebens.

Die Autoren \citeA{Levenstein1993} schlagen vor, das Instrument für Untersuchungen von Zusammenhängen zwischen Stresserleben und Krankheitsentwicklung zu verwenden. Auch wenn in dieser Arbeit weder vom Medienverhalten noch vom subjektiven Wohlbefinden als Krankheit gesprochen werden kann, so sollen die Vorzüge dieses Instruments bezüglich der ökonomisch Durchführbarkeit, die Eignung zur Erfassung möglichst verschiedener Lebenskontexten von Erwachsenen und die Fokussierung auf das gegenwärtige Erleben hervorgehoben werden \cite{Fliege2001}. Bezüglich der empirischen Prüfung und Gütekriterien dieses Testinstruments kann gemäss \citeA{Naescher2009} von einer Auswertungsobjektivität, einer mittleren bis hohen Realiabilität und einer soliden Valildierung ausgegangen werden. Der Perceived Stress Questionnaire \cite{Fliege2001} ist für alle Erwachsenen und für verschiedene Lebenssituationen geeignet. Die deutsche Stichprobe, auf die sich die Werte beziehen, wurde an Probanden zwischen 17 und 79 Jahren erhoben. Der Fragebogen kann gemäss \citeA{Naescher2009} zur Diagnostik des subjektiven Belastungserlebens bei Erwachsenen zweifellos empfohlen werden.

Durch die Validierung und Übersetzung des originalen Fragebogens ins Deutsche, wurden die ursprünglich 30 Items, basierend auf der durchgeführten exploratorischen Faktorenanalyse, auf 20 reduziert. Dadurch entstand die in dieser Arbeit verwendete Kurzversion dieses Fragebogens. Das Verfahren ist im \enquote{Elektronischen Testarchiv} des ZPID enthalten kann für nichtkommerzielle Forschungs- und Unterrichtszwecke kostenlos eingesetzt werden \cite{ZPID}. Ebenso konnten die ursprünglich siebenfaktorielle Zuordnung der Items nicht beibehalten werden. Folgende vier Skalen blieben dabei übrig: (1) Sorgen, (2) Anspannung, (3) Freude und (4) Anforderungen. Diese Veränderungen sind nachvollziehbar und bei \citeA{Fliege2001} ausführlich dokumentiert. Die Einschätzung erfolgt über eine viertstufige Likertskale, die von 1 für \textit{fast nie}, über 2 für \textit{manchmal}, 3 für \textit{häufig} bis zu 4 für \textit{meistens} reicht. Bei den Items handelt es sich um Feststellungen, die von der Testperson beurteilt werden sollen. 

TBD: Operationalisierung:
Gemäss Text von \citeA{Naescher2009}: In der nachfolgenden Aufstellung entspricht die Zahl vor der Klammer jeweils der Nummer des Items in der gekürzten und die dahinterstehende Zahl in Klammern der Nummer des Items in der Normalversion. Bei Items mit einem * muss der Wert des Items von 5 abgezogen werden, bevor das Ergebnis mit den Werten der anderen Items in die Berechnung eingeht. Hierbei handelt es sich um die Items 01 (01), 06 (10) und 19 (29). 
(a) Sorgen (worries): 05 (09), 07 (12), 10 (15), 13 (18), 15 (22); 
(b) Anspannung (tension): 01* (01*), 06* (10*), 09 (14), 17 (26), 18 (27); 
(c) Freude (joy): 04 (07), 08 (13), 12 (17), 14 (21), 16 (25); 
(d) Anforderungen (demands): 02 (02), 03 (04), 11 (16), 19* (29*), 20 (30). 
Die Werte der Items jeder Skala werden jeweils aufaddiert, mit -1 multipliziert und anschließend durch 3 dividiert. Hierdurch entsteht ein Wert zwischen 0 und 1, welcher im Anschluss mit 100 multipliziert wird. Das Ergebnis ist ein Skalenrang zwischen 0 und 100. Hohe Werte in einer Skala bedeuten jeweils auch eine hohe Ausprägung der betreffenden Eigenschaft.

\subsubsection{Subjektives Wohlbefinden: Subjective Happiness Scale (SHS)}\label{sec:SWB}
Die Erfassung des subjektiven Wohlbefindens erfolgte anhand dem von \citeA{Lyubomirsky1999} entwickelten \textit{Subjective Happiness Scale (SHS)}, in der deutschen Übersetzung und Überprüfung von \citeA{Swami2009}. Der SHS wurde für die Erfassung  eines allgemeinen subjektiven Wohlbefindens mittels Selbtseinschätzungsfragebogen entwickelt. Ein Wohlbefinden, dass in seiner globalen und grundsätzlichen Ausprägung, also ob es sich um einen glücklichen oder unglücklichen Menschen handelt, erfasst wird  \cite[S.~139ff]{Lyubomirsky1999}. Die Skala enthält insgesamt vier Items. Zwei davon erfassen das eigene Selbstbild, basierend auf einer absoluten Einschätzung des eigenen Wohlbefindens und einer Einschätzung verglichen mit dem eigenen sozialen Umfeld. Bei den weiteren beiden Items wird die befragte Person angehalten, sich bezüglich einer individuellen Beschreibungen glücklicher und unglücklicher Menschen einzuschätzen. Das Antwortformat der vier Items basiert auf einer sieben Punkte Likertskala. 

\citeA{Lyubomirsky1999} weisen darauf hin, dass der SHS sich für die Erfassung des subjektiven Wohlbefindens besser eignet als andere vergleichbare Instrumente (z.B.: \textit{Affect Balance Scale} oder \textit{the Satisfaction With Life Scale}), da der Fragebogen eine globale subjektive Einschätzung bezügliche Wohlbefinden vornimmt, im Gegensatz zur Erfassung von eigentlichen Glückszuständen, die einer bestimmte Zeitspanne zuzuordnen sind (z.B.: affektive und kognitive Zustände). Der Fragebogen wurde in 14 Studien mit einem Total von 2,732 Teilnehmern validiert. Die Daten wurden innerhalb der vereinigten Staaten anhand von Studenten zweier Universitäten und einer Mittelschule, von Erwachsenen innerhalb zweier Städten in Kalifornien   erhoben. Zudem wurden Teilnehmer aus Moswkau, Russland beigezogen. Demzufolge soll der SHS eine hohe interne Konsistenz aufweisen, welche über alle Stichproben stabil blieben. Die Test-Retest-Korrelation und die Korrelation zwischen Eigen- und Fremdeinschätzung deuten auf eine gute bis sehr gute Realiabilität hin. 
Zudem deute die Konstruktvalidität darauf hin, dass der Fragebogen auch wirklich das Konstrukt subjektives Wohlbefinden misst (vgl. ebd.). \citeA{Swami2009} konnten in ihrer Studie nachweisen, dass es sich bei der deutschen Übersetzung um eine unidimensionale Struktur, mit einer hohen internen Konsitenz handelt. Diese Übersetzung korrelierte hoch mit anderen Messinstrumenten, welche das subjektive Wohlbefinden erfassen. Demzufolge habe der neue Fragebogen eine gute Konvergenzvalidität. Hingegen konnten die Forscher keine Aussage über die Test-Retest-Reliabilität machen, noch nahmen sie eine Validierungsprüfung vor, da dies nicht zum Gegenstand ihrer Überprüfung zählte (vgl. ebd.).

TBD: Operationalisierung:


\subsection{Datenerhebung \& Rekrutierung}
Zur Datengewinnung wurde die von der ZHAW zur Verfügung gestellten Umfragesoftware Enterprise Feedback Suite in der Version 10.9 der Firma \citeA{Questback2018} verwendet. Der Fragebogen wurde mit Hilfe der zur Verfügung gestellten Dokumentation kreiert \cite{EFS2016} und wurde Mitte Januar bis Ende Mai 2018 aktiv geschaltet.  Der Zugang zur Befragung erfolgte über die von der Firma \citeA{Questback2018} zur Verfügung gestellten öffentlichen Internetadresse, welche zum Zeitpunkt der Befragung wie folgt lautete: https://ww2.unipark.de/uc/elternfragebogen/.

Die Rekrutierung erfolgte im Sinne einer Gelegenheitsstichprobe, da das Ziel dieser Untersuchungen in der Erfassung der Veränderung der abhängigen Varibalen Mediennutzung anhand der unabhängigen Variablen Bindung und Stress ist und sich dieses Vorgehen als ökonomisch und praktikabel im Hinblick auf die Forschungsfrage herausstellte \cite{TUDresden2015}. Für den Versand der Umfrage wurde ein Infomail und ein Infozettel (Flyer) mit den wichtigsten Angabe zur Studie und dem Link zur Umfrage erstellt (siehe dazu \titleref{app:Mailing} und \titleref{app:Flyer} im Anhang). Der Flyer wurde zusätzlich in Papierform ausgedruckt. Für den Versand des Flyers und des Infomailings wurden unterschiedliche Kanäle verwendet: Einerseits wurden diese per Post für die jeweilige Auslage und den Aushang versendet, andererseits erfolgte der Versand elektronisch über Mail, Facebook-Posting und Text-Nachrichten. Zudem wurde der Flyer auf unterschiedlichen einschlägigen Webseiten publiziert. Um den Medienbruch möglichst gering zu halten, wurde wann immer möglich versucht, den Link für die Umfrage elektronisch zugänglich zu machen. 

Die Rekrutierung fand an folgenden Orten und Institutionen statt (für eine detaillierte Übersicht, siehe Anhang \titleref{app:Rekrutierung}): Direkt an der ZHAW, über die Elternberatungsstellen der deutschsprachigen Schweiz \cite{Sfmvb2018}, über Spitäler mit Wochenbett und Kinderärzte, über die deutschsprachigen Familienzentren \cite{NetzwerkBildung2018}, über die Anschrift aller deutschsprachiger Niederlassungen der \citeA{Elternbildung2018} und über das private Umfeld des Autors, mit der Bitte, die Umfrage im eigenen Umkreis weiter zu verteilen. Zudem wurde die Umfrage auf der Internetplattform der Elternzeitschrift \enquote{Fritz \&  Fränzi}, dem Marktplatz des Elternmagazins \enquote{Wir Eltern} und dem Forum der Internetplattform \enquote{Swissmom} publiziert.

\textit{TBD: Beschreibung der Datenbereinigung:} Am Ende der Umfrage erfolgte die Datenbereinigung...


\subsection{Beschreibung der Stichprobe}
Die Stichprobe umfasste insgesamt 233 Personen. Die Merkmale Geschlecht, Alter (in Jahren), etc. sind in Tabelle \ref{table:Stichprobe} dargestellt.

\begin{table}[htbp]

\begin{tabular}{m{0.5em}  m{10em}  m{5em}} 
  \hline
  \multicolumn{2}{l}{\textbf{Merkmal}} & \textbf{$n$ (\%)} \\
  \hline
  \multicolumn{2}{l}{Geschlecht} \\ 
   & weiblich & 212 (91)\\ 
   & männlich & 20 (8.6)\\ 
   & keine Angbabe & 1 (.4)\\ 
   
  \multicolumn{2}{l}{Alter in Jahren} \\
  & $M$ (Min./Max.) & 33.8 (20/48) \\
  & $SD$ & 4.6 \\
   
  \hline
\end{tabular}
\caption{Gesamtstichprobe}
\label{table:Stichprobe}
\end{table}


\subsection{Empirische Hypothesen}\label{sec:EmpirischeHypothesen}
\subsection{Statistische Analyseverfahren}