% ---------------------------------------
\subsection{Versuchssituation}\label{sec:Versuchssituation}
Die Erstellung dieser Arbeit erfolgte im Frühlingssemester 2018 im Rahmen einer Masterarbeit an der Zürcher Hochschule für Angewandte Wissenschaften (ZHAW) im Studiengang Angewandte Psychologie in der Vertiefungsrichtung klinische Psychologie. Die Wahl des Themas und die Erstellung der Disposition erfolgte im Herbstsemester 2017. In dieser wurde anhand aktueller theoretischer Ansätze und der bisherigen Forschung (siehe auch Kap. \titleref{sec:Hintergrund}) das Vorgehen und das Design vordefiniert und zusammen mit dem betreuenden Dozenten abgesprochen. 

In einem ersten Schritt erfolgte die Erstellung eines standardisierter Fragebogens. Dieser basierte auf der theoretischen Vorarbeit und beinhaltete die für die Beantwortung der Fragestellung notwendigen Konstrukte \textit{Bindung}, \textit{Stress}, \textit{Medienverhalten}, \textit{subjektives Wohlbefinden} sowie der Erfassung soziodemografischer Daten. Für die Erstellung des Fragebogens wurden soweit als möglich bereits normierte und validierte Instrumente oder Teile aus bereits existierender Umfragen verwendet. Die Umfrage erfolgte elektronisch über das Internet und setzte sich aus einer kurzen Erläuterung der Studie, sowie entsprechende Anweisungen zur Beantwortung der Fragen zusammen. 

Der Fragebogen wurde mittels der von der ZHAW zur Verfügung gestellten Umfragesoftware \textit{Enterprise Feedback Suite} der Firma \citeA{Questback2018} erstellt. Die grafische Umsetzung richtete sich an die von der Software vorgegebenen Fragebogentypen und deren Antwortoptionen (ein Auszug des Fragebogens ist in angepasster Form im Anhang \ref{app:Fragebogen} zu finden). Der Aufbau der Umfrage gliedert sich grob in eine kurze Einführungssektion, gefolgt von der Erfassung der demografischen Daten und dem Hauptteil, der die einzelnen Konstrukte erfasste und dem Ende der Umfrage. In der Einführung wurde die Anonymitätserklärung eingebunden, einen Hinweis auf die mögliche Teilnahme an einem  Wettbewerb gemacht und über die Option für Psychologiestudierende der ZHAW, sich Versuchspersonenstunden anrechnen zu lassen, informiert. Anschliessend erfolgte die Erfassung der demografischen Daten. Im Hauptteil wurden die einzelnen Konstrukte erhoben, beginnend mit dem Mediennutzungs-Fragebogen, gefolgt vom Bindungsfragebogen, dem Stressfragebogen und dem Fragebogen für die Einschätzung des subjektiven Wohlbefindens. Am Ende der Befragung konnten sich die Teilnehmer für die Teilnahme am Wettbewerb oder dem Erhalt für Versuchspersonenstunden entscheiden. Versuchspersonenstunden konnten sich nur Studierende der ZHAW in Psychologie anrechnen lassen. Ebenso am Schluss wurden die Teilnehmer gebeten, die Seriosität ihrer Antworten anzugeben (Umfrage seriös beantwortet oder nicht). 

Die Testung des Fragebogens erfolgte mittels Pretest bezüglich seiner Funktionalität. Die Software erlaubte eine erste elektronische Testung auf Korrektheit. Dabei wurde geprüft, ob bei der Beantwortung alle Fragen durchlaufen wurden. In einem weiteren Test wurden fünf willkürlich ausgewählte Testpersonen aus dem Bekanntenkreis des Autoren als Testpersonen rekrutiert, welche den Fragebogen durchspielten. Diese wurden angehalten, die Fragen und deren Anweisungen auf Verständlichkeit zu überprüfen. Zudem sollten sie prüfen, ob die Beantwortung sinnvoll möglich war. Eine weitere detaillierte Prüfung wurde durch den betreuenden Dozenten vorgenommen. 

Nach dem Abschluss der Testphase wurden alle bis dahin erfassten Daten der Tester gelöscht und die gesamte Umfrage frisch initialisiert. Die Umfrage wurde aktiviert und für die Teilnehmer verfügbar gemacht. In diesem Zeitraum erfolgte die Rekrutierung der Stichprobe (siehe dazu \titleref{sec:Datenerhebung} auf Seite \pageref{sec:Datenerhebung}) und die Vorbereitung für die Auswertung der Daten. 

Am Ende der Rekrutierungsphase wurde eine Mail an alle involvierten Personen und Institutionen gesandt, um auf das Ende der Umfrage hinzuweisen. Anschliessend wurden die Daten bereinigt und in einem Brutto- und Nettodatensatz abgelegt. Es wurden die zum Bezug von Versuchpersonenstunden berechtigte Personen ermittelt und der ZHAW gemeldet. Zudem wurden die Gewinner des Wettberwerbs mittels randomisierter Ziehung ausgewählt und über ihren Gewinn informiert. Es erfolgte die Aufarbeitung der Daten und die Prüfung der empirischen Hypothesen (siehe Kap. \titleref{sec:EmpirischeHypothesen}). Mittels Dikussion erfolgte eine kritische Auseinandersetzung mit den Befunden und dem Abschluss dieser Arbeit.

Die gesamte Masterarbeit dauerte von der Erstellung der Disposition im Herbstsemester 2017 bis zur Fertigstellung der Masterarbeit Ende Frühlingssemster 2018 zwei Semester und wurde Ende Juli 2018 zur Bewertung eingereicht.

% ---------------------------------------
\subsection{Design} \label{sec:Design}
Bei der vorliegenden Studie handelt es sich um eine empirische Querschnitts-Studie mit quantitativem Charakter. Aus ökonomischer Sicht und bezüglich einem hohen Mass an Standardisierung wurde eine schriftliche Befragung via Internet-Fragebogen durchgeführt \cite[S.~86ff]{sedlmeier2008}. Dadurch konnten die Befragten leichter kontaktiert und ein höherer Grad an Anonymität erreicht werden. Dazu wurden die für die Hypothesen relevanten Parameter empirisch erhoben und mittels deskriptiver Statistik ausgewertet. Der Fragebogen wurde in deutscher Sprache verfasst, weshalb die Rekrutierung ausschliesslich in der deutschsprachigen Schweiz erfolgte. Eine Übersetzung der Umfrage in weiter Sprachen hätte den Umfang dieser Arbeit gesprengt, da die einzelnen Fragebögen durch eine der wissenschaft genügenden Hin- und Rückübersetzung durch Personen aus allen involvierten Muttersprachen zu erfolgen gehabt hätte \cite{Pfetsch2016}.

Bezüglich Minimierung von Risiken für die Versuchspersonen wurden keine spezifischen Vorkehrungen getroffen, da von einer Unbedenklichkeit des Fragebogens bezüglich ethischen Richtlinien ausgegangen wurde.

Im Folgenden werden die Stichprobenumfangsplanung erläutert und die einzelnen für die Beantwortung der Fragestellung notwendiger Konstrukte operationalisiert.

\subsubsection{Stichprobenumfangsplanung}\label{sec:Stichprobenumfangsplanung}
Gemäss Fragestellung wurde eine Stichprobengrösse von insgesamt 210 Personen angestrebt. Diese Stichprobengrösse wurde mittels G*Power \cite{Faul2009} für eine einfaktorielle Varianzanalyse gemäss \citeA{Rasch2014} ermittelt (F-Test - ANOVA: Fixed effects, omnibus, one-way).
Für die Stichprobenumfangsplanung wird ein mittlerer Effekt \cite{Cohen1988a} als inhaltlich relevant festgelegt: $f = 0.25$ resp. $\Omega^2 = 0.06$. Das Signifikanzniveau beträgt $\alpha=0.05$. Die Teststärke soll mindesten $1-\beta=0.95$ betragen. Es werden zwei Gruppen für den Faktor Bindung benötigt (sicher und unsicher).

Für die Beantwortung des Zusammenhangs zwischen Medienverhalten und subjektiven Wohlbefinden wird eine lineare Regressionsanalyse durchgeführt. Das Kriterium bildet das subjektive Wohlbefinden und der Prädiktor das Medienverhalten. Die Stichprobengrösse für die Beantwortung dieser Fragestellung wurde wieder mittels G*Power für eine Korrelation zweiseitig berechnet, da ein ungerichteter Zusammenhang angenommen wird. Gemäss der Einteilung von \citeA{Cohen1988a} wird ein mittlerer Effekt von $r=.3$ erwartet. Das Signifikanzniveau beträgt $\alpha=0.05$. Die Teststärke soll mindesten $1-\beta=0.95$ betragen. Gemäss diesen Vorgaben wurde eine Stichprobengrösse von 138 Probanden angestrebt.


\subsubsection{Operationalisiserung} \label{sec:Operationalisierung}


\paragraph{Soziodemografische Daten}\label{sec:SoziodemografischeDaten}
Bei den soziodemografischen Daten wurden folgende Parameter erfasst: Geschlecht, Jahrgang (Alter), Alter des Kindes in Monaten, Geschlecht des Kindes, brutto Familieneinkommen gemäss \citeA{NZZ2014}, Bildungsabschluss gemäss \citeA{Bfsnd}, Anzahl Personen im gleichen Haushalt, Lebensform, Anzahl Tage, an denen das Kind fremdbetreut ist und der durchschnittliche Betreuungsaufwand pro Woche.

\paragraph{Medien und Mediennutzung}\label{sec:MedienMediennutzung}
Die Fragebogenitems für die Erfassung der Medien und der Mediennutzung wurden basierend auf den Vorlagen der Studien von \citeA{Feierabend2017, Blikk2017, Waller2016, Suter2015, Feierabend2015, Kabali2015} neu zusammengestellt. Dabei stand die Mediennutzung während der Betreuung der Kinder in Minuten und das dazu verwendeten Medium im Fokus. Zudem erfolgte ein Spezifizierung der Einschätzung der Mediennutzung hinsichtlich des betreuten Kindes, ob dieses während der Betreuung wach war oder geschlafen hat. Des Weiteren wurde erfragt, ob das Medium für die Nutzung privat oder geschäftlich genutzt wurde. 

Die Einschätzung der absoluten Mediennutzung der Eltern erfolgte anhand der letzten Betreuung ihrer Kinder. Es wurde davon ausgegangen, dass die Nutzung für die Probanden am einfachsten aus der Erinnerung abrufbar sei. Eine Unterteilung der Nutzung nach Wochentag oder Wochenende wurde als wenig zielführend erachtet. Für die Beantwortung der Fragestellung hätte bereits die reine Mediennutzung gereicht. Es wurde jedoch als hilfreich erachtet, zusätzliche Parameter zur reinen Nutzung zu erheben, um gegebenenfalls die Fragestellung detaillierter beantworten zu können. Erfasst wurden neben der Mediennutzung und der aufgewendeten Zeit, die dazu benutzten Medien, die im Haushalt vorhanden Geräte und auf welches der benutzen Geräte die Eltern während der Betreuung am wenigsten verzichten konnten. Zu den erfassten Medien gehörten: (1) Smartphone, (2) TV, (3) Desktop / Laptop Computer, (4) Tablet, (5) Radio / Stereoanlage / CD-Player, (6) Printmedien (Buch, Zeitung, Heft, Comic), (7) Foto- und / oder Videokamera, (8) Spielekonsole und (9) MP3 Player.

Die Operationalisiserung erfolgte anhand der Summenbildung der Mediennutzung in Minuten, während denen das Kind betreut wurde. Es wurden insgesamt 11 Tätigkeiten abgefragt: (1) Telefonieren, (2) Textnachrichten lesen und schreiben (SMS, Whatsapp, etc.), (3) Emails lesen und bearbeiten, (4) Lesen / anschauen von Printmedien (Bilder-Bücher), (5) Internet nutzen (surfen, Informationen suchen, etc.), (6) Musik hören, (7) Fernsehen und Videos schauen (Sender, Netflix, DVD, etc.), (8) Radio hören, (9) Fotos oder Videos machen, (10) Video-Games spielen (Handy, Computer, Konsole, etc.) und (11) Hörspiele oder Hörbücher hören. 

%Formel
\begin{equation}\label{eq:MedienTotal}
    MN_{Total}=MN_{Wach} + MN_{Schlafend}
\end{equation}
\begin{equation}\label{eq:MedienWach}
    MN_{Wach}=\sum_{i=1}^{11} i_{Wach}
\end{equation}
\begin{equation}\label{eq:MedienSchlafend}
    MN_{Schlafend}=\sum_{i=1}^{11} i_{Schlafend}
\end{equation}

Die Gesamtsumme der Mediennutzung (MN) $MN_{Total}$ (Formel \ref{eq:MedienTotal}) setzt sich aus den Teilsummen der Tätigkeiten zusammen, an denen das Kind wach war $MN_{Wach}$ (Formel \ref{eq:MedienWach}) oder geschlafen hat $MN_{Geschlafen}$ (Formel \ref{eq:MedienSchlafend}).



\paragraph{Bindung: Adult Attachment Scale (AAS)}\label{sec:AAS}
Für die Erfassung des Bindungsstils (\textit{engl. attachment}) wurde der von \citeA{Schmidt2004} ins Deutsche übersetzte Fragebogen \textit{Adult Attachment Scale (\acrshort{aas})} von \citeA{Collins1990} verwendet. Dieser erfasst die Bindung über die Selbst\-beschreibungs\-masse, in Ahnlehnung an die 1-Item-Selbst\-beschreibungs\-masse von \citeA{Hazan1987}. Dabei gilt dieser gemäss \citeA{Fraley2000} als einer der weit verbreiteten Selbst\-er\-fassungs\-bögen im englisch\-sprachigen Raum.

Der \acrshort{aas} Fragebogen besteht aus insgesamt 18 Items, die auf einer fünfstufigen Likert-Skala von \enquote{stimmt gar nicht} (1), \enquote{stimmt eher nicht} (2), \enquote{stimmt teils / teils} (3), \enquote{stimmt eher} (4) bis \enquote{stimmt genau} (5) eingeschätzt werden. Dabei werden die Skalenwerte als Summenwerte der Itemantworten jeder Skala berechnet. Der Fragebogen bezieht sich auf bindungsrelevante Einstellungen der Befragten und wird mit drei Bindungsskalen erhoben: Die Skala \textit{Nähe} (1) beschreibt das Ausmass, in dem sich eine Person mit Nähe wohl fühlt und diese Nähe nicht mit übermässigen Ängsten verbindet. Die Skala \textit{Vertrauen} (2) beschreibt das Ausmass, in dem eine Person darauf vertraut, dass andere für sie verfügbar sind und dass die Person sich diesen anderen gegebenenfalls tatsächlich anvertrauen kann. Die Skala \textit{Angst} (3) beschreibt in erster Linie Ängste, allein gelassen oder verlassen zu werden. Was sich in einem übermässigen Bedürfnis nach Nähe aus, oder in Befürchtungen, der andere würde diese Bedürnisse zurückweisen. 

Gemäss der Prüfung der Faktorenstruktur des \acrshort{aas} durch \citeA{Schmidt2004} wurden die Elemente 2 und 9 auf Basis der inhaltlichen Ambivalenz der Itemformulierung und der Ergebnisse der konfirmatorischen Faktorenanalyse eliminiert. Diese Reduktion führte zu einer fünf-Item-Version der Skala Nähe und Angst. Die dadurch erhaltene interne Konsistenz (Cronbachs $\alpha$) erreichte in der Untersuchung von \citeA{Schmidt2004} ein $\alpha$ von 0,80 für die Skala Nähe, ein $\alpha$ von 0,72 für die Skala Vertauen und ein $\alpha$ von 0,78 für die Skala Angst. Somit lieferte die überarbeitete Version gemäss \citeA{Buehner2011} niedrige bis mittlere Werte und sind somit zufriedenstellend (Endgültige Fragebogen-Version siehe Anhang \ref{app:AAS}).

Die Operationalisiserung erfolgte anhand der Indizes für die drei Skalen Nähe, Vertrauen und Angst (siehe Formel \ref{eq:IndexNähe}, \ref{eq:IndexVertrauen}, \ref{eq:IndexAngst}). Dabei wurden die Mittelwerte aus den skalenzugehörigen Items anhand des Likertskalenwerts 1 bis 5 jeder Person berechnet. Zu beachten ist, dass alle Items in der Skala Nähe und die Items $AAS_{5}$, $AAS_{10}$, $AAS_{15}$ und $AAS_{17}$ der Skala Vertrauen für die Erstellung des Skalenwerts umkodiert werden müssen (im Anhang \ref{app:AAS} mit einem * gekennzeichnet). 

%Formel
\begin{equation}\label{eq:IndexNähe}
    AAS_{\text{\textit{N{\"a}he}}}=6-\frac{AAS_{3}+AAS_{8}+AAS_{13}+AAS_{14}+AAS_{18}}{5}
\end{equation}
\begin{equation}\label{eq:IndexVertrauen}
    AAS_{Vertrauen}=\frac{AAS_{1}+(6-AAS_{5})+(6-AAS_{10})+AAS_{12}+(6-AAS_{15})+(6-AAS_{17})}{6}
\end{equation}
\begin{equation}\label{eq:IndexAngst}
    AAS_{Angst}=\frac{AAS_{4}+AAS_{6}+AAS_{7}+AAS_{11}+AAS_{16}}{5}
\end{equation}

Je höher die Mittelwerte der Skalen ausfallen, desto grösser die entsprechende Ausprägung. Menschen mit einem hohen Wert in der Skala Nähe fühlen sich bei Nähe zu anderen Menschen eher wohl. Hohe Ausprägungen bei der Skala Vertrauen geht tendenziell mit einem grösseren Vertrauen anderen Menschen gegenüber einher. Je höher die Werte der Skala Angst, desto eher fürchtet sich ein Mensch alleine gelassen zu werden.

Bezügliche einer besseren Interpretier- und Lesbarkeit können die oben beschriebenen Skalen mittels Skalenrange-Transformation in einen Wertebereich zwischen 0 und 100 transformiert werden. Dies geschieht mit folgender Formel, die stellvertretend für alle obigen Skalen des \acrshort{aas} steht (Formel \ref{eq:AASIndexTrans}):

%Formel
\begin{equation}\label{eq:AASIndexTrans}
    AAS_{Trans}=\frac{AAS_{OrigSkala - 1}}{4}*100
\end{equation}

Dem Mittelwert der zu tranformierenden Formel, die einen Skalenrange von 1 bis 5 hat, wird durch Subtraktion von 1 eine lineare Transformation zu einem Wertebereich 1 bis 4 vorgenommen. Mittels Division durch 4 und anschliessender Multiplikation mit 100 wird eine weitere lineare Transformation zu einem Wertebereich von 0 bis 100 vorgenommen.

Wie bereits oben beschrieben, lassen sich gemäss \citeA{Schmidt2004} mit dem deutschen Instrument die Bindungsstile nicht direkt zuordnen, sondern in zugrunde liegende Dimensionen (Cluster) einteilen \cite{Schuetzmann2004}. Anhand den Ausprägungen der Probanden in den drei Skalen lassen sich drei Cluster definieren: Im Cluster \textit{sicher} sind Probanden mit hohen Werten bei den Skalen Nähe und Vertrauen und niedrige Werte bei der Skala Angst enthalten. Das Cluster \textit{ängstlich} beinhaltet Probanden mit hohen Werten bei der Skala Angst und mittlere Werte bei den Skalen Nähe und Vertrauen. Im dritten Cluster \textit{vermeidend} befinden sich Probanden, die auf allen drei Skalen Nähe, Vertrauen und Angst tiefe Werte aufweisen.

Für die Beantwortung der Fragestellung genügt die Einteilung in sichere und unsichere Probanden. Demzufolge werden die Probanden anhand dem Cluster \textit{sicher} aufgeteilt, wodurch zwei Gruppen sicher und unsicher gebundene Probanden entstehen. 
Für die Clusterbildung werden alle Probanden innerhalb der drei Skalen in zwei Bereiche \textit{hohe Werte} = 2 und \textit{tiefe Werte} = 1 unterteilt. Dies geschieht mit den entsprechenden Trennwert (Cutoff), der sich aus dem Mittelwerten ($M$) und der Standardabweichung ($SD$) der einzelnen Skalen bildet (siehe Formel \ref{eq:AASCutoffNähe}, \ref{eq:AASCutoffVertrauen} und \ref{eq:AASCutoffAngst}). 

%Formel
\begin{equation}\label{eq:AASCutoffNähe}
    Cutoff_{\text{\textit{N{\"a}he}}}=M_{\text{\textit{N{\"a}he}}} - SD_{\text{\textit{N{\"a}he}}}
\end{equation}
\begin{equation}\label{eq:AASCutoffVertrauen}
    Cutoff_{Vertrauen}=M_{Vertrauen} - SD_{Vertrauen}
\end{equation}
\begin{equation}\label{eq:AASCutoffAngst}
    Cutoff_{Angst}=M_{Angst} + SD_{Angst}
\end{equation}

Für die Einteilung in hohe und tiefe Werte der Skala Nähe, wird erst der Trennwert berechnet (Formel \ref{eq:AASCutoffNähe}), indem die Standardabweichung $SD$ vom Mittelwert $M$ der Skala Nähe abgezogen wird. Anschliessend wird jeder Proband, dessen Skalenausprägung kleiner diesem Wert ist mit 1 und jeder dessen Wert grösser als dieser Wert ist mit 2 eingeteilt. Dieses Vorgehen wird analog für die Skala Vertrauen und Angst wiederholt. Zu beachten ist, dass bei der Skala Angst die Standardabweichung zum Mittelwert addiert wird, da hohe Werte eine höhere Ausprägung von Ängsten beinhaltet und für die Einteilung in sicher Probanden eine tiefe Ausprägung gewünscht wird. Anschliessend werden die Probanden gemäss den Konventionen von Cluster \textit{sicher} in sicher und unsicher gebundene Probanden kodiert (Formel \ref{eq:AASSicherVsUnsicher}).

%Formel
\begin{equation}\label{eq:AASSicherVsUnsicher}
    AAS_{Sicher}=( AAS_{\text{\textit{N{\"a}he}}} > 1) ~AND~ (AAS_{Vertrauen} > 1) ~AND~ (AAS_{Angst} < 2)
\end{equation}

\paragraph{Stress: Perceived Stress Questionnaire (PSQ)}\label{sec:PSQ}
Für die Erfassung der aktuellen subjektiv erlebten Belastung, wurde der \textit{Perceived Stress Questionnaire (\acrshort{psq})} von \citeA{Levenstein1993} in der deutschen Übersetzung von \citeA{Fliege2001} verwendet.
\citeA{Levenstein1993} kreierte für den anglo-amerikanischen und italienischen Sprachraum ein Instrument, mit dem das Ausmass der subjektiv wahrgenommenen und erlebten aktuellen Belastung erfasst werden soll. Also das Ausmass aktuell wahrgenommener Belastungsfaktoren und das Erleben der eigenen Belastetheit auf der kognitiven und emotionalen Ebene \cite{Fliege2001}. Der Begriff Belastetheit soll verdeutlichen, dass es sich nicht um die Quelle der Belastung handelt, sondern die Reaktion darauf gemeint ist. Die inhaltlichen Kriterien des Fragebogens sind (vgl. ebd.): (1) Stress wird als subjektives Belastungserleben erfasst. Damit ist gemeint, dass sich die Itemformulierung so weit wie möglich an der Perspektive der Wahrnehmung und Bewertung durch die Person orientiert. (2) Belastungsfaktoren und subjektive Belastetheit sollen als übergeordnete Klassen erfasst werden. Dies wird durch eine Vermeiden von Person- und situationsspezifische Formulierungen erreicht (z.B. keine eindeutig berufsbezogene Items). (3) Es wird lediglich das Belastungserleben und nicht der konkrete Umgang mit der Belastung erfragt. Dadurch soll sich der Test von der Erfassung von Bewältigungsbemühungen abgrenzen. Das Belastungserleben wird unabhängig von der Stelle erfasst, an der sich eine Person in einem möglichen Bewältingungsprozess befindet. (4) Es wird die Selbsteinschätzung der Person erhoben und somit nur der bewusste Anteil des Belastungserlebens.

Die Autoren \citeA{Levenstein1993} schlagen vor, das Instrument für Untersuchungen von Zusammenhängen zwischen Stresserleben und Krankheitsentwicklung zu verwenden. Auch wenn in dieser Arbeit weder vom Medienverhalten noch vom subjektiven Wohlbefinden als Krankheit gesprochen werden kann, so sollen die Vorzüge dieses Instruments bezüglich der ökonomisch Durchführbarkeit, die Eignung zur Erfassung möglichst verschiedener Lebenskontexten von Erwachsenen und die Fokussierung auf das gegenwärtige Erleben hervorgehoben werden \cite{Fliege2001}. Bezüglich der empirischen Prüfung und Gütekriterien dieses Testinstruments kann gemäss \citeA{Naescher2009} von einer Auswertungsobjektivität, einer mittleren bis hohen Realiabilität und einer soliden Valildierung ausgegangen werden. 
Für die Reliabilitätsprüfung wurden sowohl die innere Konsistenz als auch die Split-half-Reliabilität für die unterschiedlichen Stichproben berechnet. Die Ergebnisse befanden sich für alle Skalen im oberen Bereich. Die Werte für Cronbachs Alpha lagen gemäss \cite{Fliege2001} der Gesamtstichprobe bei $\alpha$ = .86 (Faktor I Sorgen), .84 (Faktor II: Anspannung), .85 (Faktor III: Freude), .80 (Faktor IV: Anforderungen) und bezogen auf alle vier Skalen bei .85. Bezüglich der Split-half-Reliabilität ergaben sich ähnliche Werte, weswegen von einer mittleren bis hohen Reliabilität ausgegangen werden kann \cite{Fliege2001}.

Der \acrshort{psq} \cite{Fliege2001} ist für alle Erwachsenen und für verschiedene Lebenssituationen geeignet. Die deutsche Stichprobe, auf die sich die Werte beziehen, wurde an Probanden zwischen 17 und 79 Jahren erhoben. Der Fragebogen kann gemäss \citeA{Naescher2009} zur Diagnostik des subjektiven Belastungserlebens bei Erwachsenen zweifellos empfohlen werden.

Durch die Validierung und Übersetzung des originalen Fragebogens ins Deutsche, wurden die ursprünglich 30 Items, basierend auf der durchgeführten exploratorischen Faktorenanalyse, auf 20 reduziert. Dadurch entstand die in dieser Arbeit verwendete Kurzversion dieses Fragebogens. Das Verfahren ist im \enquote{Elektronischen Testarchiv} des ZPID enthalten kann für nichtkommerzielle Forschungs- und Unterrichtszwecke kostenlos eingesetzt werden \cite{ZPID}. Ebenso konnten die ursprünglich siebenfaktorielle Zuordnung der Items nicht beibehalten werden. Folgende vier Skalen blieben dabei übrig: (1) \textit{Sorgen}, (2) \textit{Anspannung}, (3) \textit{Freude} und (4) \textit{Anforderungen}. Diese Veränderungen sind nachvollziehbar und bei \citeA{Fliege2001} ausführlich dokumentiert. Die Einschätzung erfolgt über eine viertstufige Likertskale, die von 1 für \enquote{fast nie}, über 2 für \enquote{manchmal}, 3 für \enquote{häufig} bis zu 4 für \enquote{meistens} reicht. Bei den Items handelt es sich um Feststellungen, die von der Testperson beurteilt werden sollen (siehe dazu auch Anhang \ref{app:PSQ}). 

Die Operationalisierung erfolgte anhand der Anleitung von \citeA{Naescher2009}. Dabei wurden die Indizes der vier Skalen Sorgen, Anspannung, Freude und Anforderungen mit Hilfe der Mittelwertsbildung erstellt (siehe \ref{eq:PSQIndexSorgen}, \ref{eq:PSQIndexAnspannung}, \ref{eq:PSQIndexFreude} und \ref{eq:PSQIndexAnforderung}). Dazu werden die jeweiligen Items der zugehörigen Skala aufaddiert und durch deren Anzahl dividiert. Hohe Werte in einer Skala bedeuten jeweils auch eine hohe Ausprägung der betreffenden Eigenschaft.

%Formel
\begin{equation}\label{eq:PSQIndexSorgen}
    PSQ_{Sorgen}=\frac{\frac{PSQ_{05}+PSQ_{07}+PSQ_{10}+PSQ_{13}+PSQ_{15}}{5}-1}{3}*100
\end{equation}
\begin{equation}\label{eq:PSQIndexAnspannung}
    PSQ_{Anspannung}=\frac{\frac{(5-PSQ_{01})+(5-PSQ_{06})+PSQ_{09}+PSQ_{17}+PSQ_{18}}{5}-1}{3}*100
\end{equation}
\begin{equation}\label{eq:PSQIndexFreude}
    PSQ_{Freude}=\frac{\frac{PSQ_{04}+PSQ_{08}+PSQ_{12}+PSQ_{14}+PSQ_{16}}{5}-1}{3}*100
\end{equation}
\begin{equation}\label{eq:PSQIndexAnforderung}
    PSQ_{Anforderung}=\frac{\frac{PSQ_{02}+PSQ_{03}+PSQ_{11}+(5-PSQ_{19})+PSQ_{20}}{5}-1}{3}*100
\end{equation}

Zu beachten ist, dass gewisse Items für die Berechnung der Indizes umkodiert werden müssen. Hierbei handelt es sich um die drei Items $PSQ_{01}$, $PSQ_{06}$ und $PSQ_{19}$ (im Anhang \ref{app:PSQ} mit einem * gekennzeichnet).

Zudem wird von \citeA{Naescher2009} eine Skalenrange-Transformation von 0 bis 100 vorgeschlagen. Dazu wird aus dem Mittelwert der Skala mit dem Skalenranges von 1 bis 4 durch Subtraktion von 1 eine lineare Transformation zu 0 bis 3 vorgenommen. Die weitere lineare Transformation mittels Division durch 3 ergibt einen Wert zwischen 0 und 1, welcher multipliziert mit 100 ein Skalenrange zwischen 0 und 100 ergibt.  

Für den Gesamtscore muss die Skala Freude umkodiert werden:

%Formel
\begin{equation}\label{eq:PSQGesamtscore}
    PSQ_{Gesamtscore}=\frac{PSQ_{Sorgen}+PSQ_{Anspannung}+ (100-PSQ_{Freude})+PSQ_{Anforderung}}{4}
\end{equation}

In den Gesamtscore fliessen die oben berechneten Skalen ein und ergeben das allgemeine Stresserleben, respektive die aktuelle, subjektiv erlebte Belastung einer Person.

\paragraph{Subjektives Wohlbefinden: Subjective Happiness Scale (SHS)}\label{sec:SWB}
Die Erfassung des subjektiven Wohlbefindens (\acrshort{swb}) erfolgte anhand dem von \citeA{Lyubomirsky1999} entwickelten \textit{Subjective Happiness Scale (\acrshort{shs})}, in der deutschen Übersetzung und Überprüfung von \citeA{Swami2009}. Der \acrshort{shs} wurde für die Erfassung  eines allgemeinen subjektiven Wohlbefindens mittels Selbtseinschätzungsfragebogen entwickelt. Ein Wohlbefinden, dass in seiner globalen und grundsätzlichen Ausprägung erfasst wird, also ob es sich um einen glücklichen oder unglücklichen Menschen handelt  \cite[S.~139ff]{Lyubomirsky1999}.

Die Skala enthält insgesamt vier Items. Zwei davon erfassen das eigene Selbstbild, basierend auf einer absoluten Einschätzung des eigenen Wohlbefindens und einer Einschätzung verglichen mit dem eigenen sozialen Umfeld. Bei den weiteren beiden Items wird die befragte Person angehalten, sich bezüglich einer individuellen Beschreibungen glücklicher und unglücklicher Menschen einzuschätzen (siehe dazu auch Anhang \ref{app:SHS}). Das Antwortformat der vier Items basiert auf einer sieben Punkte Likertskala (z.B.: 1 für \enquote{Kein glücklicher Mensch} bis 7 \enquote{Sehr glücklicher Mensch}). 

\citeA{Lyubomirsky1999} weisen darauf hin, dass sich der \acrshort{shs} für die Erfassung des subjektiven Wohlbefindens besser eignet als andere vergleichbare Instrumente (z.B.: \textit{Affect Balance Scale} oder \textit{the Satisfaction With Life Scale}), da der Fragebogen eine globale subjektive Einschätzung bezügliche Wohlbefinden vornimmt, im Gegensatz zur Erfassung von eigentlichen Glückszuständen, die einer bestimmte Zeitspanne zuzuordnen sind (z.B.: affektive und kognitive Zustände). Der Fragebogen wurde in 14 Studien mit einem Total von 2,732 Teilnehmern validiert. Die Daten wurden innerhalb der vereinigten Staaten anhand von Studenten zweier Universitäten und einer Mittelschule von Erwachsenen innerhalb zweier Städten in Kalifornien erhoben. Zudem wurden Teilnehmer aus Moswkau, Russland beigezogen. Demzufolge soll der \acrshort{shs} eine hohe interne Konsistenz aufweisen, welche über alle Stichproben stabil blieb. Die Test-Retest-Korrelation und die Korrelation zwischen Eigen- und Fremdeinschätzung deuten auf eine gute bis sehr gute Realiabilität hin.  
Zudem deute die Konstruktvalidität darauf hin, dass der Fragebogen auch wirklich das Konstrukt subjektives Wohlbefinden misst (vgl. ebd.). \citeA{Swami2009} konnten in ihrer Studie nachweisen, dass es sich bei der deutschen Übersetzung um eine unidimensionale Struktur, mit einer hohen internen Konsistenz handelt. Das Cronbach $\alpha$ für die deutsche Version belief sich auf .82. Diese Übersetzung korrelierte hoch mit anderen Messinstrumenten, welche das subjektive Wohlbefinden erfassen. Demzufolge habe der neue Fragebogen eine gute Konvergenzvalidität. Hingegen konnten die Forscher keine Aussage über die Test-Retest-Reliabilität machen, noch nahmen sie eine Validierungsprüfung vor, da dies nicht zum Gegenstand ihrer Überprüfung zählte (vgl. ebd.).

Die Operationalisierung erfolgt anhand der Mittelwertsbildung der vier Items \cite{Lyubomirsky1999} und ergibt einen \acrshort{shs}-Index (siehe \titleref{eq:SHS}). Die mögliche Ausprägung dieses Index reicht von 1.0 bis 7.0. Eine hohe Ausprägung steht für ein hohe subjektive Zufriedenheit.

%Formel
\begin{equation}\label{eq:SHS}
    SHS~I=\frac{SHS_{1}+SHS_{2}+SHS_{3}+SHS_{4}}{4}
\end{equation}

\subsection{Datenerhebung \& Rekrutierung}\label{sec:Datenerhebung}
Zur Datengewinnung wurde die von der ZHAW zur Verfügung gestellten Umfragesoftware Enterprise Feedback Suite in der Version 10.9 der Firma \citeA{Questback2018} verwendet. Der Fragebogen wurde mit Hilfe der zur Verfügung gestellten Dokumentation kreiert \cite{EFS2016} und wurde Mitte Januar bis Ende Mai 2018 aktiv geschaltet.  Der Zugang zur Befragung erfolgte über die von der Firma \citeA{Questback2018} zur Verfügung gestellten öffentlichen Internetadresse, welche zum Zeitpunkt der Befragung wie folgt lautete: https://ww2.unipark.de/uc/elternfragebogen/.

Die Rekrutierung erfolgte im Sinne einer Gelegenheitsstichprobe, da das Ziel dieser Untersuchungen in der Erfassung der Veränderung der abhängigen Variable Mediennutzung anhand der unabhängigen Variablen Bindung und Stress war und sich dieses Vorgehen als ökonomisch und praktikabel im Hinblick auf die Forschungsfrage herausstellte \cite{TUDresden2015}. Für den Versand der Umfrage wurde ein Infomail und ein Infozettel (Flyer) in elektronischer Form mit den wichtigsten Angabe zur Studie und dem Link zur Umfrage erstellt (siehe dazu Anhang \ref{app:Mailing} und \ref{app:Flyer}). Der Flyer wurde zusätzlich in Papierform für den Versand per Post und die Auslage ausgedruckt. 

Für den Versand des Infomailings und des Flyers wurden unterschiedliche Kanäle verwendet: Einerseits wurden diese per Post für die jeweilige Auslage und den Aushang versendet, andererseits erfolgte der Versand elektronisch über Mail, Facebook-Posting und Text-Nachrichten. Zudem wurde der Flyer auf unterschiedlichen einschlägigen Webseiten publiziert. Um den Medienbruch möglichst gering zu halten, wurde wann immer möglich versucht, den Link für die Umfrage elektronisch zugänglich zu machen. 

Die Rekrutierung fand an folgenden Orten und Institutionen statt (für eine detaillierte Übersicht, siehe Auflistung in Anhang \ref{app:Rekrutierung}): Direkt an der ZHAW, über die Elternberatungsstellen der deutschsprachigen Schweiz \cite{Sfmvb2018}, über Spitäler mit Wochenbett und Kinderärzte, über die deutschsprachigen Familienzentren \cite{NetzwerkBildung2018}, über die Anschrift aller deutschsprachiger Niederlassungen der \citeA{Elternbildung2018} und über das private Umfeld des Autors, mit der Bitte, die Umfrage im eigenen Umkreis weiter zu verteilen. Zudem wurde die Umfrage auf der Internetplattform der Elternzeitschrift \enquote{Fritz \&  Fränzi}, dem Marktplatz des Elternmagazins \enquote{Wir Eltern} und dem Forum der Internetplattform \enquote{Swissmom} publiziert.

Um für die Teilnahme an der Umfrage möglichst viele Eltern zu gewinnen, wurde mittels Wettbewerb und Versuchspersonenstunden Anreize geschaffen. Beim Wettbewerb handelte es sich um Warengutscheine bei einem renomierten Onlineshop für Babybedarf von 10 x CHF 50.- (www.baby-rose.ch). Dabei wurde am Ende des Fragebogens die Möglichkeit geboten, am Wettbewerb teilzunehmen und die eigene Emailadresse für die Benachrichtigung im Fall eines Gewinns zu hinterlegen. Am Schluss der Rekrutierungsphase wurden die Gewinner per Zufall gezogen. Dazu wurden alle Wettberwerbsteilnehmen in eine separate Liste kopiert und nummeriert. Anschlissend wurde mittels minimalem Swift-Skript \cite{Swift2017} zehn Teilnehmer randomiesiert ausgewählt (Code siehe Anhang \ref{app:Wettbewerb}). Diese wurden mittels Mail über ihren Gewinn informiert. Neben dem Wettbewerb konnten sich Studierende der ZHAW in Angewandter Psychologie 0.5 Versuchspersonenstunden anrechenen lassen, wenn sie die Umfrage durchführten. Die Validierung über die Berechtigung der Versuchspersonenstunden erfolgte innerhalb der ZHAW im Departement Psychologie (mail: diana.graf@zhaw.ch). 

\subsection{Beschreibung der Stichprobe}
Insgesamt schlossen brutto 248 Personen die Umfrage ab. 1454 Personen sind zumindest auf die erste Seite gelangt und haben anschliessend abgebrochen. Dies entspricht einer Beendingungsquote von 17.06\%. Die meisten Abbrüche sind auf der Startseite mit 1045 zu verzeichnen (71.87\%), gefolgt von den demografischen Daten mit 100 (6.88\%) und Medien und Mediennutzung mit 44 (3\%). Die mittlere Bearbeitungszeit des Fragebogens dauerte 27 Minuten. 

Nach der Datenbereinigung blieben netto 218 Personen übrig (87.9\%). Eine Personen wurde ausgemustert, da sie beim Jahrgang einen fehlerhaften Wert angegeben hatte (Jg 2017). Zwei weitere Person gaben an, die Umfrage nicht seriös ausgefüllt zu haben. Zwei Personen mussten entfernt werden, da die Personen bei der Angabe der Mediennutzungszeit in Minuten mit Ja und Nein geantwortet haben. 25 Personen wurden entfernt, da das Alter ihrer Kinder 13 Monate überstieg. 

Die Merkmale Geschlecht, Alter in Jahren, Geschlecht Kind und Alter Kind in Jahren sind in Tabelle \ref{table:Stichprobe} \textit{\titleref{table:Stichprobe}} dargestellt (eine um die Merkmale Familieneinkommen, Bidlungsabschluss, Anzahl Personen im Haushalt, Lebensfrom und Fremdbetreuung in Tagen ist in Tabelle \textit{\ref{table:Gesamtstichprobe}} im Anhang zu finden).

\begin{table}[htbp]
\begin{tabular}{m{0.5em}  m{10em}  m{5em}} 
  \hline\hline
  \multicolumn{2}{l}{\textbf{Merkmal}} & \textbf{$n$ (\%)} \\
  \hline
  \multicolumn{2}{l}{Geschlecht} \\ 
   & weiblich & 198 (90.8)\\ 
   & männlich & 19 (8.7)\\ 
   & keine Angabe & 1 (.5)\\ 
   
  \multicolumn{2}{l}{Alter in Jahren} \\
   & $M$ (Min./Max.) & 33.9 (20/48) \\
   & $SD$ & 4.5 \\
  
  \multicolumn{2}{l}{Geschlecht Kind} \\
   & weiblich & 97 (44.5)\\ 
   & männlich & 119 (54.6)\\ 
   & keine Angabe & 2 (.9)\\
  
  \multicolumn{2}{l}{Alter Kind in Monaten} \\
   &  $M$ (Min./Max.) & 6.6 (0/12)\\
   & $SD$ & 3.4\\
  \hline\hline
  &&\\
\end{tabular}
\caption{Stichprobe}
\label{table:Stichprobe}
\end{table}

Das Alter der Befragten liegt zwischen 20 und 48 Jahren mit einem Durchschnittsalter von 33.9 Jahren ($SD$ = 4.5). Weibliche Probanden sind mit 90.8 \% ($n$ = 198) in der Stichprobe vertreten. Das Geschlecht der Kinder ist mit 44.5\% weiblichen ($n$ = 97) und 54.6\% männlichen ($n$ = 119) Kindern vertreten. Das Alter der Kinder liegt zwischen 0 und 12 Monaten mit einem Durchschnitt von 6.6 Monaten ($SD$ = 3.4).

Bei der Angabe zum Familieneinkommen (siehe Tabelle \textit{\ref{table:Gesamtstichprobe}} im Anhang) zeigt sich die stärkste Ausprägung bei Personen, die über ein Familieneinkommen von über CHF 104'001 und mehr verfügen mit 34.3\% ($n$ = 75), gefolgt von Familien mit einem Einkommen zwischen CHF 78'001 und CHF 104'000 mit 30.3\% ($n$ = 66) und Familien mit einem Einkommen zwischen CHF 52'000 und CHF 78'000 mit 23.9\% ($n$ = 52). 

73 Personen gaben an über ein Uni oder Fachhochschultitel zu verfügen, was mit 33.5\% die höchste Ausprägung darstellt, dicht gefolgt von einem Lehrabschluss mit 23.9\% ($n$ = 52) und einem Abschluss an einer höheren Fachschule mit 22.9\% ($n$ = 50). Zusammenfassend ergibt das einen Abschluss im tertiären Bereich von 60.6\% ($n$ = 132), im Sekundar II Bereich von 36.6\% ($n$ = 79) und im Bereich der Grundschule 2.8\% ($n$ = 6).

Im Durchschnitt leben zwischen 2 und 8 Personen im Haushalt mit einem Durchschnitt von 3.6 Personen ($SD$ = .8). Die dominante Lebensform stellt die verheiratete Partnerschaft mit 71.6\% ($n$ = 156) dar, gefolgt von der nicht verheirateten Partnerschaft mit 25.2\% ($n$ = 55). Die Kinder werden im Schnitt 1 Tag ($SD$ = 1.3)  zwischen 0 und 5 Tagen fremd betreut.

Insgesamt haben 158 Personen am Wettbewerb teilgenommen, was 72.5\% der Stichprobe entspricht. Eine Person liess sich Versuchspersonen anrechnen (.5\%) und 59 (27\%) Personen wollten weder am Wettbewerb teilnehmen noch sich Versuchspersonenstunden anrechnen lassen.



\subsection{Empirische Hypothesen}\label{sec:EmpirischeHypothesen}
Im Folgenden werden die konzeptionellen Hypothesen aus Kapitel \titleref{sec:Fragestellung} in empirische Hypothesen überführt:

(1) Die Mediennutzung von sicher gebundenen Eltern ist im Durchschnitt signifikant tiefer als diejenige der unsicher gebundenen Eltern.

(2) Eltern, deren Stressempfinden hoch ist, haben im Durchschnitt eine signifikant höhere Mediennutzung als Eltern, deren Stressempfindung tiefer ist.

(3) Eltern, die eine tiefe Mediennutzung vorweisen, haben im Durchschnitt ein signifikant höheres subjektives Wohlbefinden.

(4) Das Stressempfinden und die Mediennutzung von sicher gebundenen Eltern ist im Durchschnitt signifikant tiefer als dasjenige der unsicher gebundenen Eltern.


\subsection{Statistische Analyseverfahren}
Für die Datenauswertung wurde in einem ersten Schritt eine deskriptive, univariate Analyse der Stichprobendaten durchgeführt. Dies ermöglichte die Erfassung möglicher Fehler bei der Datenerfassung und die Entdeckung möglicher Ausreisser im Datensatz. Dies erfolgte mittels Häufigkeitsverteilung, um die verschiedenen Merkmalsausprägungen einer Variable im Datensatz zu beschreiben, und mittels Verteilungsparameter (Lage- und Streuungsparameter). Des Weiteren wurde die Schiefe der Häufigkeitsverteilung gegenüber der Normalverteilung überprüft. Die Prüfung der Normalverteilung wurde einerseits mittels Shapiro-Wilks-Test \cite{Shapiro1965} und mittels Sichtvergleich vorgenommen. Für den Sichtvergleich kam einerseits ein Histogramm zum Einsatz, welches die Verteilung der Daten grafisch darstellt \cite{Hemmerich2018}. Bei normalverteilten Daten sollte das Histogramm ein klassisch glockenförmiges Aussehen aufweisen. Zudem soll ein Q-Q-Plot erstellt werden, um die Daten visuell auf Normalverteilung zu prüfen \cite{Hemmerich2018}. 

Anschliessend an die deskriptive Analyse wurden in einem weiteren Schritt die statistischen Zusammenhänge der Stichprobendaten für die Hypothesen berechnet. Für die Voraussetzungen der einfaktoriellen Varianzanalyse soll die Homogenität der Varianzen mittels Levene-Test geprüft werden. Für die Voraussetzungen der einfachen linearen Regressionsanalyse soll die Linearität des Zusammenhangs, der bedingte Erwartungswert (Gauss-Markov-Annahme 3) und die Homoskedastiziät (Gauss-Markov-Annahme 5) visuell mittels Streudiagramm geprüft werden \cite{UniversitatZurich2018}.

Die Beantwortung der Hypothese 1 erfolgt mittels einfaktoriellen Varianzanalyse, engl. ANOVA (Analysis of Variance). Dabei soll das Signifikanzniveau $\alpha=0.05$ und die Teststärke mindesten $1-\beta=0.95$ betragen. Es wird ein mittlerer Effekt \cite{Cohen1988a} als inhaltlich relevant angesehen $f = 0.25$. Der Bindungsstil stellt dabei den zweistufigen Faktor, resp. die unabhängige Variable dar (Faktorstufen sicher / unsicher). Die Abhängige Variable wird mit der Mediennutzung besetzt. Die Nullhypothese $H1_{0}$ (\ref{eq:Hypothese_1_Null}) lautet: Die Bindung hat keinen signifikanten Einfluss auf die Mediennutzung, somit existiert kein Unterschied zwischen den Mittelwerten der einzelnen Gruppen \enquote{sicher} und \enquote{unsicher} und es besteht kein Effekt. Die Alternativhypothese $H1_{1}$ (\ref{eq:Hypothese_1_Alt})  lautet: Die Bindung hat einen signifikanten Einfluss auf die Mediennutzung, somit unterscheiden sich die beiden Gruppen voneinander.

%Formel Hypothesen
\begin{equation}\label{eq:Hypothese_1_Null}
    H1_{0} = \mu_1 = \mu_2
\end{equation}
\begin{equation}\label{eq:Hypothese_1_Alt}
    H1_{1}:p=.000 ~ AND ~ f>=.25~ (\sigma_1 = \sigma_2)
\end{equation}

Die Beantwortung der Hypothese 2 erfolgt mittels einfacher linearen Regression. Gemäss der Einteilung von \citeA{Cohen1988a} wird ein mittlerer Effekt von $f=.25$ erwartet. Das Signifikanzniveau soll $\alpha=0.05$ und die Teststärke $1-\beta=0.95$ betragen. Es soll die Regression von der Mediennutzung auf das Stressemfpinden der Eltern berechnet werden. Die Nullhypothese $H2_{0}$ (\ref{eq:Hypothese_2_Null}) lautet: Das Stressemfpinden der Eltern korreliert nicht  mit der Mediennutzung der Eltern, somit sind die Regressionskoeffizienten = 0. Die Alternativhypothese $H2_{1}$ (\ref{eq:Hypothese_2_Alt}) lautet: Das Stressempfinden der Eltern korreliert mit der Mediennutzung der Eltern, dabei wird mindestens ein Regressionskoeffizient $\neq$ 0.

%Formel Hypothesen
\begin{equation}\label{eq:Hypothese_2_Null}
    H2_{0}=\beta_0 = \beta_1 = 0
\end{equation}
\begin{equation}\label{eq:Hypothese_2_Alt}
    H2_{0}=\beta_0 \text{ und/oder } \beta_1 \neq 0
\end{equation}

Die Beantwortung der Hypothese 3 erfolgt analog zu Hypothese 2 mittels einfacher linearen Regression. Dabei soll die Regression des subjektiven Wohlbefindens auf die Mediennutzung berechnet werden. Die Nullhypothese $H3_{0}$ (\ref{eq:Hypothese_3_Null}) lautet: Es besteht kein Zusammenhang zwischen Eltern mit einer geringen Mediennutzung und einem hohen subjektiven Wohlbefinden, somit sind die Regressionskoeffizienten = 0. Die Alternativhypothese $H3_{1}$ (\ref{eq:Hypothese_3_Alt}) lautet: Es besteht ein Zusammenhang zwischen Eltern mit einer geringen Mediennutzung und einem hohen subjektiven Wohlbefinden, dabei wird mindestens ein Regressionskoeffizient $\neq$ 0.

%Formel Hypothesen
%Formel Hypothesen
\begin{equation}\label{eq:Hypothese_3_Null}
    H3_{0}=\beta_0 = \beta_1 = 0
\end{equation}
\begin{equation}\label{eq:Hypothese_3_Alt}
    H3_{0}=\beta_0 \text{ und/oder } \beta_1 \neq 0
\end{equation}

Für die Beantwortung der Hypothese 4 wird eine einfaktorielle Varianzanlyse verwendet. Dabei soll das Signifikanzniveau $\alpha=0.05$ und die Teststärke mindesten $1-\beta=0.95$ betragen. Es wird ein mittlerer Effekt \cite{Cohen1988a} als inhaltlich relevant angesehen $f = 0.25$. Der Bindungsstil stellt dabei den zweistufigen Faktor, resp. die unabhängige Variable dar (Faktorstufen sicher / unsicher) dar und das Stressempfinden fliesst als Kovariate in die Berechung ein. Die Abhängige Variable wird mit der Mediennutzung besetzt. Die Nullhypothese $H4_{0}$ (\ref{eq:Hypothese_4_Null}) lautet: Die Bindung und das Stressempfinden haben keinen signifikanten Einfluss auf die Mediennutzung, somit existiert kein Unterschied zwischen Mittelwertn der beiden Gruppen \enquote{sicher} und \enquote{unsicher} und es besteht kein Effekt. Die Alternativhypothese $H4_{1}$ (\ref{eq:Hypothese_4_Alt}) lautet: Die Bindung und das Stressempfinden haben einen signifikanten Einfluss auf die Mediennutzung, somit unterscheiden sich die beiden Gruppen voneinander.

%Formel Hypothesen
\begin{equation}\label{eq:Hypothese_4_Null}
    H4_{0} = \mu_1 = \mu_2
\end{equation}
\begin{equation}\label{eq:Hypothese_4_Alt}
    H4_{1}:p=.000 ~ AND ~ f>=.25~ (\sigma_1 = \sigma_2)
\end{equation}