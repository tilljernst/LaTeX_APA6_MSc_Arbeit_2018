% ---------------------------------------
\subsection{Beantwortung der Fragestellung} \label{sec:BeantwortungFragestellung}
Auf die Beantwortung der Fragestellung, ob der Bindungsstil und das aktuelle Stressempfinden der Eltern einen Zusammenhang mit dem im Beisein der Kinder praktizierten Medienverhalten haben und ob dieses Verhalten einen Effekt auf das subjektive Wohlbefinden ausübt, soll in diesem Abschnitt eingegangen werden.

Gemäss den Ergebnissen dieser Untersuchung konnte kein direkter Zusammenhang zwischen dem Bindungsstil der Eltern, unterteilt nach sicher und unsicher gebunden, auf die Mediennutzung der Eltern im Beisein ihrer Kinder festgestellt werden. Es konnte auch dann kein Zusammenhang festgestellt werden, wenn die Mediennutzung danach unterteilt wurde, ob das Kind während der Betreuung wach war oder geschlafen hat. Zudem scheint der Faktor Stress in keinem zusätzlichen Zusammenhang mit der Mediennutzung zu stehen. Aus den vorliegenden Daten kann demzufolge Bindung und Stress in keinen Zusammenhang mit der totalen Mediennutzung der Eltern während der Betreuung ihrer Kinder gebracht werden.

Des Weiteren konnte kein Zusammenhang zwischen der Mediennutzung der Eltern und dem subjektiv erlebten Wohlbefinden der Eltern gefunden werden. Daraus lässt sich schliessen, dass Mediennutzung und Wohlbefinden in der vorliegenden Stichprobe keinen Zusammenhang bilden.

Die Fragestellung setzte sich aus den vier Hypothesen und der Grundhypothese zusammen. Gemäss den Ergebnissen mussten alle vier Hypothesen abgelehnt werden und es gelten die Alternativhypothesen. Daraus folgend musste die Grundhypothese ebenfalls abgelehnt werden, da sich diese aus den vier Hypothesen zusammensetzt. Gemäss Hypothese $H1$ geht ein sicherer Bindungsstil, verglichen mit einem unsicheren Bindungsstil, nicht mit einer geringeren Mediennutzung der Eltern einher. Bei Hypothese $H2$ ist Stress kein Prädiktor für eine erhöhte Mediennutzung während der Betreuung der Kinder. Eine erhöhte Mediennutzung während der Betreuung geht auch nicht mit einem geringeren subjektiven Wohlbefinden einher (Hypothese $H3$). Schliesslich schlagen sich ein sicherer Bindungsstil und ein tiefes Stressempfinden nicht in der geringeren Mediennutzung nieder (Hypothese $H4$).

Aus diesen Ergebnissen lässt sich ableiten, dass die Bindung der Eltern während der Betreuung ihrer Kindern und der subjektiv erlebte Stress der Eltern keine Prädiktoren für die Mediennutzung sind. Ebenso widerspiegelt sich die Mediennutzung nicht im subjektiven Wohlbefinden der Eltern.

% ---------------------------------------
\subsection{Interpretation} \label{sec:Interpretation}
Steht nun die Bindung in einem Zusammenhang mit der Mediennutzung der Eltern und wird diese Nutzung nicht auch vom Stress moderiert? Lässt sich zudem ein Zusammenhang zwischen der Mediennutzung der Eltern und dem subjektiven Wohlbefinden der Eltern finden? Das vorliegende Kapitel soll auf einzelne Aspekte näher eingehen und dies zu beantworten versuchen.

% Stichprobe
\subsubsection{Die Stichprobe ist weiblich und gut gebildet}
Knapp 91 \% der Studienteilnehmenden sind Frauen. Nur gerade 19 Männer haben an der Befragung teilgenommen. Diese ungleiche Verteilung kann unterschiedlich begründet werden. Einerseits scheint der Hauptanteil bei der Rollenverteilung für die Betreuung von Kleinkindern nach wie vor dem weiblichen Elternteil zuzufallen. Gemäss \citeA{Bien2006} leben zwei Drittel der Mütter in Deutschland nach dem traditionellen Familienmodell. Ein weiterer Grund könnte sein, dass der Anreiz der Gutscheine für Männer weniger interessant gewesen ist. Zudem haben Männer möglicherweise generell weniger Interesse an psychologischen Umfragen und nehmen deshalb weniger an solchen Studien teil. Trotz dieser ungleichen Verteilung konnte die Fragestellung ausreichend beantwortet werden, da im Vorfeld kein Einfluss des Geschlechts postuliert wurde. Basierend auf der geringen Teilnahme männlicher Probanden konnte kein signifikanter Unterschied bezüglich der Mediennutzung zwischen den Geschlechtern festgestellt werden. Dies scheint darauf hinzudeuten, dass für die Mediennutzung während der Betreuung der Kinder das Geschlecht keine Rolle spielt.

Ein weiteres auffälliges Merkmal der Stichprobe zeigte sich im hohen Bildungslevel der Befragten. 60.6 \% gaben an, einen Abschluss im tertiären Sektor zu besitzen, und 36.2 \% gaben an, einen Bildungsabschluss im Sekundar-II-Sektor gemacht zu haben. Gemäss \citeA{Bfs2017} haben in der Altersgruppe der 25- bis 34-Jährigen in der Schweiz 41.8 \% einen Sekundar-II-Abschluss und 50.2 \% einen Abschluss auf Tertiärstufe. Die Stichprobe liegt demzufolge leicht über dem schweizerischen Schnitt. Dies könnte auf die Rekrutierung aus dem privaten Umfeld des Autors zurückzuführen sein, da dieser sich vornehmlich an Bekannte aus dem Studium richtete. Zudem scheint die hohe Bildung mit dem Durchschnittsalter der Stichprobe von 33.9 Jahren einherzugehen. Gemäss \citeA{Bfs2017a} sind  Frauen und Männer mit einer Tertiärbildung tendenziell über 30 Jahre alt, wenn sie ihr erstes Kind bekommen. Obwohl diese Zuordnung nachträglich nicht mehr möglich ist, so kann, basierend auf der durchschnittlichen Anzahl Personen im Haushalt von 3.6, davon ausgegangen werden, dass sich die Stichprobe im Schweizer Mittel bewegt. Ob sich die Bildung, das Einkommen und das Alter auf die Ergebnisse dieser Arbeit auswirken, ist schwer zu sagen. Gemäss \citeA{Livingstone2015} haben höher gebildete Eltern mehr Vertrauen in ihre digitalen Fähigkeiten im Umgang mit Medien. Zudem scheinen höher gebildete Eltern mit einem höheren Einkommen digitale Aktivitäten mit ihrem Nachwuchs eher zu reglementieren. Des Weiteren scheinen soziodemografische Merkmale wichtige Faktoren der Eltern-Kind-Beziehung zu sein \cite{Kammerl2012}. All diese Aussagen stehen in direktem Zusammenhang mit dieser Studie. Demzufolge könnten die hohen Bildungsabschlüsse und die hohen Einkommen protektiv auf die Mediennutzung der Eltern im Beisein der Kinder wirken und dem postulierten Effekt einer unsicheren Bindung auf eine erhöhte Mediennutzung entgegenwirken. 

\subsubsection{Haben Fachhochschulabgänger weniger Zeitmangel und Termindruck?}
Werden die demografischen Daten mit den einzelnen erfassten Skalen in Verbindung gesetzt, so kann ein Zusammenhang zwischen der Bildung und der Stressskala Anforderung gefunden werden. Diese Skala bildet die Wahrnehmung vor allem externer Anforderungen ab, beispielsweise Zeitmangel, Termindruck oder Aufgabenbelastung. Die Gruppe Berufsmatura / Fachmittelschule hatte im Vergleich zur Gruppe Höhere Fachschule im Schnitt eine höhere Ausprägung im Bereich Anforderung. Dies könnte ein Hinweis darauf sein, dass Personen mit einer Berufsmatura in der Familie eine höhere Anforderung erleben als Personen mit einem Abschluss einer höheren Fachschule. Letztere haben höhere Anforderungen für die Erreichung des Diploms durchlaufen, da sich diese  Ausbildung auf einem höheren und demzufolge anspruchsvolleren Niveau befindet. Deshalb könnten diese Personen die an sie gestellten Anforderungen gewohnt sein und tendenziell besser aushalten. Für eine schlüssige Interpretation müssten weitere Untersuchungen in diesem Bereich angestellt werden. Im Bezug zur hier gestellten Forschungsfrage könnte diese höhere Belastbarkeit wiederum protektiv auf das Stresserleben der Stichprobe gewirkt haben. Da dies jedoch nur anhand einer einzigen Ausprägung des \acrshort{psq} aufgetreten ist, kann davon ausgegangen werden, dass dieser Effekt vernachlässigbar ist. Rückschliessend kann gefolgert werden, dass bezüglich der demografischen Daten der Stichprobe keine signifikanten Zusammenhänge zwischen den einzelnen Skalen Bindung, Stress und subjektives Wohlbefinden gefunden werden konnten. Bezogen auf die demografischen Daten scheint die Stichprobe sehr heterogen zu sein, da keine statistisch signifikanten Gruppen gebildet werden konnten. Dies wiederum würde für die Gültigkeit der Stichprobe (repräsentativ) und der untersuchten Instrumente sprechen, da keine Kohorteneffekte mittels dieser Daten ersichtlich wurden.

\subsubsection{Die Stichprobenaufteilung bezüglich Bindung widerspiegelt die Forschung}
Die Bindungsskala \acrfull{aas}, bestehend aus den drei Unterskalen \enquote{Nähe}, \enquote{Vertrauen} und \enquote{Angst}, kann zu einer erweiterten Skala kombiniert werden. In dieser neuen Skala können die Probanden anhand sicherer und unsicherer Bindung aufgeteilt werden. In der vorliegenden Stichprobe ergab das eine Verteilung von 
69.7 \% sicher und 30.3 \% unsicher gebundene Eltern. Dies entspricht etwa dem Befund der Metaanalyse von \citeA{VanIJzendoorn1988}, die annähernd 2000  Probanden bezüglich ihrer Bindungsmuster anhand des Fremde-Situations-Tests untersuchten. Aus dieser Studie geht eine Aufteilung von 65 \% sicher und 35 \% unsicher gebundene Probanden hervor. Dies wiederum entspricht etwa den Befunden von \citeA{Ainsworth1970} von 70 \% sicher gebundenen und 30 \% unsicher gebundenen Probanden. Somit kann davon ausgegangen werden, dass die vorliegenden Stichprobe hinsichtlich der Bindungsmuster sicher und unsicher der üblichen Verteilung in der Bevölkerung entspricht.

\subsubsection{Eltern nutzen am liebsten das Smartphone während der Betreuung}
Eltern in der Stichprobe bevorzugen die Nutzung des Smartphones während der Betreuung. Über zwei Drittel der befragten Eltern wollen während der Betreuung ihrer Kinder nicht auf das Smartphone verzichten. Darüber hinaus haben über vier Fünftel der Familien zwei solcher Geräte im Haushalt. Ein Zehntel sogar drei und mehr. Damit übertrifft das Smartphone den Computer und den Fernseher. Gemessen an den Medientätigkeiten wird während der Betreuung jedoch am häufigsten Radio gehört, gefolgt von Fernsehen und Video schauen. Die Frage kann aufgeworfen werden, weshalb hier nicht Smartphonetätigkeiten auftreten. Das kann dadurch beantwortet werden, dass sich die erfassten Medientätigkeiten rückwirkend nicht mehr eindeutig einem Gerät zuordnen lassen (siehe auch \textit{\nameref{sec:Methodenkritik}} weiter unten). Am ehesten kann das Schreiben von Textnachrichten als eine Medientätigkeit identifiziert werden, die aussschliesslich auf dem Smartphone ausgeführt wird (auch wenn dies rein technisch ebenso von anderen Geräten aus möglich ist). Dennoch kann anhand der Befunde darauf geschlossen werden, dass die Smartphonenutzung einen hohen Stellenwert während der Betreuung von Kindern einnimmt. Durch das breite Einsatz\-spektrum lassen sich anhand dieser Befunde, neben dem bereits erwähnten Schreiben von Textnachrichten, beliebte Medientätigkeiten wie Telefonieren, im Internet surfen und Musik hören identifizieren. Für künftige Arbeiten könnte es nützlich sein, in der Befragung die einzelnen Medientätigkeiten explizit einem Gerät zuzuordnen, um somit ein Bild der elterlichen Medientätigkeit zu erhalten, das definierter erscheint.  

\subsubsection{Unsicher Gebundene verbringen mehr Zeit mit Textnachrichten}
Die erfasste Mediennutzung der Eltern enthält eine Sammlung von unterschiedlichen Medientätigkeiten, die auf unterschiedlichen Medien ausgeführt werden können. Auf dem Smartphone zum Beispiel können alle die in der Umfrage abgebildeten Tätigkeiten ausgeführt werden. Im Gegensatz zu einem herkömmlichen Buch, mit dem keine Musik gehört und keine Filme geschaut werden können. Wie bereits geschildert, konnte gemäss Hypothese $H1$ kein Zusammenhang zwischen der Bindung und der totalen Mediennutzung festgestellt werden. Werden jedoch die Bindungskategorien \enquote{sicher} und \enquote{unsicher} in Beziehung zu den einzelnen Medientätigkeiten gesetzt, so entsteht zwischen dem Schreiben von Textnachrichten, was vorwiegend auf dem Smartphone erfolgt, und der Bindung ein signifikanter Zusammenhang. Dieser Zusammen\-hang wird grösser, wenn nur die Zeit der Mediennutzung betrachtet wird, während das zu betreuende Kind wach war. Das heisst, dass die Bindung der Eltern durchaus in einem Zusammen\-hang mit der Mediennutzung der Eltern während der Betreuung ihrer Kinder steht - wenn auch nur vorwiegend mit dem Schreiben von Textnachrichten und dies in einem geringen Masse (geringer Effekt gemäss \citeA{Cohen1988a}). Wird davon ausgegangen, dass das Schreiben von Textnachrichten vorwiegend auf dem Smartphone erfolgt, so könnte hier ein generealisierter Zusammenhang zwischen der Smartphonenutzung und der Bindung der Eltern postuliert werden. Leider lässt sich aus den Daten dieser Arbeit keine Smartphonenutzung in Minuten extrahieren, weshalb diese These auf einer Vermutung aufbaut. Für künftige Forschung könnte es interessant sein, die Bindung der Eltern in einen direkten Zusammen\-hang mit der Smartphonenutzung und den darauf ausgeführten Tätigkeiten zu setzen. Insofern lässt sich die Befürchtung von \citeA{Prekop2017}, dass unsicher Gebundene lieber eine Bindung zu technischen Dingen aufbauen, im Zusammenhang mit dem Smartphone und den Befunden dieser Arbeit nicht ganz ausschliessen. Zudem unterstreichen die Ergebnisse dieser Arbeit die Ergebnisse weiterer Studien, welche die Bindungstheorie mit der problematischen Internet\-nutzung (\textit{engl. problematic internet use (PIU)}) in Verbindung brachten \nohyphens{\cite{Jia2016, Chang2015, Lin2011}}. Dabei stellten sie fest, dass die unsichere Bindung als Prädiktor für die problematische Internetnutzung bezeichnet werden kann. An dieser Stelle sei ebenso auf die Masterarbeit von \citeA{Dietziker2018} verwiesen, welche den Zusammenhang der Smartphonenutzung der Eltern auf die Bindungssicherheit ihrer Kinder untersuchte. Dabei konnte die Studie feststellen, dass Eltern mit ambivalent (unsicher) gebundenen Kindern ihr Smartphone im Beisein ihrer Kinder signifikant häufiger nutzten als Eltern sicher gebundener Kinder. Dies beschreibt keinen gerichteten Zusammenhang und es bleibt offen, ob die Eltern wegen ihrer ambivalent gebundenen Kinder das Smartphone mehr nutzen oder ob die häufige Nutzung sich auf die Bindung der Kinder niederschlägt. In diesem Bereich ist weitere Forschung notwendig, da beide Befunde von einem Zusammehang zwischen der Smartphonenutzung und dem unsicheren Bindungsstil ausgehen - sowohl auf Seiten der Eltern als auch auf Seiten der Kinder.

\subsubsection{Nähe, Vertrauen, Freude und Wohlbefinden als Prädiktoren für weniger SMS}
Gemäss der Hypothesenprüfung konnte kein Zusammenhang zwischen der Mediennutzung und der Bindung, bzw. dem subjektiv erlebten Stress gefunden werden. Sobald jedoch die Mediennutzung in die einzelnen Tätigkeiten unterteilt wird, kann ein Zusammenhang zwischen der Bindung und dem Schreiben von Textnachrichten gefunden werden (siehe vorheriges Kapitel). Sobald die einzelnen Skalenausprägungen der Konstrukte Bindung, Stress und subjektives Wohlbefinden in ihre Einzelteile zerlegt werden, können diverse Korrelationen zwischen den einzelnen Medientätigkeiten und den einzelnen Skalenausprägungen gefunden werden. Gemäss \citeA{Schmidt2004} lassen sich in der deutschen Version des \acrshort{aas} die Bindungsstile nicht direkt zuordnen (siehe auch \textit{\nameref{sec:Methodenkritik}}). Somit erscheint es legitim, die einzelnen Skalenausprägungen hinsichtlich der einzelnen ausgeführten Medientätigkeiten zu prüfen. Auffällig bei den Befunden ist, dass das Schreiben von Textnachrichten mit nahezu jeder Skala korreliert (einzelne Ergebnisse siehe \nameref{app:ResultateMedientaetigkeiten} im Anhang). Die Bindungsskalen Nähe und Vertrauen gehen negativ mit dem Schreiben von Textnachrichten während das Kind wach ist einher, die Bindungsskala Angst hingegen positiv. Wird davon ausgegangen, dass Textnachrichten vorwiegend auf dem Smartphone geschrieben werden, scheinen Probanden, die sich mit Nähe und Intimität in Beziehung zu anderen wohl fühlen und darauf vertrauen, dass andere für sie erreichbar sind, während der Betreuung weniger mit dem Smartphone beschäftigt zu sein als solche, die sich mit Nähe und Intimität unwohl fühlen. Zudem scheinen Probanden, die Angst haben, in einer Beziehung nicht geliebt zu werden, einen erhöhten Smartphonegebrauch aufzuweisen. Ebenso scheinen Probanden mit einer tiefen Ausprägung auf der Bindungsskala Vertrauen weniger Textnachrichten zu verfassen während das Kind schläft. Ein erhöhter Smartphonegebrauch scheint mit erhöhten Sorgen, Zukunftsängsten und Frustrationsgefühlen einherzugehen. Probanden mit einem hohen Ausmass an Freude nutzen das Smartphone hingegen eher weniger, um Textnachrichten zu schreiben. Hingegen schreiben Probanden mit einem erhöhten Stresserleben tendenziell mehr Textnachrichten. Bezogen auf das subjektive Wohlbefinden geht ein hohes Wohlbefinden mit einem verringerten Smartphonegebrauch einher. All diese Skalen deuten darauf hin, dass ein schwacher Zusammenhang zwischen dem Schreiben von Textnachrichten und den einzelnen Skalen besteht. Insgesamt gesehen deuten diese Ergebnisse auf einen Zusammenhang zwischen der Bindung und dem Stress zur Mediennutzung hin, explizit dem Smartphonegebrauch. Da es sich jedoch um schwache Effekte handelt, wird der Zusammenhang marginal sein und wurde deshalb auch in den aufgestellten Hypothesen nicht sichtbar. Vielmehr könnte von einer Tendenz in diesem Bereich gesprochen werden. Ebenso kann durch diese Befunde die Richtung des Zusammenhangs nicht bestimmt werden. Es ist also unklar, ob das Schreiben von Textnachtichten sich auf das subjektive Wohlbefinden auswirkt, oder ob ein geringes Wohlbefinden mit einem erhöhten Schreiben von Nachrichten einhergeht.  


\subsubsection{Zusammenfassung Interpretation}
Die oben diskutierten Befunde sollen für eine bessere Übersicht an dieser Stelle nochmals aufgelistet werden.
\begin{itemize}
    \item Die Stichprobe ist weiblich und gut gebildet, was sich protektiv auf die Mediennutzung der Eltern auswirken könnte.
    
    \item Gut gebildete Eltern scheinen Anforderungen während der Betreuung als geringer zu erleben als weniger gut gebildete Eltern.
    
    \item Die Stichprobe scheint sich repräsentativ, gemessen an bestehende Studien, in sicher und unsicher gebundene Probanden zu unterteilen.
    
    \item Eltern der Stichprobe scheinen während der Betreuung ihrer Kinder am liebsten das Smartphone zu nutzen oder Radio zu hören.
    
    \item Unsicher gebundene Eltern scheinen während der Betreuung ihrer Kinder mehr Textnachrichten zu schreiben als sicher gebundene Eltern. 
    
    \item Zudem scheint das Schreiben von Textnachrichten in einem Zusammenhang mit erhöhten Sorgen, weniger Freude, einem erhöhten Stresserleben und einem geringeren subjektiven Wohlbefinden zu stehen. 
\end{itemize}


% ---------------------------------------
\subsection{Methodenkritik} \label{sec:Methodenkritik}
In diesem Kapitel soll auf die in dieser Arbeit verwendete Methode eingegangen  und deren Vorteile sowie Limitation diskutiert werden. Dabei wird auf einzelne Skalen näher eingegangen.

\subsubsection{Vor- und Nachteile eines Onlinefragebogens}
Es gibt verschiedene Vorteile von empirischen Untersuchungen mittels internetbasierten elektronischen Fragebögen \cite{Rey2009}: (1) Die schnelle und kostengünstige Erhebung. Dabei stellte sich die automatische Datenerfassung und die Einsicht in die aktuellen Zwischenergebnisse mittels \citeA{Questback2018} als überaus komfortabel heraus. Durch die Darstellung des aktuellen Teilnehmerstands konnte jederzeit geprüft werden, ob das zu erreichende Soll komplettiert wurde. Dies half bei der vorliegenden Arbeit nach etwa zwei Monaten festzustellen, dass die zu erreichende Anzahl bei Weitem noch nicht erreicht war und weitere Aufwände in der Rekrutierung notwendig machten. (2) Adaptive Struktur der Fragen. Teilnehmer konnten sich nach dem Abschluss des Fragebogens für den Wettbewerb oder Versuchspersonenstunden entscheiden. Dabei wurden diese unterschiedlichen Gruppen zu unterschiedlichen Seiten im Fragebogen geleitet. (3) Zeitliche und örtliche Flexibilität. Da der Fragebogen rund um die Uhr zur Verfügung stand, konnten die Teilnehmer ihn leicht erreichen. Durch diese Flexibilität konnten Teilnehmer den Fragebogen zu einem Zeitpunkt beginnen und zu einem späteren abschliessen. (4) Erfassungsfehler wurden vermieden. Im Gegensatz zu Papierfragebogen konnten gewisse Antwortoptionen bereits bei der Eingabe kontrolliert werden. Zum Beispiel war es bei der Angabe des Jahrgangs für die Teilnehmenden nicht möglich, Buchstaben einzutippen. Sie mussten sich zwischen der Eingabe einer Jahreszahl oder dem Enthalten einer Antwort entscheiden. 

Neben diesen Vorteilen einer Onlinebefragung konnten einige Nachteile identifiziert werden \cite{Rey2009}: (1) Mehrfachteilnahme möglich. Diese Gefahr wurde zu einem gewissen Grad abgefangen, indem die Software sich die Geräte-Adresse der Teilnehmenden merkte und eine erneute Teilnahme vom gleichen Gerät unterband. Wurde das Gerät gewechselt, so hätte erneut an der Umfrage teilgenommen werden können. Für diese Umfrage scheint dieses Risiko jedoch vernachlässigbar, denn der zeitliche Aufwand steht verglichen zu einem möglichen Vorteil in einem denkbar ungünstigen Verhältnis (z. B. Mehrfachteilnahme für die Erhöung der Chancen, am Wettbewerb zu gewinnen). (2) Fragwürdige Stichprobenrepräsentativität. Natürlich mussten die Teilnehmenden über einen Internetzugang verfügen, um an der Befragung teil\-zu\-nehmen. In der Umfrage ging es um die Erhebung der Mediennutzung, um elektronische Medien und um die Nutzung des Internets. Aus diesem Grund scheint die Onlineumfrage ein akzeptables Mittel gewesen zu sein. Eine mögliche Stichproben\-verzerrung kann wohl auf die Rekrutierung der Probanden aus dem Bekanntenkreis des Autors sowie aus der Verbreitung an den entsprechenden Departementen der Hochschule und einschlägigen Elternplattformen herrühren. (3) Problematische Vergleichbarkeit. Das Ausfüllen des Fragebogens benötigte Zeit. Nicht alle Eltern haben diese zur Verfügung. Es könnte durchaus sein, dass Eltern mit einem aktuell hohen Stressempfinden den Fragebogen aus Zeitmangel nicht ausfüllten. Somit hätten nur Eltern an der Studie teilgenommen, die sich die Zeit dazu nehmen konnten. Dies würde sich negativ auf die Forschungsfrage bezüglich des erlebten Stresses der Eltern und dessen Auswirkung auf die Mediennutzung auswirken. Dies kann für die vorliegende Arbeit nicht ausgeschlossen werden, zumal sich der Stressindex der Stichprobe auf einem tiefen Level befand. Die Stichprobe könnte sich demzufolge aus Eltern zusammen\-setzen, die zum Zeitpunkt der Befragung ein geringes Stressempfinden aufgewiesen und deshalb eher an der Umfrage teilgenommen haben. Für die Beantwortung der Fragestellung wäre es jedoch hilfreich gewesen, ebenso Eltern mit einem hohen erlebten Stress in der Stichprobe vertreten zu haben. (4) Hohe Abbruchquote. Die Abbruchquote in dieser Befragung war ziemlich hoch, trotz Anreiz des Wettbewerbs und der Versuchspersonenstunden. Knapp ein Fünftel der Teilnehmenden hat die Umfrage abgeschlossen. Der Rest hat die Umfrage während des Ausfüllens abgebrochen. Durch diese selektiven Abbrüche scheint das Vorliegen einer Zufallsstichprobe in der vorliegenden Arbeit gefährdet zu sein. 

Demgegenüber stehen weitere Vorteile einer Befragung,  unter anderem das Erfassen des subjektiven Erlebens der Teilnehmenden \cite{Berk2005}, welches für die Beantwortung der erhobenen Skalen notwendig war. Zudem konnten in der Befragung diejenigen Konstrukte erhoben werden, die für die Beantwortung der Hypothesen notwendig waren \cite{Berk2005}. Ein Nachteil der Befragung könnte hingegen das Thema Mediennutzung bezüglich der sozialen Erwünschtheit gewesen sein und somit die Antworten verzerrt haben \cite{Rey2012}. Dies könnte geschehen sein, indem der eigenen Medienkonsum während der Betreuung des eigenen Kindes abgeschwächt wurde. Durch die Erfassung der Medienzeit aus der Erinnerung könnte eine weitere Verzerrung aufgetreten sein, da Menschen sich an ihr vergangenes Verhalten und Erleben nicht immer vollständig erinnern \cite{Berk2005}. Eine Voraussetzung für die Teilnahme an der Befragung war eine gewisse Kommunikationsfähigkeit, welche nicht geprüft wurde. Insgesamt betrachtet überwiegen aus Sicht des Autors die Vorteile einer Onlinebefragung. 

\subsubsection{Querschnittsdesign als Mittel der Wahl}
Gemäss \citeA{Berk2005} und \citeA{Trautner1997} weisen Querschnittsstudien zwei Vorteile in der Befragung auf: (1) Ökonomie, da die Ergebnisse der Untersuchung unmittelbar zur Verfügung stehen, denn es muss nur auf einen Untersuchungszeitpunkt zurückgegriffen werden. Da es sich bei dieser Studie um eine Masterarbeit handelt, konnten Personalkosten gespart werden. (2) Bereitschaft zur Teilnahme an der Untersuchung. Die Bereitschaft an einer Studie teilzunehmen scheint grösser zu sein, wenn die Studie nur aus einem Messzeitpunkt besteht. Dadurch können grössere Stichproben rekrutiert werden, was im Hinblick auf die gestellten Hypothesen notwendig war. 

Neben diesen Vorteilen gehen Querschnittstudien gemäss \citeA{Berk2005} und \citeA{Trautner1997} mit einigen Nachteilen einher, auf die hier eingegangen werden soll: (1) Konfundierung von Alter und Kohorte. Die Teilnehmenden sind nicht nur unterschiedlich alt, sondern stammen auch aus unterschiedlichen Kohorten. Somit sind sie in Querschnittsstudien vermischt (konfundiert). Dieser Umstand wird für diese Arbeit als nicht relevant angesehen, da angenommen wird, dass der Zusammenhang von Mediennutzung und Bindung kohortenübergreifend stattfindet. (2) Konstruierte Entwicklungsverläufe. Einzelnen Entwicklungsverläufen wurde in dieser Studie keine Rechnung getragen. Intraindividuelle Veränderungen hinsichtlich der Bindung oder des erlebten Stressniveaus werden nicht berücksichtigt. Diese Studie lässt sich auf Entwicklungszustände reduzieren und kann keine Aussage betreffend einer Entwicklung über die Zeit machen. (3) Statistische Ineffizienz. Systematische Unterschiede zwischen einzelnen Probanden im Verlauf der Messung bleiben unberücksichtigt. Die Vorhersage einzelner Messwerte ist demzufolge ungenau. Um diesem Umstand entgegenzuwirken, wurde in dieser Arbeit mit einer relativ grossen Stichprobe gearbeitet. (4) Selektive Stichprobe. Das Erstellen von Gruppen anhand der Bindung stellt eine selektive Stichprobe dar. Probanden innerhalb der erstellten Gruppe unterscheiden sich nicht nur anhand der Bindung, sondern auch durch weitere Eigenschaften. (5) Fragwürdige Generalisierbarkeit. Die in dieser Arbeit gewonnenen Ergebnisse der Untersuchung hängen vom aktuellen kulturellen, sozialen und politischen Kontext der Umfrage ab. Dadurch lassen sie sich nicht zwangsläufig für künftige Stichproben verallgemeinern. 

Die Durchführbarkeit dieser Arbeit stand im Zentrum, da es um einen ersten Vorstoss in die Richtung der Fragestellung ging. Es wurde eine Bestandesaufnahme angestrebt, auf der weiterführende Forschung betrieben werden kann.

\subsubsection{Reine Smartphonenutzung nicht eindeutig abzugrenzen}
Die Mediennutzungszeit beinhaltete verschiedene Tätigkeiten, die auf unterschiedlichen Medien ausführbar waren. Wie in den Ergebnissen ersichtlich, konnten zwischen der Bindung und dem Schreiben von Textnachrichten statistisch signifikante Werte gefunden werden. Zudem korrelieren die Bindungs- und Stressskalen sowie die Skala für die Erfassung des subjektiven Wohlbefindens vorwiegend mit dem Schreiben von Textnachrichten. Dies deutet darauf hin, dass das Medium, mit dem die Tätigkeit während der Betreuung ausgeführt wurde, eine Rolle spielen könnte. In dieser Studie wurde davon ausgegangen, dass die Teilnehmenden ihre Textnachrichten auf dem Smartphone geschrieben haben und nicht über eine entsprechende Funktion auf dem Tablet oder Desktopcomputer. Dies kann jedoch nur für das Schreiben von Textnachrichten rückgeschlossen werden und auch hier bleibt eine gewisse Unsicherheit. Für die übrigen Tätigkeiten ist dies nicht möglich, da sie gut auf anderen Geräten ausgeführt werden können. Im Nachhinein ist die Zuweisung nicht mehr möglich. Deshalb ist es in Zukunft angebracht, das entsprechende Medium mit der damit verbrachten Zeit, im Besonderen das Smartphone, explizit zu erfassen. Im Bereich der Bindung und der Mediennutzung der Eltern könnte das Smartphone, basierend auf den gefundenen Ergebnissen, eine wichtige Rolle während der Betreuung der Kinder spielen, zumal es als das Gerät in dieser Stichprobe gehandelt wurde, auf das am wenigsten gerne verzichtet wird. Die Popularität ist leicht nachvollziehbar, da alle hier erfassten Medientätigkeiten gut auf diesem Gerät durchgeführt werden können, es handlich genug ist, um es immer dabei\-zu\-haben und es jederzeit genutzt werden kann. 

\subsubsection{Erfassung der Bindung mittels \acrshort{aas}}
Es ist zu hinterfragen, ob das verwendete Instrument, die \acrfull{aas}, für die Erfassung der Bindungstypen geeignet ist. Gemäss \citeA{Schmidt2004} können in der deutschen Fassung keine Zuordnungen gemacht werden. Hier wäre zu untersuchen, ob ein ebenso ökonomisches Instrument gefunden werden kann, welches sich für die Bindungseinteilung besser eignet. In dieser Arbeit scheint das Instrument jedoch plausible Werte zu liefern, weshalb die Verwendung des \acrshort{aas} gerechtfertigt erscheint. 

\subsubsection{Erfasst der \acrshort{psq} das Belastungserleben der Eltern?}
Für die Erfassung des subjektiven aktuellen Belastungserlebens der Eltern wurde das Instrument \acrfull{psq} in der deutschen Version eingesetzt. Im Unterschied zu der von \citeA{Fliege2001} untersuchten Stichprobe konnten in dieser Arbeit keine Gruppenunterschiede bezüglich Geschlecht gefunden werden. Zudem konnten bezogen auf die demografischen Daten keine Effekte gefunden werden. Ob dies damit zusammenhing, dass die vorliegende Stichprobe vorwiegend aus Frauen besteht, wäre zu untersuchen. Bezüglich der ökonomischen Durchführbarkeit scheint das Instrument angebracht gewesen zu sein, um die subjektiv erlebte Belastung der Eltern adäquat zu erfassen.

% ---------------------------------------
\subsection{Ausblick und persönliche Überlegungen} \label{sec:Ausblick}
Gemäss den Hypothesentests konnte kein Zusammenhang zwischen der Bindung, dem erlebten Stress der Eltern und der Medientätigkeit während der Betreuung gefunden werden. Die Medientätigkeit, so wie sie in dieser Studie erfasst wurde, scheint zu breit erfasst worden zu sein. Die Effekte sind, wenn sie vorhanden sind, viel subtiler vorhanden, als dass sie sich über die Gesamtmediennutzung abbilden lassen würden. Zudem scheint die Bindung nicht mit allen Medientätigkeiten im Zusammenhang zu stehen - in dieser Studie nur mit dem Schreiben von Textnachrichten. Kritisch zu hinterfragen ist hier sehr wohl, ob die Bindung der Eltern durch den eingesetzten Fragebogen (\acrshort{aas}) exakt erfasst wurde und ob dies generell mit einem Fragebogen überhaupt möglich ist. In der Psychologie sind Abgrenzungen nicht immer klar zu ziehen. Ebenso zeigen die gefundenen Effekte weder eine Richtung des Zusammenhangs zwischen der Bindung und der Mediennutzung noch einen Einfluss auf die Entwicklungsaufgaben in der Eltern-Kind-Interaktion. Für künftige Forschung wäre es interessant, diese Interaktion im Zusammenhang mit Medien und Bindung näher zu untersuchen. Zudem ist unklar, wie das vermehrte Schreiben von Textnachrichten in Beziehung zu einem unsicheren Bindungsstil steht. Tendieren unsicher Gebundene dazu, sich vermehrt aus der Beziehung zu ihrem Kind zu nehmen indem sie sich mittels Smartphone distanzieren, oder könnte das Smartphone sogar eine Veränderung des Bindungsstils begünstigen? Dass der Bindungsstil sich über die Lebenszeit verändern kann, haben \citeA{Lewis2000} in einer Langzeitstudie untersucht. Dieser Punkt kann durch diese Arbeit nicht beantwortet werden. Vielmehr liefert diese Arbeit eine Anzahl von Indizien, welchen weiter nachgegangen werden könnte.

Weiter wäre zu untersuchen, in welchem Verhältnis die Bindung der Eltern über die Mediennutzung zur Bindung der Kinder steht. Die Medien als \enquote{Übertragungsmittel} der Bindung zum Kind betrachten. Dabei wäre zu untersuchen, ob gewisse Medien in einem Zusammenhang zwischen der Bindung der Eltern und der sich entwickelnden Bindung des Kindes stehen und in welcher Form diese Medien den Bindungsstil und das Bindungsverhalten tangieren. Aus dieser Arbeit lässt sich ein kleiner Zusammenhang zwischen dem Bindungsstil der Eltern auf deren Smartphonegebrauch während der Betreuung herleiten. Zudem scheint es einen Zusammenhang zwischen dem Smartphonegebrauch der Eltern während der Betreuung und dem Bindungsstil der Kinder gemäss \citeA{Dietziker2018} zu geben. Basierend darauf lassen sich nur Vermutungen anstellen. Der Bindungsstil der Eltern könnte sich über die Verwendung gewisser Medien zum Kind übertragen. Es liegt jedoch an fortführenden Untersuchungen, dies zu untersuchen und das Smartphone oder generell Medien als Übertragungsinstrumente der Bindung zu betrachten. Die Bindung scheint gemäss diesen kürzlich gemachten Untersuchungen eine Rolle im Umgang mit dem Smartphone zu spielen.

Basierend auf diesen Überlegungen könnten neue Medien künftig als Projektionsfläche für innere psychologische Konstrukte der Nutzer herangezogen werden - als Mittel, um das innere psychische Konstrukt durch den Umgang sichtbar zu machen, zum Beispiel den Bindungsstil, der während der Betreuung der Kinder durch ein bestimmtes Verhalten sichtbar gemacht würde. Die Nutzung des Smartphones könnte dadurch um einen psychologischen Zusatznutzen erweitert werden. Durch die technisch nahezu grenzenlosen Möglichkeiten heutiger Geräte könnten sich der Psychologie neue Ufer erschliessen. Die Geräte könnten quasi als \enquote{Spiegel der Seele} dienen. Es scheint aber wichtig, das Smartphone nicht als Ursache für ein gewisses Verhalten zu betrachten, sondern als Ausdrucksform einer psychologischen Eigenschaft. Vorstellbar ist bereits heute, das Smartphone therapeutisch einzusetzen (\textit{engl. smartphone-based treatment}). Dies wurde bereits in Studien untersucht (z. B. Prokrastination mit Hilfe des Smartphones verringern \cite{Lukas2018}). Trotz aller Zukunftsvisionen und technischer Möglichkeiten soll dabei nicht vergessen werden, dass in dieser Studie kleine Effekte gefunden wurden. Dass sich diese für allgemeine Aussagen über Persönlichkeitseigenschaften nicht nutzen lassen, ist selbsterklärend. Gut möglich ist aber, dass sich psychologische Eigenschaften durch den Gebrauch von neuen Medien abbilden lassen. In Frage zu stellen ist hingegen, ab damit Voraussagen gemacht werden können. Zu vielfältig ist die Psyche, als dass sie sich so einfach durch Technik abbilden lassen würde.  

Schön wäre es, wenn die Ergebnisse aus dieser Studie zu weiterführenden Untersuchungen führen würden. Dem Autor haben sie auf alle Fälle Hinweise auf das eigene Verhalten, aber auch auf das Verhalten anderer Eltern gegeben. Die Allgegenwärtigkeit und einfache Nutzbarkeit der Medien wird weitere spannende Erkenntnisse zum menschlichen Verhalten liefern können. Schön wäre es, wenn dies als Chance für Entwicklung und Erkenntnis der Menschheit genutzt würde.

\newpage