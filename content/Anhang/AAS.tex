Die im Folgenden dargestellte Adult Attachment Sacale (AAS) wurde aus dem Paper von \citeA{Schmidt2004} übernommen. Die Spalte \enquote{Orig.} in Tabelle \titleref{table:AAS} beinhaltet die originale Nummerierung der Fragebogenitems. Die Spalte \enquote{Nr.} beinhaltet die neue Nummerierung, so wie sie im Umfragebogen (siehe auch \titleref{app:Fragebogen}) verwendet wurde.


\begin{table}[htbp]
\begin{tabular}{|p{2em} m{30em}  l|} 
  \hline
  \textbf{Orig.} & \textbf{Text} & \textbf{Nr.} \\ 
  \hline\hline
  \rowcolor{lightgray}
  \multicolumn{3}{|l|}{Nähe*}\\
  \hline
  3 & Es macht mich nervös, wenn mir jemand zu nahe ist. & 2 \\
  8 & Für mich ist es schwierig, andere an mich heranzulassen. & 7 \\
  13 & Es ist mir irgendwie unangenehm, mit anderen zu vertraut zu werden. & 11 \\
  14 & In Freundschaften wünschen sich meine Freunde/meine Freundinnen häufig mehr Nähe von mir, als mir angenehm ist. & 12 \\
  18 & Die Vorstellung, mir könnte jemand nahe kommen, beunruhigt mich. & 16 \\
  \hline
  \rowcolor{lightgray} \multicolumn{3}{|l|}{Vertrauen}\\
  \hline
  1 & Ich weiß, wenn ich jemand brauche, wird auch jemand da sein. & 1 \\
  5 & Ich bin mir nicht sicher, ob ich mich immer darauf verlassen kann, dass andere da sind, wenn ich sie brauche. & 4\\
  10 & Menschen sind nie da, wenn man sie braucht. & 8 \\
  12 & Ich kann mich gut auf andere verlassen. & 10\\
  15 & Ich kann es mir nur schwer zugestehen, mich auf andere zu verlassen. & 13\\
  17 & Es fällt mir schwer, anderen voll und ganz zu vertrauen. & 15\\
  \hline
  \rowcolor{lightgray} \multicolumn{3}{|l|}{Angst}\\
  \hline
  4 & Ich mache mir oft Sorgen, dass meine Freunde/meine Freundinnen mich nicht wirklich mögen. & 3\\
  6 & Mein Wunsch, in einem anderen Menschen völlig aufzugehen, schreckt andere manchmal ab. & 5 \\
  7 & Ich merke, dass andere mich nicht so nahe an sich herankommen lassen, wie ich es gerne hätte. & 6 \\
  11 & Ich mache mir oft Sorgen, ein wichtiger Mensch könnte mich verlassen. & 9 \\
  16 & Ich mache mir oft Sorgen, dass meine Freunde/meine Freundinnen eines Tages nicht mehr mit mir befreundet sein möchten. & 14 \\
  \hline
  \multicolumn{3}{l}{* Diese Skala wurde bei der Berechnung des Mittelwerts invertiert,}\\
  \multicolumn{3}{l}{~~~damit \textit{Nähe} eine positive Ausprägung bekommt.}\\
  \multicolumn{3}{l}{}\\
\end{tabular}
\caption{Items der Adult Attachment Scale}
\label{table:AAS}
\end{table}
