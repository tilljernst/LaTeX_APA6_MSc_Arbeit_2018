Das im Folgenden abgebildete Infomail diente der Rekrutierung der Stichprobe. Es handelt sich dabei um ein exemplarisches Beispiel, welches an die deutschsprachigen Elternberatungsstellen versendet wurde.

\begin{flushleft}

\textbf{Betreff:}

Auslage und Publikation für eine Eltern-Umfrage im Bereich Mediennutzung der ZHAW
\vspace{2mm}

\textbf{Inhalt:}

Sehr geehrte Damen und Herren

\vspace{2mm}
Aktuell schreibe ich eine Masterarbeit in angewandter Psychologie an der Zürcher Hochschule für Angewandte Wissenschaften (ZHAW) im Thema: Mediennutzung der Eltern im Beisein ihrer 0-1 Jahre alter Kinder.
Darin möchte ich untersuchen, ob das Medienverhalten der Eltern eine Auswirkung auf das Wohlbefinden der Eltern hat und ob das eigene Bindungsverhalten der Eltern und der aktuell empfundene Stress sich auf das Medienverhalten auswirken. 

\vspace{2mm}
Für meine Untersuchung benötige ich mindestens 280 Teilnehmer (je mehr, desto aussagekräftiger das Resultat). Da der Fragebogen für die Eltern mit etwas Aufwand und 10 bis 15 Minuten ihrer Zeit verbunden ist (alles was in dieser delikaten Zeit Mangelware ist), habe ich mittels Wettbewerb einen gewissen Anreiz für das Beantworten der Fragen geschafft (10 x 50.- bei Baby-Rose).

\vspace{2mm}
Umfragelink: https://ww2.unipark.de/uc/elternfragebogen/

\vspace{2mm}
Meine Bitte an Sie: Könnten Sie sich vorstellen diese Umfrage zu unterstützen, indem Sie diese Umfrage auf Ihrer Webseite publizieren oder Flyer in Ihrem Aushang auflegen? Dadurch würden Sie mir eine grosse Hilfe erweisen. Die Flyer würde ich Ihnen per Post zukommen lassen. Bei einer Publikation auf Ihrer Webseite ist der Medienbruch nicht so hoch (Wechsel von Analogmedium auf ein Digitales) und dadurch die Rücklaufquote etwas verbessert.
Die Umfrage wird bis Ende Mai aufgeschaltet sein.

\vspace{2mm}
Im Anhang habe ich Ihnen meine Disposition zu dieser Arbeit beigelegt, falls Sie zusätzliche Informationen zur Arbeit benötigen. Zudem habe ich Ihnen den Flyer beigelegt, den ich an diversen weiteren Fachstellen auflegen werde (Familienzentren, Mütter und Väter Beratung, etc.). 
Sollten Sie weitere Fragen haben, so werde ich Ihnen diese gerne beantworten.

\vspace{2mm}
Auf eine Rückmeldung von Ihnen freue ich mich sehr.

\vspace{2mm}
Mit besten Grüssen 

Till ERNST

Stud. MSc in angewandter klinischer Psychologie

\end{flushleft}

Zusätzlich zum Schreiben wurde dem Mailing ein Flyer angehängt. Dieser wurde auch in Papierform ausgedruckt und aufgelegt.


\begin{figure}[h]
  \centering
     \includegraphics[scale=0.9,angle=90]{content/Grafik/Flyer_Umfrage_v2.pdf}
  \caption{Flyer für die Umfrage}
  \label{fig:flyer}
\end{figure}

\newpage

Für die Auswertung des Wettbewerbs wurden die Teilnehmer randomisiert gezogen. Unten aufgeführt ist das dazu verwendete SWIFT-Skript, die gezogenen Teilnehmer anhand ihrer ID und das versendete Gewinnerschreiben.

\subsection*{Swift 4.0 Code für die randomiesierte Ziehung}

\begin{lstlisting}
import UIKit
for i in 0 ..< 10 {
    print ("Gewinner \(i)", arc4random_uniform(159))
}
\end{lstlisting}

Dabei ist zu beachten dass $N$=159 beim Wettbewerb mitgemacht haben.

\subsection*{Gewinner anhand Personen-ID}
\begin{itemize}
    \item Gewinner 0: 20
    \item Gewinner 1: 147
    \item Gewinner 2: 124
    \item Gewinner 3: 4
    \item Gewinner 4: 114
    \item Gewinner 5: 35
    \item Gewinner 6: 86
    \item Gewinner 7: 106
    \item Gewinner 8: 64
    \item Gewinner 9: 154
\end{itemize}

\subsection*{Mailing an Gewinner}
\begin{flushleft}
\textit{Betreff:}
Sie haben bei der Elternumfrage gewonnen

\textit{Inhalt:}
Liebe Gewinnerin, lieber Gewinner

Vor einiger Zeit haben Sie an der Elternumfrage meiner Masterarbeit der ZHAW teilgenommen. Vielen Dank dafür. 

Sie wurden als glückliche Gewinnerin/Gewinner ausgewählt und erhalten einen Gutschein im Wert von CHF 50.- des online Shops Baby-Rose.

Dazu bitte ich Sie, mir Ihre vollständige Postanschrift zukommen zu lassen, damit ich Ihnen via Baby-Rose den Gutschein aushändigen kann.

Vielen Dank und beste Grüsse

Till ERNST
\end{flushleft}