
\begin{table}[htbp]
\centering
\captionsetup{margin=5pt,skip=5pt}
\caption{Items des Perceived Stress Questionnaire}
\label{table:PSQ}
\begin{tabular}{|p{2em} | m{30em} | l|} 
  \hline
  \multicolumn{3}{|c|}{\textbf{Perceived Stress Questionnaire**}}\\
  \hline
  Nr. & Text & Orig. \\ 
  \hline\hline
  \rowcolor{lightgray}
  \multicolumn{3}{|l|}{Sorgen (worries)}\\
  \hline
  05 & Sie fürchten Ihre Ziele nicht erreichen zu können. & 09\\
  07 & Sie fühlen sich frustriert. & 12\\
  10 & Ihre Probleme scheinen sich aufzutürmen. & 15\\
  13 & Sie haben viele Sorgen. & 18\\
  15 & Sie haben Angst vor der Zukunft. & 22\\
  \rowcolor{lightgray}
  \multicolumn{3}{|l|}{Anspannung (tension)}\\
  \hline
  01* & Sie fühlen sich ausgeruht. & 01*\\
  06* & Sie fühlen sich ruhig. & 10*\\
  09 & Sie fühlen sich angespannt. & 14\\
  17 & Sie fühlen sich mental erschöpft. & 26\\
  18 & Sie haben Probleme, sich zu entspannen. & 27\\
  \rowcolor{lightgray}
  \multicolumn{3}{|l|}{Freude (joy)}\\
  \hline
  04 & Sie haben das Gefühl, Dinge zu tun, die Sie wirklich mögen. & 07\\
  08 & Sie sind voller Energie. & 13\\
  12 & Sie fühlen sich sicher und geschützt. & 17\\
  14 & Sie haben Spass. & 21\\
  16 & Sie sind leichten Herzens. & 25\\
  \rowcolor{lightgray}
  \multicolumn{3}{|l|}{Anforderungen (demands)}\\
  \hline
  02 & Sie haben das Gefühl, dass zu viele Forderungen an Sie gestellt werden. & 02\\
  03 & Sie haben viel zu tun. & 04\\
  11 & Sie fühlen sich gehetzt. & 16\\
  19* & Sie haben genug Zeit für sich. & 29*\\
  20 & Sie fühlen sich unter Termindruck. & 30\\
  \hline
  \multicolumn{3}{l}{* Bei diesen Items muss der Wert des Items von 5 abgezogen werden,}\\
  \multicolumn{3}{l}{~~~bevor das Ergebnis mit den Werten der anderen Items in die Berechnung eingeht.}\\
  \multicolumn{3}{l}{** Deutsche Übersetzung gemäss \citeA{Fliege2001}}.\\
\end{tabular}
\end{table}