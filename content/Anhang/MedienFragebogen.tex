Der im Folgenden dargestellte Medien und Mediennutzung enthält die Items, die in der Onine-Umfrage verwendet wurden. 

Verwendete Medien während der Betreuung: \begin{seriate}
  \item Smartphone
  \item Desktop / Laptop Computer
  \item Tablet
  \item Radio / Stereoanlage / CD-Player
  \item Printmedien (Buch / Zeitung / Heft / Comic)
  \item Foto- und / oder Videokamera
  \item Spielekonsole
  \item MP3-Player
\end{seriate}.

Medientätigkeiten in Minuten während der letzten Betreuung (Unterscheidung, ob das Kind wach war oder geschlafen hat):
\begin{seriate}
  \item Telefonieren
  \item Textnachrichten lesen und schreiben (SMS, Whatsapp, etc.)
  \item Emails lesen und bearbeiten
  \item Lesen / anschauen von Printmedien (Bilder-Bücher)
  \item Internet nutzen (surfen, Informationen suchen, etc.)
  \item Musik hören
  \item Fernsehen und Videos schauen (Sender, Netflix, DVD, etc.)
  \item Radio hören
  \item Fotos oder Videos machen
  \item Video-Games spielen (Handy, Computer, Konsole, etc.)
  \item Hörspiele oder Hörbücher hören
\end{seriate}.

Anzahl Geräte und Medien im Haushalt (Auswahl zwischen O, 1, 2, 3 oder mehr Geräte):
\begin{seriate}
  \item Smartphone
  \item Desktop / Laptop Computer
  \item Tablet
  \item Radio / Stereoanlage / CD-Player
  \item Abo Tageszeitung / Zeitschrift
  \item Foto- Videokamera
  \item Spielekonsole
  \item MP3 Player
\end{seriate}.


Frage nach dem Medium, auf das während der Betreuung nicht verzichtet werden kann (maximale Auswahl von drei Geräten): 
\begin{seriate}
  \item Smartphone
  \item Desktop / Laptop Computer
  \item Tablet
  \item Radio / Stereoanlage / CD-Player
  \item Abo Tageszeitung / Zeitschrift
  \item Foto- oder Videokamera
  \item Spielekonsole
  \item MP3 Player
\end{seriate}.
