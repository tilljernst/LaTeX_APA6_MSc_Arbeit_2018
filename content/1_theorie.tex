% ---------------------------------------
\subsection{Theoretische Überlegungen anhand der Eltern-Kind-Beziehung}\label{sec:TheretischeÜberlegungen}
Die Eltern-Kind-Beziehung bezeichnet das aufeinander bezogene, gegenseitig bindende Verhaltensrepertoire zwischen dem Kind und den erwachsenen Menschen, welche die Elternrolle übernommen haben \cite{ElternKindBeziehung1999}. Ursprünglich als Mutter-Kind-Beziehung, wurde dies aufgrund wissenschaftlicher Ergebnisse zum Begriff Eltern-Kind-Beziehung erweitert, da auch ein konstanter, zuverlässiger Vater als Bezugsperson für das Kind dienen kann. Neben der Bindung zu den Hauptbezugspersonen können Kinder individualisierte Bindung in abgestufter Intensität zu andern Mitgliedern der Familie oder Sozialgruppe aufbauen. Dabei muss die Gelegenheit zu regelmässigen Zwiegesprächen gegeben sein, wodurch durch die Anwesenheit und Zuspruch der vertraut gewordenen Menschen die innere Sicherheit, Geborgenheit und Angstfreiheit beim Kind. Durch Kontinuität und Zuverlässigkeit in der liebevollen Betreuung entsteht eine sichere Bindung \cite{ElternKindBeziehung1999}. Gemäss \citeA{Wirtz2013} bezeichnet die Eltern-Kind-Beziehung verschiedene Aspekte des Verhältnisses zwischen Eltern und Kindern. Dabei wird zwischen Struktur- und Prozessmerkmale unterschieden. In dieser Arbeit stehen vor allem die Prozessmerkmale im Fokus, die sich auf die Qualität des Verhältnisses zwischen Eltern und Kindern bezieht. Wo sich hingegen die Strukturmerkmale auf verwandtschaftliche oder gesetzliche Verhältnisse zwischen Eltern und Kinder beziehen. Die Qualität der Beziehung kann unter der systemischen Perspektive betrachtet werden, in dem der Familienprozess als das Ergebnis wechselseitiger Beeinflussung aller Familienmitglieder betrachtet wird. Charakteristisch dafür sind Offenheit, Wechselseitigkeit, Harmonie, Streitkultur oder Grenzen. Des weiteren kann das spezifische Generationenverhältnis zwischen Eltern und Kindern betrachtet werden, wobei der Fokus meist auf der Erziehung und deren Erziehungsstile gesetzt wird. 

Der Übergang von einer Paargemeinschaft zur Elternschaft führt zu einer grundlegenden Umstrukturierung von einer Zweier- zu einer Dreierbeziehung \cite{Hofer1992, Buergin1998}. Dies gelte als eine zentrale Entwicklungsaufgabe von Familien, wobei die Eltern-Kind-Beziehung nicht als einseitige Einflussnahme zu verstehen ist, sondern die Eltern und die Kinder in einer reziproken sozialisatorischen Beziehung zueinander stehen. \citeA{Hofer1992} nennt wichtige Einflussgrössen der dyadischen Beziehung zwischen Vater, Mutter und Kind, die sich auf die kindliche Entwicklung auswirken können: Dies ist die Synchronizität der Interaktion, die Persönlichkeit der Eltern, die Beziehungsqualität der Partner, sowie die affektive und kognitive Ausstattung des Kindes selbst. 

\begin{figure}%[htbp] -> am Schluss fürs Feintuning :-)
  \centering
     \includegraphics[width=1.0\textwidth]{content/Grafik/Infografik_ElternKindBeziehung_Uebersicht.png}
  \caption{Infografik: Eltern-Kind-Beziehung}
  \label{fig:InfografikElternKindBeziehung}
\end{figure}

% ---------------------------------------
\subsubsection{Eltern-Kind-Interaktion}\label{sec:Interaktion}
Grundsätzlich ist der Mensch gemäss \citeA[S.~91]{Resch1999} in ein Gefüge von zwischenmenschlichen Relationen eingebettet, welche einen wesentlichen Einfluss auf das innerer Weltverständnis und das Selbstbild nehmen.  Die frühe Kindheit stellt eine kritische Periode in der Entwicklung dar, wobei die wichtigste frühe Umgebung des Kindes die Familie, resp. das primäre Umfeld, ist (vgl. ebd., S. 92ff.). Wird die Familie im Entwicklungskontext angeschaut, so kann der elterliche Einfluss auf das Kind durch die zwei Begriffe \textit{Beziehung} und \textit{Erziehung} definiert werden. Dadurch weist die Eltern-Kind-Beziehung immer Beziehungsqualitäten und Erziehungsqualitäten auf.  

Auch \textit{kindliche Faktoren} beeinflussen die frühen Interaktionen deutlich. Das Kind ist schon sehr früh in der Lage, selektiv und adäquat auf die emotionale Ausdrucksform der Bezugsperson einzugehen \cite{Harris1994}. Dabei haben Kinder im Säuglingsalter von der Bezugsperson die Erwartung, dass diese adäquat auf die Gefühlsausdrücke reagiert. Dies ermöglicht bereits dem Säugling einen Informationsaustausch zu seiner Bezugsperson. Somit ist ein Säugling in der Lage, sich anhand dem emotionalen Ausdruck der Bezugsperson anzunähern oder es bleiben zu lassen \cite{Resch1999}. Dieser Vorgang wird als \textit{soziale Vergewisserung (engl. social referencing)} bezeichnet und meint damit, dass ein Kind nach der Beantwortung seiner affektiven Gestimmtheit sucht, wenn es sich seiner Bezugsperson zuwendet. Es konsultiert den Gesichtsausdruck der Bezugsperson, um über die Bedeutung eines ihm unbekannten Ereignisses eine Sinnzuschreibung \textit{(engl. meaning attribution)} zu erhalten. Diese frühen affektiven Regulatoren der Interaktion dienen der Entwicklung des kindlichen Interessens und des kindlichen Bewertungsschemas. Je nach emotionalem Austausch wird das Kind eine Kohärenz in seinen Erwartungen entwickeln oder nicht. Wenn die kindlichen emotionalen Signale adäquat beantwortet werden, entsteht eine günstige Basis des Erfahrungslernens \cite[S.~95]{Resch1999}. Die emotionale Kommunikation konstituiert das kindliche Weltbild in Form von Selbst- und Objektrepräsentanzen. Inkonsistente emotionale Kommunikation führt zu einem weniger köhärenten und weniger vorhersagbaren inneren Schemata der Welt. 

Ein weiterer Interaktionsfaktor stellt die \textit{Induktion} aus der empirischen Säuglingsforschung dar. Kindliche Gefühlszustände und Verhaltensmuster werden durch elterliche Verhaltensmuster induziert \cite{Cummings1994}. Intrusive feindliche und unsensitive Elternreaktionen können einen negativen Erregungsprozess im Kind auslösen, woraus sich ungünstige Interferenzen mit den sich entwickelnden Fähigkeiten des Kindes ergeben, seine eigenen Erregungsimpulse zu modulieren und zu regulieren.  

Über Imitationsprozesse kann die \textit{Modellbildung} im Kind durch das elterliche Verhalten beeinflusst werden. Elterliche Verhaltensweisen und Sichtweisen gegenüber der Welt werden internalisiert \cite{Resch1999}. Die Eltern beeinflussen die Entwicklung des Kindes nicht nur als direkte Akteure, sondern eben auch durch ihre Vorbildfunktion. 

Durch die frühe Interaktion mit einer Bezugsperson wird auch die interaktionelle \textit{Affektregulation} geprägt. Nach \citeA{Dornes1993, Dornes1997} werden in den frühen Interaktionen zwischen Mutter und Kind affekthaltige Handlungen ausgetauscht. Das Kind kann seine innere Erfahrung mit einem anderen Menschen teilen und mit diesem darüber Kommunizieren. Eine Form dieser affektiven Kommunikation wird gemäss \citeA{Stern1985} als \textit{Affektabstimmung (engl. affective attunement)} bezeichnet. Dabei wird auf bestimmte Gefühlsäusserungen des Kindes von Seiten der Mutter differenziert geantwortet, wobei die Antwort etwas stärker oder schwächer als die des Kindes ausfällt. Durch gezieltes Abdämpfen oder Stimulieren der kindlichen Gefühlsausdrücke und Aktionen können Erlebnis- und Handlungsfolgen akzentuiert, abgedämpft oder sogar gelöscht werden. Reagiert eine Bezugsperson unsensibel auf das Kind und werden Handlungsintentionen immer wieder unterbrochen, so kann sich dies negativ auf die Entwicklung des Kindes auswirken \cite{Resch1999}.

Die wechselseitige Bedingtheit des Verhaltens von Menschen wird \textit{Kontingenz} genannt. Gemeint ist dabei die regelhafte Aufeinanderfolge einzelnen Verhaltensschritte. Die Wichtigkeit kontingenter Beantwortung kindlicher Signale, vor allem in der Anfangsphase der Eltern-Kind-Beziehung, betont \citeA{Papousek1987, Papousek1989}. Kontingenz bedeutet dabei die zeitliche und inhaltliche Passung der elterlichen Reaktion auf das Verhalten des Kindes, also die zeitliche Folge der Reaktion und die inhaltliche Entsprechung. Diese frühe Verständigung zwischen Eltern und Kind würde nicht gelingen, wenn die Eltern nicht kompensatorisch an die begrenzte kindlichen Voraussetzungen anpassen würden. Diese Fähigkeit schafft die Voraussetzung für eine gelingende Kommunikation mit dem Säugling im vorsprachlichen Alter. Kontingentes Antworten von Seiten der Eltern ermöglicht eine Kohärenz der Interaktion, die zum subjektiven Gefühl der Kontrolle beim Kind führt.

\textit{Erziehungsstile} sind Verhaltensweisen und Einstellungen, die das emotionale Klima für die Eltern-Kind-Interaktion bestimmen \cite[S.~649]{Siegler2008}. Um den Einfluss den Eltern auf die Entwicklung von Kindern nehmen können, wurden zwei Dimensionen des Erziehungsstils identifiziert: (1) das Ausmass an elterlichen Wärme, Unterstützung und Akzeptanz, und (2) das Ausmass an elterlicher Kontrolle und Anforderung \cite{Maccoby1983}. Diese Aspekte der der elterlichen Erziehung scheinen eine wichtige Rolle bei der Ausbildung interindividueller Unterschiede zwischen Kindern zu spielen. \citeA{Baumrind1973} entwickelte anhand der Dimensionen Unterstützung und Kontrolle vier Erziehungsstile: autoritativ, autoritär, permissiv und vernachlässigend-zurückweisend \cite{Baumrind1991}. Nach dieser Typologie neigen \textbf{autoritative} Eltern dazu Anforderungen zu stellen, aber auch auf das Kind einzugehen und warmherzig zu sein. Gemäss \citeA{Baumrind1991} sind Kinder autoritativer Eltern häufig kompetent, selbstbewusst und bei ihren Altersgenossen beliebt. Als Jugendliche besitzen sie häufig hohe soziale und schulische Fähigkeiten, Selbstvertrauen und positive Verhaltensweisen \cite{Lamborn1991}. \textbf{Autoritäre} Eltern sind oft kalt und reagieren nicht auf die Bedürfnisse ihrer Kinder. Des Weiteren üben sie starke Kontrolle aus und stellen hohe Anforderungen. Forderungen an die Kinder erzwingen sie oft durch Ausüben ihrer elterlichen Macht. Kinder autoritärer Eltern besitzen oft relativ geringe soziale und schulische Kompetenzen, sind unglücklich und unfreundlich und besitzen ein geringes Selbstvertrauen. Im Jugendalter wirkt sich dies auf eine geringere soziale und schulische Kompetenz im Vergleich zu Kindern autoritätiver Eltern aus \cite{Baumrind1991, Lamborn1991}.  \textbf{Permissive} Eltern reagieren auf die Bedürfnisse und Wünsche ihrer Kinder und sind nachsichtig. Sie sind nicht konservativ und verlangen von ihren Kindern nicht sich selbst zu regulieren. Kinder permissiver Eltern sind häufig impulsiv, es fehlt ihnen an Selbstbeherrschung und zeigen oft schwache schulische Leistungen \cite{Baumrind1991}. Als Jugendliche fallen sie durch häufigen Drogenkonsum und schlechtes Benehmen gegenüber Kindern autoritativer Eltern auf \cite{Lamborn1991}. \textbf{Zurückweisend-vernachlässigende} Eltern sind uninteressierte Eltern, die wenig Anforderungen stellen und auch wenig auf sie reagieren. Sie setzen dem Verhalten der Kinder keine Grenzen und kontrollieren es auch nicht. Säuglinge und Kleinkinder zurückweisend-vernachlässigender Eltern weisen häufig gestörte Bindungsbeziehungen und Probleme in der Beziehungsgestaltung zu ihren Altersgenossen auf \cite{Parke1998}.


\subsubsection{Entwicklungsaufgaben der Kinder}\label{sec:Entwicklungsaufgaben} Über die Enticklungsaufgaben der Kinder lassen sich ganze Bücher schreiben. Es sind deren zahlreiche geschrieben worden und der Rahmen dieser Arbeit würde gesprengt werden, müssten hier alle Ansätze und Theorien Platz finden. Aus diesem Grund wurde eine Auswahl getroffen, die keinen Anspruch auf Vollständigkeit erhebt und doch die scheinbar relevantesten im Bereich der Eltern-Kind-Beziehung unter dem Fokus der Interaktion näher beleuchtet. Dadurch soll dem Anspruch genüge getan werden, wichtige theoretische Inputs für die Klärung der Forschungsfrage dieser Arbeit zu liefern. Zum Beispiel wird die körperliche Entwicklung, wenn auch eine wichtige Aufgabe eines Kindes, hier nicht weiter behandelt, da die Schnittpunkte zum Forschungsthema zu gering ausfallen.

\paragraph{Kognitiven Entwicklung}\label{par:KognitiveEntwicklung}
Bei der \textit{kognitiven Entwicklung} spielen die sozio-kulturellen Theorien eine wichtige Rolle im Eltern-Kind-Gefüge. Dabei handelt es sich um Theorien, die die Bedeutung der Interaktion zwischen Kindern und andern Menschen zusammen mit der umgebenden Kultur ins Zentrum für die Entwicklung der Kinder rücken \cite[S.~225ff]{Siegler2008}. Als Beispiel für eine sozio-kulturelle Theorie kann die \textit{gelenkte Partizipation} genommen werden. Darunter wird ein Prozess verstanden, bei dem als Experten geltende Erwachsene ihre Aktivität so gestalten, dass sich Menschen mit geringeren Kenntnissen - in diesem Fall die Kinder - an diesen beteiligen können, obwohl sie von sich aus dazu noch nicht fähig sind und daraus etwas lernen können \cite{Rogoff1990}. Die gelenkte Partizipation dient oft der Erreichung eines praktischen Ziels, wie dem Zusammenbauen eines Autos. Das Lernen ist dabei ein Nebenprodukt der Tätigkeit.

Weiter ist die Interaktion zwischen den Eltern und dem Kind in der Einbettung des kulturellen Kontext für die Entwicklung wichtig. Es gehören nicht nur andere Menschen dazu, sondern auch die zahlreichen Produkte der menschlichen Erfindungskraft, die als sogenannte Kulturwerkzeuge bezeichnet werden. Dazu gelten Symbolsysteme, Artefakte, Fähigkeiten und Werte \cite[S.~226]{Siegler2008}. Gemäss Wygotski, ein russischer Psychologe und Zeitgenosse von Piaget, sind Kinder soziale Wesen, geformt durch ihren kulturellen Kontext, den sie auch selbst wiederum mitgestalten (vgl. ebd. S.~227). 

Als Grundlage der kognitiven Entwicklung des Menschen gelten gemäss sozio-kulturellen Theoretikern in der Fähigkeit, \textit{Intersubjektivität} herzustellen. Das wechselseitige Verständnis, das Menschen bei der Kommunikation füreinander aufbringen \cite{Gauvain2001, Rogoff1990}. Dabei gilt, dass für eine effektive Verständigung, sich die Beteiligten auf dasselbe Thema konzentrieren und ebenso auf die Reaktion des andern achten müssen. Dazu ist ein wirksames Lehren und Lernen unverzichtbar. Zwei bis drei monatige Säuglingen sind lebhafter und interessierter, wenn die Bezugsperson auf ihre Aktionen reagiert \cite{Murray1985}. Sechs monatige Kinder können neue Verhaltensweisen allein durch die Beobachtung des Verhalten anderer Menschen erlernen \cite{Collie1999}. Im Zentrum der Intersubjektivität steht die \textit{geteilte Aufmerksamkeit (engl. joint attention)}, welches ein Prozess beschreibt, bei dem die sozialen Partner ihre Aufmerksamkeit intendiert auf einen gemeinsamen Gegenstand richten in der äusseren Umwelt. Zum Beispiel schauen die Kinder im Alter zwischen neun und 15 Monaten zunehmend auf dieselben Gegenstände wie ihre Sozialpartner und richten die Aufmerksamkeit von Erwachsenen aktiv auf Objekte, die sie selber interessieren \cite{Adamson1991, Gauvain2001}. Diese gemeinsame Aufmerksamkeit spielt eine wichtige Rolle bei der frühen Sprachentwicklung \cite[S.~232]{Berk2011}. Eine verwandte Verhaltensweise (\prettyref{sec:Interaktion}) ist das \textit{soziale Referenzieren (engl. social referencing)}, ein Verhalten, um vom Sozialpartner einen Rat oder eine Anleitung für eine unbekannte oder bedrohliche Situation zu bekommen, indem dieser angeschaut wird \cite{Campos1981}. 

Das frühe Umfeld ist für die Entwicklung des Kindes existentiell wichtig. \citeA{Fuligni2004, Linver2004, TamisLeMonda2004} konnten aufzeigen, dass Zuneigung, Engagement sowie Ermutigung der Eltern bei den Säuglingen und Kleinkinder bessere Ergebnisse bei Sprach- und Intelligenztests hervorrufen. Das Ausmass, indem Eltern mit ihren Säuglingen sprechen, trägt erheblich zum frühen Fortschritt des Spracherwerbs bei, was wiederum förderlich für die Intelligenz und gute Schulleistungen in der Grundschule ist \cite{Hart1995}. Warmherziges, einfühlsames elterliches Verhalten ist ein guter Indikator, wie das Kind kognitiv später abschneiden wird \cite[S.~225]{Berk2011}. Der Betreuung steht demzufolge ein grosser Anteil der Entwicklungsarbeit zu. Die Forschung konnte aufzeigen, dass schlecht betreute Kinder, in kognitiven und sozialen Fertigkeiten schlechter abschneiden \cite{NICHD2006}.

\paragraph{Soziale Entwicklung}\label{par:SozialeEntwicklung}
Theorien sozialer Entwicklung versuchen zu erklären, wie Menschen und soziale Institutionen in der Umwelt von Kindern auf ihre Entwicklung einwirken \cite[S.~470ff]{Siegler2008}. Diese Theorien sind auf die Erklärung vieler wichtiger Entwicklungsaspekte gerichtet, wie zum Beispiel Emotion, Persönlichkeit, Bindung, Selbst, Beziehung zu Gleichatligen, Moral oder Geschlecht. Bei den psychoanalytischen Theorien von Freud und Erikson wird die soziale Entwicklung  durch die biologische Reifung betont. Für Freud ist das Verhalten durch die Befriedigung von grundlegenden Trieben definiert, die weitgehend unbewusst ablaufen. In der Theorie von Erikson wird die Entwicklung durch eine reihe altersbezogener Entwicklungskrisen, resp. Entwicklungsaufgaben \hyphenation{vor-an-ge-trieben} (vgl. ebd. S.~472). Die Bedeutung dieser beider Theorien liegen auf der Betonung der Wichtigkeit früher Lebenserfahrungen und der emotionalen Beziehung, sowie die Suche nach Identität. Erikson betonte, dass für eine gesunde Entwicklung nicht die Quantität der Zuwendung zwischen Kind und Bezugsperson ausschlaggebend ist, sondern vielmehr die Qualität \cite[S.~243]{Berk2011}. Nämlich dass sich zum Beispiel die Mutter unverzüglich und einfühlend um ihr Kind kümmert und sich durch diesen liebevollen Umgang der psychische Konflikt zwischen Urvertrauen versus Misstrauen positive lösen lässt. Zudem betont Erikson, dass der Konflikt des Kleinkindes betreffend Autonomie versus Scham und Zweifel positiv aufgelöst wird, wenn Eltern ihrem Kind klare Grenzen aufzeigen und genügend Möglichkeiten zur Entfaltung bieten. 

Im Gebiet der Lerntheorien soll für diese Arbeit vor allem die Theorie des \textit{sozialen Lernens} hervorgehoben werden. Diese versucht die Persönlichkeit und andere Aspekte der sozialen Entwicklung anhand von Lernmechanismen zu erklären, wobei die Beobachtung und Nachahmung als Mechanismen im Zentrum stehen. \citeA{Bandura1979, Bandura1986} gibt an, dass der grösste Teil des menschlichen Lernens sozialer Natur sei und auf der Beobachtung des Verhaltens anderer Menschen beruhe. Kinder würden am schnellsten lernen, indem sie anderen Menschen zuschauen und sie anschliessend imitieren \cite[S.~482ff]{Siegler2008}. Im Gegensatz zu den meisten anderen Lerntheoretikern betont Bandura die aktive Rolle der Kindern in ihrer Umwelt und Entwicklung als reziproken Determinismus zwischen ihnen und ihrer sozialen Umwelt. Kinder werden durch die Umweltaspekte beeinflusst, beeinflussen aber auch selbst die Umwelt (ebd.). 

Die Theorien der \textit{sozialen Kognition} befassen sich mit der Fähigkeit von Kindern, über Gefühle, Gedanken, Motive und Verhaltensweisen bei sich und anderer Menschen nachzudenken und daraus Schlüsse zu ziehen \cite[S.~486ff]{Siegler2008}. Somit umfasst die soziale Kognition alle Leistungen, die Aufschluss über die psychische Verfassung eines Anderen liefern \cite[S.~237ff]{Bischof2011}. Kinder verarbeiten soziale Informationen aktiv. Dabei achten sie darauf was andere Menschen sagen und tun, ziehen daraus permanten Schlüsse, nehmen Interpretationen vor und konstruieren Erklärungen \cite{Siegler2008}. Die soziale Kompetenz im ersten Lebensjahr beruht im Wesentlichen auf emotionalem Erleben, indem auf soziale Situationen mit angemessenen Emotionen reagiert wird. Die Emotionen dienen somit als Grundlage der sozialen Kognition bei Säuglingen \cite{Bischof2011}. Bereits Neugeborene lassen sich vom Geschrei anderer Babys anstecken \cite{Simner1971}. Die Reaktion erfolgt auf die menschliche Stimme und nicht etwa auf das Geräusch, was mit dem Begriff \textit{Gefühlsansteckung (engl. sharing of emotion)} gemäss \citeA{Tomasello2005} umschreiben wird.

Zusammenfassend kann gesagt werden, dass die Entwicklungsaufgaben bezüglich sozialer Entwicklung Vertrauen und Autonomie aufzubauen sind. Dies gelingt besonders gut in einer warmherzigen und einfühlsamen Beziehung zwischen Eltern und Kind \cite[S.~224]{Berk2011}.

\paragraph{Emotionale Entwicklung}\label{par:EmotionaleEntwicklung}
Emotionen sind durch ihre motivationale Kraft oder Handlungstendenz gekennzeichnet. Sie werden von Funktionalisten als der Versuch oder die Bereitschaft definiert, eine Beziehung mit der Umwelt hinsichtlich relevanter Aspekte herzustellen \cite[S.~529ff]{Siegler2008}. Emotionen sind von grundlegender Bedeutung für die Organisation von Fähigkeiten wie der Aufbau von Beziehungen, die Erkundung der Umwelt und die Entdeckung des eigenen Selbst \cite{Halle2003, Saarni2006}. Grundemotionen wie Freude, Furcht (Angst), Ärger (Wut) und Traurigkeit sind bei Menschen unviversell. Die frühe Gefühlslage eines Säuglings besteht aus wenig mehr als zwei umfassende Erregunszustände: einem sich zu einem angenehmen Stimuli hingezogenen und einem von unangenehmen Stimuli zurückziehenden \cite{Camras2003, Fox1991}. Erst mit der Zeit entwickeln sich Emotionen zu deutlicheren Zeichen. Sensible und passende Kommunikation von Eltern, bei der sie auf einzelne Aspekte des undifferenzierten emotionalen Verhaltens eines Säuglings eingehen, helfen dem Kind emotionale Ausdrucksformen zu entwickeln \cite{Gergely1999}. Eltern, die für die Signale ihrer Kinder wenig empfänglich sind, rufen bei ihren Kindern häufig traurige Gesichter, unklare Verbalisierungen, einen zusammensinkenden Körper, ein ärgerliches Gesicht oder Weinen und "nimm-mich-auf-den-Arm"-Gesten hervor \cite{Weinberg1994, Yale1999}. Bis in die Mitte des ersten Lebensjahr sind emotionale Ausdrucksformen bereits gut organisiert und spezifisch, die viel über das innere Befinden des Kindes aussagen \cite{Berk2011}. Wenn die Kommunikation zwischen Bezugsperson und Säugling ernsthaft gestört ist, kann häufig Traurigkeit auf Seiten des Säuglings beobachtet werden. Dies kann zum Beispiel bei einer depressiven Mutter oder eines depressiven Vaters geschehen, da die Fürsorge für ihr Kind beeinträchtigt ist und die Entwicklung erheblich stören kann.

Die \textit{emotionale Selbstregulierung} bezeichnet die Fähigkeit, innerer Gefühlszustände zu initiieren, zu hemmen oder zu modulieren, um ein Ziel zu erreichen \cite{Siegler2008}. In den ersten Lebensmonaten helfen Eltern ihren Kindern ihre emotionale Erregung zu regulieren, indem sie kontrollieren, wie stark diese den stimulierenden Ereignissen ausgesetzt sind \cite{Gianino1988}. Ein Säugling ist weniger empfindlich, äussert häufig angenehme Emotionen, erkundet sein Umfeld interessierter und ist leichter zu beruhigen, wenn dessen Eltern angemessen und liebevoll auf seine Hinweisreize reagieren \cite{Crockenberg2004}. Bezugspersonen, die gefühlsbelastenden Erlebnissen des Kindes unzureichend regulieren, riskieren, dass sich Gehirnstrukturen unzureichend entwickeln, welche für Stressregulierung dienen \cite[S.~250]{Berk2011}. Ab dem achten Monat beginnt der Säugling mit der sozialen Bezugnahme, er sucht aktiv nach emotional bewertenden Informationen von einem ihm vertrauten Person \cite{Mumme2007}. Im Allgemein geht die soziale Selbstregulierung mit einer höheren sozialen Kompetenz und einem geringeren Problemverhalten einher \cite[S.~580]{Siegler2008}. 

Die Qualität der Eltern-Kind-Beziehung kann die emotionale Entwicklung der Kinder auf verschiedene Arten beeinflussen. Es scheint als ob die Beziehungsqualität das Sicherheitsgefühl des Kindes sowie die Empfindung gegenüber der eigenen Person und andere Menschen beeinflusst (vgl. Unterkapitel \titleref{par:Bindung}). Diese Gefühle wiederum, beeinflusst die Emotionalität der Kinder \cite[S.~561]{Siegler2008}. Im Allgemeinen zeigen Kinder mit einer sicheren Bindung zu ihren Eltern eher positive Emotionen und weniger soziale Ängstlichkeit, verglichen mit Kindern mit einer unsichere Bindung \cite{Bohlin2000}. Der Ausdruck von Emotionen der Eltern kann die soziale Kompetenz und das psychische Wohlbefinden der Kinder beeinflussen. Die Sicht der Kinder auf sich selbst und andere hängt im hohen Masse von den Emotionen ab, die zu Hause gezeigt werden \cite{Dunsmore1997}. Zudem dient der elterliche Umgang mit Emotionen den Kindern als Modell, im Sinne von wann und wie Emotionen ausgedrückt werden \cite{Denham1994}. Dies kann das kindliche Verständnis zwischenmenschlicher Beziehung insofern beeinflussen, welche Formen des emotionalen Ausdrucks angemessen und effektiv ist \cite{Halberstadt1995}. Die Reaktionen der Eltern auf die Emotionen der Kinder scheinen einen Einfluss auf die emotionale Ausdrucksweise der Kinder zu haben. Werden Traurigkeit und Ängstlichkeit bei Kinder von den Eltern abgetan oder deren Gefühle kritisiert, so sind diese Kinder im Allgemeinen weniger emotional und weisen eine geringere soziale Kompetenz auf als Kinder, deren Eltern sie emotional unterstützen \cite{Eisenberg1998, McDowell2000}. 

\paragraph{Bindung}\label{par:Bindung}
Unter Bindung wird eine emotionale Beziehung zu einer bestimmten Person verstanden, die räumlich und zeitlich Bestand hat \cite[S.~585ff]{Siegler2008}. Die frühe Beziehungen der Kinder zu ihren Eltern beeinflusst die Art der Interaktionen mit anderen Menschen vom Kleinkindalter bis zum Erwachsenenalter, sowie ihr Selbstwertgefühl. Bindung kann als eine besondere Art einer affektiv getragenen sozialen Beziehung zwischen dem Kind und einer bevorzugten Person, die als stärker und klüger angesehen wird, verstanden werden \cite{Biringen1994}. Diese Gefühlsbeziehung zwischen dem Kind und der primären Bezugsperson wird auch als dyadische Beziehung in der psychoanalytischen Lieteratur beschrieben \cite{Resch1999}. Die Bindungstheorie wurde von John \citeA{Bowlby1969, Bowlby1973, Bowlby1988} und \citeA{Ainsworth1978} entwickelt und konzeptualisisert. Bolwbys Bindungstheorie ist stark von den Lehren Freuds beeinflusst. Insbesondere dadurch, dass die frühsten Beziehungen der Säuglinge zu ihren Müttern ihre späte Entwicklung formen \cite{Siegler2008}. Zudem wurde die Theorie interdisziplinär formuliert und hat ihre Wurzeln neben der psychonalytischen auch in der ethologischen und kognitiven Wissenschaft \cite{Resch1999}. Die Bindungstheorie beschreibt die angeboren Tendenz des menschlichen Individuums, starke Gefühlsbande zu spezifischen Personen der Umgebung zu bilden. Eine Zerstörung dieser Bande durch Trennung oder Verlust zu einer Bezugsperson kann zu affektiven Störungen führen \cite{Resch1999}. 

Bindung wir durch die Präferenz einer Bezugsperson charakterisiert, wobei ein Kind unterschiedliche Bindungsarten zu unterschiedlichen Bezugspersonen entwickeln kann. Es werden vier Hauptcharakteristiken der Bindung unterschieden: (1) Angst inhibitiert das Spiel- und Probierverhalten und verstärkt die Bindung. (2) Bei Anwesenheit einer Bezugsperson wird das Explorationsverhalten des Kindes in der Umgebung gefördert (engl. secure base phenomenon). (3) Ist die Bezugsperson anwesend, so verringert sich die Angst des Kindes in Krisensituationen und ungwohnten Sitautionen. (4) Erfolgt eine Trennung zur Bezugsperson, so wird mit Protest darauf reagiert \cite{Resch1999}. Somit ist das Bindungsverhalten die universelle Tendenz des Kindes, nach Nähe und Zuwendung zu einer Bezugsperson in Belastungssituationen zu suchen. Fehlt eine Bezugsperson, so kann das Kind auf dieser Suche auch Ersatzobjekte mit Bindung belegen \cite{Resch1999}.

Nach Bolwby findet die anfängliche Entwicklung von Bindung in vier Phasen statt \cite{Siegler2008}: (1) \textit{Vorphase der Bindung} (Geburt bis 6 Wochen). Das Kind reagiert auf alle Personen in seiner Umgebung mit universellen Mustern. (2) \textit{Entstehende Bindung} (6 Wochen bis 6 - 8 Monate). Das Kind reagiert bevorzugt auf vertraute Personen und entwickelt Erwartungen, wie ihre Fürsorger auf ihre Bedürfnisse reagieren und ein Gefühl, wie sehr sie ihnen vertrauen können. (3) \textit{Ausgeprägte Bindung} (zwischen 6 - 8 Monaten und $1\nicefrac{1}{2}$ - 2 Jahren. Das Kind zeigt ein selektives Bindungsverhalten und sucht die Nähe der Bezugsperson. Für die meisten Kinder dient die Mutter als sichere Basis. (4) \textit{Reziproke Beziehung} (von $1\nicefrac{1}{2}$ oder 2 Jahren an). Die rapide ansteigenden kognitiven und sprachlichen Fähigkeiten der Kinder ermöglicht ihnen, die Gefühle, Ziele und Motive der Eltern zu verstehen. Es ist zu Empathie fähig.

Bindung soll nicht mit Abhängigkeit verwechselt werden \cite{Schmidt1996}. Bindung scheint die Autonomieentwicklung des Kindes zu fördern, beruht auf einer geglückten gegenseitigen Beziehung und ist somit ein Merkmal der Interaktion. Vier Typen von Bindung können beschrieben werden \cite{Resch1999}: (1) \textbf{Sichere Bindung (Typ B):} Das Kind kann eine ausgewogene Balance zwischen der Suche nach Nähe und dem Explorationsverhalten im Beisein der Hauptbezugsperson aufrecht halten. Bei Trennung zeigt es Irritation, sucht nach Kontakt und lässt sich bei der Rückkehr der Bezugsperson leicht beruhigen. (2) \textbf{Unsicher vermeidenden Bindung (Typ A):} Im Beisein der Bezugsperson zeigt das Kind eine Pseudounabhängigkeit und scheint ihr gegenüber gleichgültig zu sein. Bei Trennung zeigt das Kind nur wenig Irritation, äussert bei Wiedervereinigung mit der Bezugsperson deutliche Ablehnung. Dieses äussere Desinteresse geht jedoch gemäss \citeA{Biringen1994} mit einer erhöhten inneren Erregung einher. (2) \textbf{Die ambivalent unsichere Bindung (Typ C):} Gekennzeichnet wird dieser Typ durch einen besondere Abhängigkeit zur Bezugsperson. Bei Trennung ist das Kind sehr irritiert, zeigt eine Kombination von Näheverhältnis und Verweigerung, und lässt sich bei der Wiedervereinigung nur schwer beruhigen. (4) \textbf{Desorganisierter Bindungstyp (Typ D):} Das Kind zeigt bei der Trennung keine konsistente Stressbewältigungsstrategie. Das Verhalten ist oft konfus oder sogar widersprüchlich \cite{Siegler2008}. Diese Bindungstypen wurden durch ein von \citeA{Ainsworth1978} entwickeltes Untersuchungsverfahren, das \enquote{Strange Situation Procedure}, entwickelt. Dabei wurden bei nicht klinischen Gruppen folgende Befunde erhoben: Etwa 57\% der Kinder waren in einer sicheren Bindung, 26\% zeigten ein unsicher-vermeidendes und 17\% ein ambivalent-unsicheres Bindungsmuster. Das vierte Bindungsmuster (Typ D) wird nur selten und vornehmlich in high-risk Gruppen von Kindern gefunden \cite{Resch1999}.

Sicher gebundene Kinder zeigen im Kindergarten und Schule ein adäquateres Sozialverhalten, mehr Phantasie und positive Affekte beim Spiel, ein höheres Selbstwertgefühl, weniger depressive Sympotme und eine längere Aufmerksamkeitsspanne \cite{Dornes1993, Zeanah1994}. Sicher gebundene Kinder scheinen im späteren Lebensalter offener und aufgeschlossener für neue Sozialkontakte mit Gleichaltrigen und Erwachsenen zu sein, als vermeidend- oder ambivalent-unsichere Kinder \cite{Resch1999}. Zudem zeigen Kinder, die mit einem Jahr als sicher gebunden eingeschätzt werden, mit 6 Jahren signifikant weniger Psychopathologie als unsicher gebundene \cite{Lewis1984}.

\paragraph{Die Entwicklung des Selbst und das Selbstwertgefühl}\label{par:EntwSelbst}
Bindungserfahrungen in den ersten Jahren beeinflusst die Entwicklung des Selbstgefühls. Die Vorstellung vom Selbst bei Kindern entsteht in den ersten Lebensjahren durch die Interaktionen mit andern ihnen wichtigen Bezugspersonen. Sie entwickelt sich bis ins Erwachsenenalter wobei sie mit der emotionaler und kognitiver Entwicklung mit steigendem Alter komplexer wird \cite[S.~602ff]{Siegler2008}. Vielen Theorien zufolge entwickelt sich das Selbstgefühl bei Säuglingen und Kleinkindern aufgrund der Erkenntnis, dass ihre eigenen Handlungen Gegensände, wie z.B. ein Modbile oder Menschen dazu veranlassen auf vorhersehbare Weise zu reagieren \cite{Harter1998}. Eltern, die sensibel auf die Signale ihre Kinder reagieren und sie ermutigen, ihre Umgebung zu erkunden, sind gegenüber ihren Altersgenossen bezüglich der Entwicklung ihres Selbst häufig voraus \cite{Pipp1992}. 

Das \textit{Selbstwertgefühl} ist ein wichtiges Element des Selbstkonzepts und bezeichnet die allgemeine Bewertung des Selbst und die Gefühle, die durch diese Bewertung erzeugt werden \cite{Crocker2001}. Das Selbstwertgefühl ist insofern wichtig, da es ein Prädiktor für die Zufriedenheit der Menschen in ihrem Leben darstellt. Individuen mit einem tiefen Selbstwertgefühl neigen dazu, sich wertlos, deprimiert und hoffnungslos zu fühlen, im Gegenzug zu Menschen, die ein hohes Selbstwertgefühl besitzen und die sich im Allgemeinen eher gut fühlen und hoffnungsvoll sind \cite{Harter1999}. Eine der wichtigste Einflussfaktoren auf den Selbstwert von Kindern ist die Anerkennung und Unterstützung, die sie von anderen erhalten \cite{Siegler2008}. Eltern, die ihrem Kind gegenüber mit Anerkennung und Interesse auftreten und die unterstützende und doch strenge Erziehungsmethoden anwenden, haben meistens auch Kinder mit hohem Selbstwertgefühl \cite{Feiring1996}. Hingegen bringen Eltern, die das Verhalten ihrer Kinder herabsetzen oder sie zurückweisen, ihnen tendenziell ein Gefühl der Wertlosigkeit bei \cite{Harter1999}. 

% ---------------------------------------
\subsection{Medienverhalten der Eltern}\label{sec:Medienverhalten}
Unter dem Medienverhalten der Eltern laufen viele Faktoren zusammen. Wie bereits erläutert, definiert sich diese Arbeit über das Medienverhalten der Eltern und soll unter dem Gesichtspunkt der Bindung, dem Stress und dem subjektiven Wohlbefinden betrachtet werden. Diese theoretischen Konstrukte werden im Folgenden für die in dieser Arbeit notwendigen Tiefe behandelt.

% ---------------------------------------
\subsubsection{Bindungsstatus der Eltern}\label{sec:Bindungsstatus}
Dieses Kapitel behandelt die Bindungstheorie aus Sicht der Eltern. Die Grundlagen dazu wurde im Absatz Bindung im Abschnitt \titleref{sec:Entwicklungsaufgaben} behandelt.

\citeA{Bowlby1973} postuliert, dass frühe Bindungserfahrungen in einem eigenständigen, repräsentativem System zusammengelagert werden, welches als \textit{Inneres Arbeitsmodell von Bindung} bezeichnet wird. Dieses Arbeitsmodell leitet die Handlungen der Eltern gegenüber ihren Kindern und beeinflussen die Bindungssicherheit dieser \cite{Siegler2008}. Diese Bindungsmodelle bei Erwachsenen basieren auf der Wahrnehmungen der eigenen Kindheitserfahrungen hinsichtlich der Beziehung zu ihren Eltern sowie der Wahrnehmung der Einflüsse von dieser Beziehung auf das Erwachsenenalter \cite{Main1985}.

Die elterlichen Bindungsmodelle der Eltern werden in der Regel in vier Bindungkategorien unterteilt: (1) autonome oder sichere, (2) abweisende, (3) verstrickte und (4) ungelöst-desorintierte Erwachsene \cite[S.~593ff]{Siegler2008}. Da diese Bindungsmuster das Ergebnis der reziproken Interaktion zwischen Kind und Bezugsperson darstellen, sind die frühen primären Bindungsmuster auf das Feinfühligkeitsverhalten der Bindungsperson zurückzuführen. Diese können bezüglich der Reaktionen der Bindungsperson auf das Kind wie folgt klassifiziert werden: (a) Konsistente und den Bedürfnissen des Kindes entsprechende Reaktion, (b) inkonsistentes und eher auf die eigenen Bedürfnisse entsprechende Reaktion und (c) konsistent abweisende und unfeinfühlige Reaktion auf das Verhalten des Kindes \cite{Schmidt2004}. 
Erwachsene, die als autonom oder sicher eingestuft werden, beschreiben ihre Vergangenheit in einer ausgeglichenen Weise und erinnern sowohl psoitive als auch negative Eingeschaften ihrer Eltern und ihrer Beziehung zu ihnen. Autonome Erwachsene sprechen über ihre Vergangenheit in einer konsistenten und kohärenten Weise. Abweisende Erwachsene beschreiben oftmals, dass sie sich nicht an die Interaktionen mit ihren Eltern erinnern können, oder spielen den Einfluss dieser Erfahrung herunter. Verstrickte Erwachsene scheinen intensiv auf ihre Eltern fokusiert zu sein und erinnern sich ihnen gegenüber an eine Reihe von verwirrenden und wutgeladenen Bindungserfahrungen. Es scheint, als ob sie in ihren Bindungserinnerungen gefangen sind und ihnen keine zusammenhängende Beschreibung möglich ist. Ungelöst-desorganisierte Erwachsene scheinen unter den Folgen posttraumatischer Erfahrungen von Verlust oder Missbrauch zu leiden. Ihre Beschreibung weisen Fehler auf und ergeben keinen Sinn \cite{Schmidt2004}.

Die Klassifikation der Eltern sagt sowohl ihr Einfühlungsvermögen gegenüber ihrer eigenen Kinder, sowie die Bindung ihrer Kinder an sie vorher \cite{Siegler2008}. Sichere Eltern sind in der Regel sensible und warmherzige Eltern und ihre Kinder sind in der Regel sicher gebunden \cite{Magai2000, Steele1996}. Entsprechend haben verstrickte und abweisende Eltern tendentiell unsicher gebundene Kinder. Diese allgemeinen Befunde wurden an mehreren unterschiedlichen westlichen Kulturen bestätigt \cite{Hesse1999}. Obwohl dieser deutliche Zusammenhang zwischen elterlichen Bindungsmodellen und der Bindungssicherheit der Kinder besteht, ist der Grund für diesen Zusammenhang nicht klar \cite{Siegler2008}. Nehliegend ist, dass autonome Eltern sensibler auf ihre Kinder eingehen und dass dies demzufolge zu einer sicheren Bindung bei den Kinder führt \cite{Pederson1998}. Anstelle der eigenen Bindungserfahrung der Eltern könnte die eigene persönliche Theorie über die Entwicklung der Kinder und Kindererziehung oder ihre Persönlichkeit  eine Rolle spielen \cite{Thompson1998}.

% ---------------------------------------
\subsubsection{TBD: Stress}\label{sec:Stress}

% ---------------------------------------
\subsubsection{TBD: Subjektives Wohlbefinden}\label{sec:Swb}
Siehe auch \citeA{Swami2009} für theoretischen Überblick. Und \citeA{Lyubomirsky1999}.






% ---------------------------------------
\subsection{Fragestellung und Hypothesen} \label{sec:Fragestellung}
Die Studie möchte einen möglichen Effekt zwischen dem Bindungsstil der Eltern und deren Medienverhalten während der Betreuung ihrer Kinder untersuchen. Um den Bindungsstil als ausschlaggebende Variable auf das elterliche Medienverhalten zu isolieren, soll der Stress der Eltern als moderierende Variable in die Untersuchung einfliessen.

Welche Auswirkungen das Medienverhalten auf die Eltern haben, soll mittels subjektiven Wohlbefinden  untersucht werden. 

Basierend auf dem oben beschriebenen theoretischen Hintergrund und anhand der aufgezeigten Forschungslücken erschliesst sich die im Folgenden aufgelistete Fragestellung. Basierend auf der Fragestellung wurden die Hypothesen erstellt.

\textbf{Fragestellung:}
Welchen Effekt hat der Bindungsstil und das aktuelle Stressempfinden der Eltern auf das im Beisein der Kinder praktizierte Medienverhalten? Kann zwischen diesem elterlichen Verhalten und deren subjektiven Wohlbefinden ein Zusammenhang gefunden werden?

\textbf{Hypothese 1:}
Eltern mit einem sicheren Bindungsstil weisen eine geringere Mediennutzung im Beisein ihrer Kinder auf als Eltern, die einen unsicheren Bindungsstil aufweisen.

\textbf{Hypothese 2:}
Eltern, die ein hohes Ausmass an Stress empfinden, nutzen Medien im Beisein ihrer Kinder häufiger als Eltern, die ein niedriges Ausmass an Stress aufweisen.

\textbf{Hypothese 3:}
Eltern mit einem erhöhten Medienverhalten im Beisein ihrer Kinder weisen ein geringeres subjektives Wohlbefinden auf als Eltern, die ein geringeres Medienverhalten im Beisein ihrer Kinder aufweisen.

\textbf{Hypothese 4:} 
Eltern mit einem sicheren Bindungsstil weisen ein tieferes Stressempfinden und eine geringere Mediennutzung auf als Eltern, die einen unsicheren Bindungsstil aufweisen.

