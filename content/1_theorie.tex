% ---------------------------------------
\subsection{\textit{WiP:} Theoretische Überlegungen anhand der Eltern-Kind-Beziehung}\label{sec:TheretischeÜberlegungen}
Die Eltern-Kind-Beziehung bezeichnet das aufeinander bezogene, gegenseitig bindende Verhaltensrepertoire zwischen dem Kind und den erwachsenen Menschen, welche die Elternrolle übernommen haben \cite{ElternKindBeziehung1999}. Ursprünglich als Mutter-Kind-Beziehung, wurde dies aufgrund wissenschaflticher Ergebnisse zum Begriff Elter-Kind-Beziehung erweitert, da auch ein konstanter, zuverlässiger Vater als Bezugsperson für das Kind dienen kann. Neben der Bindung zu den Hauptbezugspersonen können Kinder individualisierte Bindung in abgestufter Intensität zu andern Mitgliedern der Familie oder Sozialgruppe aufbauen. Dabei muss die Gelegenheit zu regelmässigen Zwiegesprächen gegeben sein, wodurch durch die Anwesenheit und Zuspruch der vertraut gewordenen Menschen die innere Sicherheit, Geborgenehit und Angstfreiheit beim Kind. Durch Kontinuität und Zuverlässigkeit in der liebevollen Betreuung entsteht eine sichere Bindung \cite{ElternKindBeziehung1999}. Gemäss \citeA{Wirtz2013} bezeichnet die Eltern-Kind-Beziehung verschiedene Aspekte des Verhältnisses zwischen Eltern und Kindern. Dabei wird zwischen Struktur- und Prozessmerkmale unterschieden. In dieser Arbeit stehen vor allem die Prozessmerkmale im Fokus, die sich auf die Qualität des Verhältnisses zwischen Eltern und Kindern bezieht. Wo sich hingegen die Strukturmerkmale auf verwandtschaftliche oder gesetzliche Verhältnisse zwischen Eltern und Kinder bezieht. Die Qualität der Beziehung kann einerseit unter der systemischen Perspektive betrachtet werden, in dem der Familienprozess als das Ergebnis wechselseiteiger Beeinflussung aller Familienmitgliederbetrachtet wird. Charakteristisch dafür sind Offenheit, Wechselseitigkeit, Harmonie, Streitkultur oder Grenzen. Weiter kann das spezifische Generationenverhältnis zwischen Eltern und Kindern betrachtet werden, wobei der Fokus meist auf der Erziehung und deren erziehungsstile gesetzt wird. 
\begin{center}
    \textit{TBD: Infografik - Eltern Kind Beziehung}
\end{center}

\begin{itemize}
    \item Interaktion \citeA{Resch1999} S. 91, Kap. 3 (Buch von Marc)
    \begin{itemize}
        \item Familie als Kontext S. 91
        \item Ineraktionelle Affektregulation S.101
    \end{itemize}
    \item Kognitive Entwicklung $\rightarrow$ Umfeld, Vererbung aus \citeA{Berk2011} Kapitel 5.4.2; Seite 224 (Online ZHAW)
    \begin{itemize}
        \item Pos. Korrelation bei Zuneigung und Engagement
        \item schlechte Betreuung S.225 $\rightarrow$ schlechte kognitive Entwicklung
        \item Kommunizieren lernen S.231 / Kap. 5.5 $\rightarrow$ Blickkontakt, Interaktion, viel Unterhaltung förder S. 235, Förderung des frühen Spracherwerbs S.236
    \end{itemize}
    \item Soziale Entwicklung
    \begin{itemize}
        \item Soziale Entwicklung S.469ff \citeA{Siegler2008} (Buch Jeannette)
        \item Soziale Kognition \& Soziale Entwicklung aus \citeA{Bischof2011} S. 237 (Buch von Marc)
    \end{itemize}
    \item Emotionale Entwicklung von Kindern in der Famile aus \citeA{Siegler2008} (Buch Jeannette)
    \begin{itemize}
        \item Qualität Eltern Kind Beziehung S. 561
    \end{itemize}
    \item Bindung aus diversen Quellen
    \begin{itemize}
        \item Bindung und Bindungsverhalten S.98 \citeA{Resch1999} (Buch von Marc)
        \item Bindungstheorie S.583ff \citeA{Siegler2008} (Buch Jeannette)
    \end{itemize}
    \item Die Entwicklungs des Selbst S.274 aus \citeA{Berk2011} (ZHAW Online) 
    \item Erziehungsstile S. 649 aus \citeA{Siegler2008} (Buch Jeannette)
\end{itemize}

% ---------------------------------------
\subsection{TBD: Entwicklungsaufgaben der Kinder} \label{sec:Entwicklungsaufgaben}

% ---------------------------------------
\subsection{TBD: Theoretische Ansätze auf Seiten der Eltern} \label{sec:TheretischeAnsätze}

\subsubsection{Bindung}\label{sec:Bindung}

\subsubsection{Stress}\label{sec:Stress}

\subsubsection{Subjektives Wohlbefinden}\label{sec:Swb}






% ---------------------------------------
\subsection{TBD: Fragestellung und Hypothesen} \label{sec:Fragestellung}
Die Studie möchte einen möglichen Effekt zwischen dem Bindungsstil der Eltern und deren Medienverhalten während der Betreuung ihrer Kinder untersuchen. Um den Bindungsstil als ausschlaggebende Variable auf das elterliche Medienverhalten zu isolieren, soll der Stress der Eltern als moderierende Variable in die Untesuchung einfliessen.

Welche Auswirkungen das Medienverhalten auf die Eltern haben, soll mittels subjektiven Wohlbefinden  untersucht werden. 

Basierend auf dem oben beschriebenen theoretischen Hintergrund und anhand der aufgezeigten Forschungslücken erschliesst sich die im Folgenden aufgelistete Fragestellung. Basierend auf der Fragestellung wurden die Hypothesen erstellt.
\subsubsection{Fragestellung} 
Welchen Effekt hat der Bindungsstil und das aktuelle Stressempfinden der Eltern auf das im Beisein der Kinder praktizierte Medienverhalten? Kann zwischen diesem elterlichen Verhalten und deren subjektiven Wohlbefinden ein Zusammenhang gefunden werden?
\subsubsection{Hypothese 1}
Eltern mit einem sicheren Bindungsstil weisen eine geringere Mediennutzung im Beisein ihrer Kinder auf als Eltern, die einen unsicher-vermeidenden, unsicher-ambivalenten oder desorganisierten Bindungsstil aufweisen.
\subsubsection{Hypothese 2}
Eltern, die ein hohes Ausmass an Stress empfinden, nutzen Medien im Beisein ihrer Kinder häufiger als Eltern, die ein niedriges Ausmass an Stress aufweisen.
\subsubsection{Hypothese 3}
Eltern mit einem erhöhten Medienverhalten im Beisein ihrer Kinder weisen ein geringeres subjektives Wohlbefinden auf als Eltern, die ein geringeres Medienverhalten im Beisein ihrer Kinder aufweisen.

