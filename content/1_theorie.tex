% ---------------------------------------
\subsection{Theoretische Überlegungen}\label{sec:TheretischeÜberlegungen}
Dieses Kapitel stellt die theoretischen Grundlagen dieser Arbeit und Überlegungen dar, auf welchen die später generierte Forschungsfrage und die Hypothesen aufbauen. 
In einem ersten Teil wird aus theoretischer Sicht auf die Interaktion zwischen den Eltern und dem Kind eingegangen. Bei der Eltern-Kind-Interaktion werden die kindlichen Faktoren, die Induktion durch elterliche Verhaltensmuster, die Modellbildung durch Imitationsprozesse, die Affektregulierung der Säuglinge und die Erziehungsstile beschrieben. Mit den Entwicklungsaufgaben der Kinder, in denen es um die kognitive, soziale, emotionale Entwicklung und die Entwicklung der Bindung geht, wird der erste theoretische Teil der Eltern-Kind-Interaktion abgeschlossen.

Im zweiten Teil dieses Kapitels soll aus theoretischer Sicht auf das Verhalten der Eltern eingegangen werden, welches für diese Arbeit in Bezug auf die Eltern-Kind-Interaktion relevant ist und das eigentliche Medienverhalten der Eltern zu moderieren scheint. In diesem Teil enthalten ist der Bindungsstatus der Eltern, der in Bezug auf die Eltern-Kind-Interaktion betrachtet wird, das psychologische Konstrukt des Stresses, der aufgrund der täglichen Situationen und der subjektiven Empfindung erläutert wird und das subjektive Wohlbefinden, das als psychologisches Konstrukt für das Glück der Eltern steht.


\subsubsection{Die Eltern-Kind-Beziehung}\label{sec:ElternKindBeziehung}
Die Eltern-Kind-Beziehung bezeichnet das aufeinander bezogene, gegenseitig bindende Verhaltensrepertoire zwischen dem Kind und den erwachsenen Menschen, welche die Elternrolle übernommen haben \cite{LexikonDerBiologie1999}. Ursprünglich als Mutter-Kind-Beziehung bezeichnet, wurde der Begriff aufgrund wissenschaftlicher Ergebnisse zur Eltern-Kind-Beziehung erweitert, da auch ein konstant anwesender, zuverlässiger Vater als Bezugsperson für das Kind dienen kann. Neben der Bindung zu den Hauptbezugspersonen können Kinder individualisierte Bindung in abgestufter Intensität zu andern Mitgliedern der Familie oder Sozialgruppen aufbauen. Dabei muss die Gelegenheit zu regelmässigem Austausch gegeben sein. Durch die Anwesenheit und den Zuspruch der vertraut gewordenen Menschen werden die innere Sicherheit, die Geborgenheit und die Angstfreiheit beim Kind gestärkt. Durch Kontinuität und Zuverlässigkeit in der liebevollen Betreuung entsteht eine sichere Bindung \cite{LexikonDerBiologie1999}. Gemäss \citeA{Wirtz2013} bezeichnet die Eltern-Kind-Beziehung verschiedene Aspekte des Verhältnisses zwischen Eltern und Kindern. Dabei wird zwischen Struktur- und Prozessmerkmalen unterschieden. In dieser Arbeit stehen vor allem die Prozessmerkmale im Fokus, die sich auf die Qualität des Verhältnisses zwischen Eltern und Kindern beziehen. Strukturmerkmale hingegen beziehen sich oft auf verwandtschaftliche oder gesetzliche Verhältnisse zwischen Eltern und Kindern und werden in dieser Arbeit nicht einbezogen. Die Qualität der Beziehung kann unter der systemischen Perspektive betrachtet werden, in welcher der Familienprozess als das Ergebnis wechselseitiger Beeinflussung aller Familienmitglieder verstanden wird. Charakteristisch dafür sind Offenheit, Wechselseitigkeit, Harmonie, Streitkultur oder Grenzen. Des Weiteren kann das spezifische Generationenverhältnis zwischen Eltern und Kindern betrachtet werden, wobei der Fokus meist auf der Erziehung den Erziehungsstilen liegt. 

Der Übergang von einer Paargemeinschaft zur Elternschaft führt zu einer grundlegenden Umstrukturierung von einer Zweier- zu einer Dreierbeziehung  \nohyphens{\cite{Hofer1992, Buergin1998}}. Dies gilt als eine zentrale Entwicklungsaufgabe von Familien, wobei die Eltern-Kind-Beziehung nicht als einseitige Einflussnahme zu verstehen ist. Vielmehr stehen die Eltern und die Kinder in einer reziproken sozialisatorischen Beziehung zueinander. \nohyphens{\citeA{Hofer1992}} nennen wichtige Einflussgrössen der dyadischen Beziehung zwischen Vater, Mutter und Kind, die sich auf die kindliche Entwicklung auswirken können: Synchronizität der Interaktion, Persönlichkeit der Eltern, Beziehungsqualität der Partner sowie die affektive und kognitive Ausstattung des Kindes selbst. 

\begin{figure}%[htbp] -> am Schluss fürs Feintuning :-)
  \centering
     \includegraphics[width=1.0\textwidth]{content/Grafik/Infografik_ElternKindBeziehung_Uebersicht.png}
  \caption{Infografik: Eltern-Kind-Beziehung}
  \label{fig:InfografikElternKindBeziehung}
\end{figure}

% ---------------------------------------
\paragraph{Eltern-Kind-Interaktion}\label{sec:Interaktion}
Grundsätzlich ist der Mensch gemäss \citeA[S.~91]{Resch1999} in ein Gefüge von zwischenmenschlichen Relationen eingebettet, welche einen wesentlichen Einfluss auf das innere Weltverständnis und das Selbstbild nehmen.  Die frühe Kindheit ist eine kritische Periode in der Entwicklung, wobei die wichtigste frühe Umgebung des Kindes die Familie resp. das primäre Umfeld ist \cite{Resch1999}. Wird die Familie im Entwicklungskontext betrachtet, so kann der elterliche Einfluss auf das Kind durch die zwei Begriffe \textit{Beziehung} und \textit{Erziehung} definiert werden. Dadurch weist die Eltern-Kind-Beziehung immer Beziehungsqualitäten und Erziehungsqualitäten auf.  

Auch \textit{kindliche Faktoren} beeinflussen die frühen Interaktionen deutlich. Das Kind ist schon sehr früh in der Lage, selektiv und adäquat auf die emotionale Ausdrucksform der Bezugsperson einzugehen \cite{Harris1994}. Dabei haben Kinder im Säuglingsalter von der Bezugsperson die Erwartung, dass diese adäquat auf die Gefühlsausdrücke reagiert. Dies ermöglicht bereits dem Säugling einen Informationsaustausch zu seiner Bezugsperson. Somit ist ein Säugling in der Lage, sich der Bezugsperson anhand des emotionalen Ausdrucks anzunähern oder es bleiben zu lassen \cite{Resch1999}. Dieser Vorgang wird als \textit{soziale Vergewisserung (\textit{engl. social referencing})} bezeichnet und meint, dass ein Kind nach der Beantwortung seiner affektiven Gestimmtheit sucht, wenn es sich seiner Bezugsperson zuwendet. Es konsultiert den Gesichtsausdruck der Bezugsperson, um über die Bedeutung eines ihm unbekannten Ereignisses eine Sinnzuschreibung (\textit{engl. meaning attribution}) zu erhalten. Diese frühen affektiven Regulatoren der Interaktion dienen der Entwicklung des kindlichen Interesses und des kindlichen Bewertungsschemas. Je nach emotionalem Austausch wird das Kind eine Kohärenz in seinen Erwartungen entwickeln oder nicht. Wenn die kindlichen emotionalen Signale adäquat beantwortet werden, entsteht eine günstige Basis für das Erfahrungslernen \cite[S.~95]{Resch1999}. Die emotionale Kommunikation konstituiert das kindliche Weltbild in Form von Selbst- und Objektrepräsentanzen. Inkonsistente emotionale Kommunikation führt zu einem weniger kohärenten und weniger vorhersagbaren inneren Schema der Welt. 

Ein weiterer Interaktionsfaktor ist die \textit{Induktion} aus der empirischen Säuglingsforschung. Kindliche Gefühlszustände und Verhaltensmuster werden durch elterliche Verhaltensmuster induziert \cite{Cummings1994}. Intrusive feindliche und insensitive Elternreaktionen können einen negativen Erregungsprozess im Kind auslösen, woraus sich ungünstige Interferenzen mit den sich entwickelnden Fähigkeiten des Kindes ergeben, seine eigenen Erregungsimpulse zu modulieren und zu regulieren.  

Über Imitationsprozesse kann die \textit{Modellbildung} im Kind durch das elterliche Verhalten beeinflusst werden. Elterliche Verhaltensweisen und Sichtweisen gegenüber der Welt werden internalisiert \cite{Resch1999}. Die Eltern beeinflussen die Entwicklung des Kindes nicht nur als direkte Akteure, sondern auch durch ihre Vorbildfunktion. 

Durch die frühe Interaktion mit einer Bezugsperson wird auch die interaktionelle \textit{Affektregulation} geprägt. Nach \citeA{Dornes1993, Dornes1997} werden in den frühen Interaktionen zwischen Mutter und Kind affekthaltige Handlungen ausgetauscht. Das Kind kann seine innere Erfahrung mit einem anderen Menschen teilen und mit diesem darüber kommunizieren. Eine Form dieser affektiven Kommunikation wird gemäss \citeA{Stern1985} als \textit{Affektabstimmung (engl. affective attunement)} bezeichnet. Dabei wird auf bestimmte Gefühlsäusserungen des Kindes von Seiten der Bezugsperson differenziert geantwortet, wobei die Antwort etwas stärker oder schwächer als die des Kindes ausfällt. Durch gezieltes Abdämpfen oder Stimulieren der kindlichen Gefühlsausdrücke und Aktionen können Erlebnis- und Handlungsfolgen akzentuiert, abgedämpft oder sogar gelöscht werden. Reagiert eine Bezugsperson unsensibel auf das Kind und werden Handlungsintentionen immer wieder unterbrochen, so kann sich dies negativ auf die Entwicklung des Kindes auswirken \cite{Resch1999}.

Die wechselseitige Bedingtheit des Verhaltens von Menschen wird \textit{Kontingenz} genannt. Gemeint ist dabei die regelhafte Aufeinanderfolge einzelner Verhaltensschritte. Die Wichtigkeit kontingenter Beantwortung kindlicher Signale, vor allem in der Anfangsphase der Eltern-Kind-Beziehung, betont \citeA{Papousek1987, Papousek1989}. Kontingenz bedeutet dabei die zeitliche und inhaltliche Passung der elterlichen Reaktion auf das Verhalten des Kindes, also die zeitliche Folge der Reaktion und die inhaltliche Entsprechung. Diese frühe Verständigung zwischen Eltern und Kind würde nicht gelingen, wenn sich die Eltern nicht kompensatorisch an die begrenzten kindlichen Voraussetzungen anpassen würden. Diese Fähigkeit schafft die Voraussetzung für eine gelingende Kommunikation mit dem Säugling im vorsprachlichen Alter. Kontingentes Antworten von Seiten der Eltern ermöglicht eine Kohärenz der Interaktion, die zum subjektiven Gefühl der Kontrolle beim Kind führt.

\textit{Erziehungsstile} sind Verhaltensweisen und Einstellungen, die das emotionale Klima für die Eltern-Kind-Interaktion bestimmen \cite[S.~649]{Siegler2008}. Um den Einfluss festzulegen, den Eltern auf die Entwicklung von Kindern nehmen können, wurden zwei Dimensionen des Erziehungsstils identifiziert: (1) das Ausmass an elterlicher Wärme, Unterstützung und Akzeptanz sowie (2) das Ausmass an elterlicher Kontrolle und Anforderung \cite{Maccoby1983}. Diese Aspekte der elterlichen Erziehung spielen eine wichtige Rolle bei der Ausbildung interindividueller Unterschiede zwischen Kindern. \citeA{Baumrind1973} entwickelte anhand der Dimensionen Unterstützung und Kontrolle vier Erziehungsstile: autoritativ, autoritär, permissiv und vernachlässigend-zurückweisend \cite{Baumrind1991}. Nach dieser Typologie neigen \textbf{autoritative} Eltern dazu Anforderungen zu stellen, aber auch auf das Kind einzugehen und warmherzig zu sein. Gemäss \citeA{Baumrind1991} sind Kinder autoritativer Eltern häufig kompetent, selbstbewusst und bei ihren Altersgenossen beliebt. Als Jugendliche besitzen sie häufig hohe soziale und schulische Fähigkeiten, Selbstvertrauen und positive Verhaltensweisen \cite{Lamborn1991}. \textbf{Autoritäre} Eltern sind oft kalt und reagieren nicht auf die Bedürfnisse ihrer Kinder. Des Weiteren üben sie starke Kontrolle aus und stellen hohe Anforderungen. Forderungen an die Kinder erzwingen sie häufig durch Ausüben ihrer elterlichen Macht. Kinder autoritärer Eltern besitzen oft relativ geringe soziale und schulische Kompetenzen, sind unglücklich und unfreundlich und besitzen ein geringes Selbstvertrauen. Im Jugendalter wirkt sich dies auf eine geringere soziale und schulische Kompetenz im Vergleich zu Kindern autoritativer Eltern aus \cite{Baumrind1991, Lamborn1991}.  \textbf{Permissive} Eltern reagieren auf die Bedürfnisse und Wünsche ihrer Kinder und sind nachsichtig. Sie sind nicht konservativ und verlangen von ihren Kindern nicht sich selbst zu regulieren. Kinder permissiver Eltern sind häufig impulsiv, es fehlt ihnen an Selbstbeherrschung und sie zeigen oft schwache schulische Leistungen \cite{Baumrind1991}. Als Jugendliche fallen sie durch häufigen Drogenkonsum und schlechtes Benehmen gegenüber Kindern autoritativer Eltern auf \cite{Lamborn1991}. \textbf{Zurückweisend-vernachlässigende} Eltern sind uninteressierte Eltern, die wenige Anforderungen stellen und auch wenig auf die Kinder reagieren. Sie setzen dem Verhalten der Kinder keine Grenzen und kontrollieren es auch nicht. Säuglinge und Kleinkinder zurückweisend-vernachlässigender Eltern weisen häufig gestörte Bindungsbeziehungen und Probleme in der Beziehungsgestaltung zu ihren Altersgenossen auf \cite{Parke1998}.


\paragraph{Entwicklungsaufgaben der Kinder}\label{sec:Entwicklungsaufgaben} Über die Entwicklungsaufgaben der Kinder sind zahlreiche Bücher geschrieben worden und der Rahmen dieser Arbeit würde gesprengt, müssten hier alle Ansätze und Theorien Platz finden. Aus diesem Grund wurde eine Auswahl getroffen, die keinen Anspruch auf Vollständigkeit erhebt und doch die relevantesten im Bereich der Eltern-Kind-Beziehung unter dem Fokus der Interaktion näher beleuchtet. Dadurch soll dem Anspruch genüge getan werden, wichtige theoretische Inputs für die Klärung der Forschungsfrage dieser Arbeit zu liefern. Zum Beispiel wird die körperliche Entwicklung, wenn auch eine wichtige Aufgabe eines Kindes, hier nicht weiter behandelt, da die Schnittpunkte zum Forschungsthema zu gering ausfallen. Zum Beispiel wird die körperliche Entwicklung, wenn auch eine wichtige Aufgabe eines Kindes, hier nicht weiter behandelt, da die Schnittpunkte zum Forschungsthema zu gering ausfallen.

\subparagraph{Kognitiven Entwicklung}\label{par:KognitiveEntwicklung}
Im Rahmen der kognitiven Entwicklung spielen die sozio-kulturellen Theorien eine wichtige Rolle im Eltern-Kind-Gefüge. Diese Theorien stellen die Bedeutung der Interaktion zwischen Kindern und anderen Menschen zusammen mit der umgebenden Kultur ins Zentrum der Entwicklung der Kinder \cite[S.~225 ff.]{Siegler2008}. Als Beispiel für eine sozio-kulturelle Theorie kann die \textit{gelenkte Partizipation} genannt werden. Darunter wird ein Prozess verstanden, bei dem als Experten geltende Erwachsene ihre Aktivität so gestalten, dass sich Menschen mit geringeren Kenntnissen - in diesem Fall die Kinder - an diesen beteiligen können, obwohl sie von sich aus dazu noch nicht fähig sind und daraus etwas lernen können \cite{Rogoff1990}. Die gelenkte Partizipation dient oft der Erreichung eines praktischen Ziels, zum Beispiel dem Zusammenbauen eines Autos. Das Lernen ist dabei ein Nebenprodukt der Tätigkeit.

Weiter ist die Interaktion zwischen den Eltern und dem Kind in der Einbettung des kulturellen Kontexts für die Entwicklung wichtig. Es gehören nicht nur andere Menschen dazu, sondern auch die zahlreichen Produkte der menschlichen Erfindungskraft, die als sogenannte Kulturwerkzeuge bezeichnet werden. Dazu gehören Symbolsysteme, Artefakte, Fähigkeiten und Werte \cite[S.~226]{Siegler2008}. Gemäss Wygotski sind Kinder soziale Wesen, geformt durch ihren kulturellen Kontext, den sie auch selbst wiederum mitgestalten \cite[S.~227]{Siegler2008}. 

Die Grundlage der kognitiven Entwicklung des Menschen liegt gemäss sozio-kulturellen Theoretikern in der Fähigkeit, \textit{Intersubjektivität} herzustellen: das wechselseitige Verständnis, das Menschen bei der Kommunikation füreinander aufbringen \cite{Gauvain2001, Rogoff1990}. Dabei gilt, dass sich die Beteiligten für eine effektive Verständigung auf dasselbe Thema konzentrieren und auf die Reaktion des andern achten müssen. Dazu ist ein wirksames Lehren und Lernen unverzichtbar. Zwei- bis dreimonatige Säuglinge sind lebhafter und interessierter, wenn die Bezugsperson auf ihre Aktionen reagiert \cite{Murray1985}. Sechsmonatige Kinder können neue Verhaltensweisen allein durch die Beobachtung des Verhaltens anderer Menschen erlernen \cite{Collie1999}. Im Zentrum der Intersubjektivität steht die \textit{geteilte Aufmerksamkeit (engl. joint attention)}, welche einen Prozess beschreibt, bei dem die sozialen Partner ihre Aufmerksamkeit intendiert auf einen gemeinsamen Gegenstand in der äusseren Umwelt richten. Zum Beispiel schauen die Kinder im Alter zwischen 9 und 15 Monaten zunehmend auf dieselben Gegenstände wie ihre Sozialpartner und richten die Aufmerksamkeit von Erwachsenen aktiv auf Objekte, die sie selbst interessieren  \nohyphens{\cite{Adamson1991, Gauvain2001}}. Diese gemeinsame Aufmerksamkeit spielt eine wichtige Rolle bei der frühen Sprachentwicklung \cite[S.~232]{Berk2011}. Eine verwandte Verhaltensweise ist das \textit{soziale Referenzieren (engl. social referencing)}, ein Verhalten, um vom Sozialpartner einen Rat oder eine Anleitung für eine unbekannte oder bedrohliche Situation zu bekommen, indem dieser angeschaut wird \cite{Campos1981} (siehe auch Abschnitt \textit{\nameref{sec:Interaktion}}). 

Das frühe Umfeld ist für die Entwicklung des Kindes existentiell wichtig. Diverse Autoren konnten aufzeigen, dass Zuneigung, Engagement sowie Ermutigung der Eltern bei den Säuglingen und Kleinkindern bessere Ergebnisse bei Sprach- und Intelligenztests hervorrufen \nohyphens{\cite{Fuligni2004, Linver2004, TamisLeMonda2004}}. Das Ausmass, in dem Eltern mit ihren Säuglingen sprechen, trägt erheblich zum frühen Fortschritt des Spracherwerbs bei, was wiederum förderlich für die Intelligenz und gute Schulleistungen in der Grundschule ist \cite{Hart1995}. Warmherziges, einfühlsames elterliches Verhalten ist ein guter Indikator, wie das Kind kognitiv später abschneiden wird \cite[S.~225]{Berk2011}. Der Betreuung steht demzufolge ein grosser Anteil der Entwicklungsarbeit zu. Die Forschung konnte aufzeigen, dass schlecht betreute Kinder in kognitiven und sozialen Fertigkeiten schlechter abschneiden \cite{NICHD2006}.

\subparagraph{Soziale Entwicklung}\label{par:SozialeEntwicklung}
Theorien zur sozialen Entwicklung versuchen zu erklären, wie Menschen und soziale Institutionen in der Umwelt von Kindern auf ihre Entwicklung einwirken \cite[S.~470 ff.]{Siegler2008}. Diese Theorien sind auf die Erklärung vieler wichtiger Entwicklungsaspekte gerichtet, wie Emotion, Persönlichkeit, Bindung, Selbst, Beziehung zu Gleichaltrigen, Moral oder Geschlecht. Bei den psychoanalytischen Theorien von Freud und Erikson wird die soziale Entwicklung  durch die biologische Reifung betont. Für Freud ist das Verhalten durch die Befriedigung von grundlegenden Trieben definiert, die weitgehend unbewusst ablaufen. In der Theorie von Erikson wird die Entwicklung durch eine Reihe altersbezogener Entwicklungskrisen resp. Entwicklungsaufgaben bestimmt \cite[S.~472]{Siegler2008}. Die Bedeutung dieser beiden Theorien liegen auf der Betonung der Wichtigkeit früher Lebenserfahrungen und der emotionalen Beziehung sowie der Suche nach Identität. Erikson betonte, dass für eine gesunde Entwicklung nicht die Quantität der Zuwendung zwischen Kind und Bezugsperson ausschlaggebend ist, sondern die Qualität \cite[S.~243]{Berk2011}. Nämlich dass sich zum Beispiel die Mutter unverzüglich und einfühlend um ihr Kind kümmert und sich durch diesen liebevollen Umgang der psychische Konflikt zwischen Urvertrauen versus Misstrauen positiv lösen lässt. Zudem betont Erikson, dass der Konflikt des Kleinkindes betreffend Autonomie versus Scham und Zweifel positiv aufgelöst wird, wenn Eltern ihrem Kind klare Grenzen aufzeigen und genügend Möglichkeiten zur Entfaltung bieten \cite[S.~243]{Berk2011}. 

Im Gebiet der Lerntheorien soll für diese Arbeit vor allem die Theorie des \textit{sozialen Lernens} hervorgehoben werden. Diese versucht die Persönlichkeit und andere Aspekte der sozialen Entwicklung anhand von Lernmechanismen zu erklären, wobei die Beobachtung und Nachahmung im Zentrum stehen. \citeA{Bandura1979, Bandura1986} gibt an, dass der grösste Teil des menschlichen Lernens sozialer Natur sei und auf der Beobachtung des Verhaltens anderer Menschen beruhe. Kinder würden am schnellsten lernen, indem sie anderen Menschen zuschauen und sie anschliessend imitieren \cite[S.~482 ff.]{Siegler2008}. Im Gegensatz zu den meisten anderen Lerntheoretikern betont Bandura die aktive Rolle der Kinder in ihrer Umwelt und Entwicklung als reziproken Determinismus zwischen ihnen und ihrer sozialen Umwelt. Kinder werden durch die Umweltaspekte beeinflusst, beeinflussen aber auch selbst die Umwelt \cite[S.~482 ff.]{Siegler2008}. 

Die Theorien der \textit{sozialen Kognition} befassen sich mit der Fähigkeit von Kindern, über Gefühle, Gedanken, Motive und Verhaltensweisen bei sich und anderen Menschen nachzudenken und daraus Schlüsse zu ziehen \cite[S.~486 ff.]{Siegler2008}. Somit umfasst die soziale Kognition alle Leistungen, die Aufschluss über die psychische Verfassung eines anderen liefern \cite[S.~237 ff.]{Bischof2011}. Kinder verarbeiten soziale Informationen aktiv. Dabei achten sie darauf, was andere Menschen sagen und tun, ziehen daraus permanent Schlüsse, nehmen Interpretationen vor und konstruieren Erklärungen \cite{Siegler2008}. Die soziale Kompetenz im ersten Lebensjahr beruht im Wesentlichen auf emotionalem Erleben, indem auf soziale Situationen mit angemessenen Emotionen reagiert wird. Die Emotionen dienen somit als Grundlage der sozialen Kognition bei Säuglingen \cite{Bischof2011}. Bereits Neugeborene lassen sich vom Geschrei anderer Babys anstecken \cite{Simner1971}. Die Reaktion erfolgt auf die menschliche Stimme und nicht auf das Geräusch, was mit dem Begriff \textit{Gefühlsansteckung (engl. sharing of emotion)} gemäss \citeA{Tomasello2005} umschrieben wird.

Zusammenfassend kann gesagt werden, dass die Entwicklungsaufgaben bezüglich sozialer Entwicklung das Aufbauen von Vertrauen und Autonomie sind. Dies gelingt besonders gut in einer warmherzigen und einfühlsamen Beziehung zwischen Eltern und Kind \cite[S.~224]{Berk2011}.

\subparagraph{Emotionale Entwicklung}\label{par:EmotionaleEntwicklung}
Emotionen sind durch ihre motivationale Kraft oder Handlungstendenz gekennzeichnet. Sie werden von Funktionalisten als der Versuch oder die Bereitschaft definiert, eine Beziehung mit der Umwelt hinsichtlich relevanter Aspekte herzustellen \cite[S.~529 ff.]{Siegler2008}. Emotionen sind von grundlegender Bedeutung für die Organisation von Fähigkeiten wie den Aufbau von Beziehungen, die Erkundung der Umwelt und die Entdeckung des eigenen Selbst \nohyphens{\cite{Halle2003, Saarni2006}}. Grundemotionen wie Freude, Furcht (Angst), Ärger (Wut) und Traurigkeit sind bei Menschen universell. Die frühe Gefühlslage eines Säuglings besteht aus wenig mehr als zwei umfassenden Erregunszuständen: einem sich zu einem angenehmen Stimulus hingezogenen und einem von einem unangenehmen Stimulus zurückziehenden \nohyphens{\cite{Camras2003, Fox1991}}. Erst mit der Zeit entwickeln sich Emotionen zu deutlicheren Zeichen. Sensible und passende Kommunikation von Eltern, bei der sie auf einzelne Aspekte des undifferenzierten emotionalen Verhaltens eines Säuglings eingehen, helfen dem Kind emotionale Ausdrucksformen zu entwickeln \cite{Gergely1999}. Eltern, die für die Signale ihrer Kinder wenig empfänglich sind, rufen bei ihren Kindern häufig traurige Gesichter, unklare Verbalisierungen, einen zusammensinkenden Körper, ein ärgerliches Gesicht oder Weinen und \enquote{Nimm-mich-auf-den-Arm-Gesten} hervor \nohyphens{\cite{Weinberg1994, Yale1999}}. Bis in die Mitte des ersten Lebensjahrs sind emotionale Ausdrucksformen bereits gut organisiert und spezifisch. Sie sagen viel über das innere Befinden des Kindes aus \cite{Berk2011}. Wenn die Kommunikation zwischen Bezugsperson und Säugling ernsthaft gestört ist, kann häufig Traurigkeit auf Seiten des Säuglings beobachtet werden. Dies kann zum Beispiel bei einer depressiven Mutter oder einem depressiven Vater geschehen, da die Fürsorge für das Kind beeinträchtigt ist und das die Entwicklung erheblich stören kann.

Die \textit{emotionale Selbstregulierung} bezeichnet die Fähigkeit, innere Gefühlszustände zu initiieren, zu hemmen oder zu modulieren, um ein Ziel zu erreichen \cite{Siegler2008}. In den ersten Lebensmonaten helfen Eltern ihren Kindern ihre emotionale Erregung zu regulieren, indem sie kontrollieren, wie stark diese den stimulierenden Ereignissen ausgesetzt sind \cite{Gianino1988}. Ein Säugling ist weniger empfindlich, äussert häufig angenehme Emotionen, erkundet sein Umfeld interessierter und ist leichter zu beruhigen, wenn dessen Eltern angemessen und liebevoll auf seine Hinweisreize reagieren \cite{Crockenberg2004}. Bezugspersonen, die gefühlsbelastende Erlebnisse des Kindes unzureichend regulieren, riskieren, dass sich Gehirnstrukturen unzureichend entwickeln, welche der Stressregulierung dienen \cite[S.~250]{Berk2011}. Ab dem achten Monat beginnt der Säugling mit der sozialen Bezugnahme. Er sucht aktiv nach emotional bewertenden Informationen von einer ihm vertrauten Person \cite{Mumme2007}. Im Allgemeinen geht die soziale Selbstregulierung mit einer höheren sozialen Kompetenz und einem geringeren Problemverhalten einher \cite[S.~580]{Siegler2008}. 

Die Qualität der Eltern-Kind-Beziehung kann die emotionale Entwicklung der Kinder auf verschiedene Arten beeinflussen. Es scheint, als ob die Beziehungsqualität das Sicherheitsgefühl des Kindes sowie die Empfindung gegenüber der eigenen Person und anderen Menschen beeinflusst (vgl. folgendes Unterkapitel \textit{Bindung}). Diese Gefühle beeinflussen wiederum die Emotionalität der Kinder \cite[S.~561]{Siegler2008}. Im Allgemeinen zeigen Kinder mit einer sicheren Bindung zu ihren Eltern eher positive Emotionen und weniger soziale Ängstlichkeit, verglichen mit Kindern mit einer unsicheren Bindung \cite{Bohlin2000}. Der Ausdruck von Emotionen der Eltern kann die soziale Kompetenz und das psychische Wohlbefinden der Kinder beeinflussen. Die Sicht der Kinder auf sich selbst und andere hängt im hohen Masse von den Emotionen ab, die zu Hause gezeigt werden \cite{Dunsmore1997}. Zudem dient der elterliche Umgang mit Emotionen den Kindern als Modell, im Sinne von wann und wie Emotionen ausgedrückt werden \cite{Denham1994}. Dies kann im kindlichen Verständnis zwischenmenschlicher Beziehung beeinflussen, welche Formen des emotionalen Ausdrucks als angemessen und effektiv wahrgenommen werden \cite{Halberstadt1995}. Die Reaktionen der Eltern auf die Emotionen der Kinder scheinen einen Einfluss auf die emotionale Ausdrucksweise der Kinder zu haben. Werden Traurigkeit und Ängstlichkeit bei Kindern von den Eltern abgetan oder deren Gefühle kritisiert, so sind diese Kinder im Allgemeinen weniger emotional und weisen eine geringere soziale Kompetenz auf als Kinder, deren Eltern sie emotional unterstützen \nohyphens{\cite{Eisenberg1998, McDowell2000}}. 

\subparagraph{Bindung}\label{par:Bindung}
Unter Bindung wird eine emotionale Beziehung zu einer bestimmten Person verstanden, die räumlich und zeitlich Bestand hat \cite[S.~585 ff.]{Siegler2008}. Die frühen Beziehungen der Kinder zu ihren Bezugspersonen beeinflussen die Art der Interaktionen mit anderen Menschen sowie ihr Selbstwertgefühl vom Kleinkindalter bis zum Erwachsenenalter. Bezugspersonen können die biologischen Eltern sein, aber auch Personen, die einen engen Bezug zum Kind haben. Bindung kann verstanden werden als eine besondere Art einer affektiv getragenen sozialen Beziehung zwischen dem Kind und einer bevorzugten Person, die als stärker und klüger angesehen wird \cite{Biringen1994}. Diese Gefühlsbeziehung zwischen dem Kind und der primären Bezugsperson wird auch als dyadische Beziehung in der psychoanalytischen Literatur beschrieben \cite{Resch1999}. Die Bindungstheorie wurde von John \nohyphens{\citeA{Bowlby1969, Bowlby1973, Bowlby1988}} und \citeA{Ainsworth1978} entwickelt und konzeptualisiert. Bolwbys Bindungstheorie ist stark von den Lehren Freuds beeinflusst. Insbesondere da die frühsten Beziehungen der Säuglinge zu ihren Müttern ihre spätere Entwicklung formen \cite{Siegler2008}. Zudem wurde die Theorie interdisziplinär formuliert und hat ihre Wurzeln neben der psychonalytischen auch in der ethnologischen und kognitiven Wissenschaft \cite{Resch1999}. Die Bindungstheorie beschreibt die angeborene Tendenz des menschlichen Individuums, starke Gefühlsbande zu spezifischen Personen der Umgebung zu bilden. Eine Zerstörung dieser Bande durch Trennung oder Verlust zu einer Bezugsperson kann zu affektiven Störungen führen \cite{Resch1999}. 

Bindung wir durch die Präferenz einer Bezugsperson charakterisiert, wobei ein Kind unterschiedliche Bindungsarten zu unterschiedlichen Bezugspersonen entwickeln kann. Es werden vier Hauptcharakteristiken der Bindung unterschieden: (1) Angst inhibitiert das Spiel- und Probierverhalten und verstärkt die Bindung; (2) bei Anwesenheit einer Bezugsperson wird das Explorationsverhalten des Kindes in der Umgebung gefördert (\textit{engl. secure base phenomenon}); (3) ist die Bezugsperson anwesend, so verringert sich die Angst des Kindes in Krisensituationen und ungewohnten Situationen; (4) erfolgt eine Trennung zur Bezugsperson, so wird mit Protest reagiert \cite{Resch1999}. Somit ist das Bindungsverhalten die universelle Tendenz des Kindes, nach Nähe und Zuwendung zu einer Bezugsperson in Belastungssituationen zu suchen. Fehlt eine Bezugsperson, kann das Kind auf dieser Suche auch Ersatzobjekte mit Bindung belegen \cite{Resch1999}.

Nach Bowlby findet die anfängliche Entwicklung von Bindung in vier Phasen statt \cite{Siegler2008}). In der (1) \textit{Vorphase der Bindung} (Geburt bis 6 Wochen) reagiert das Kind auf alle Personen in seiner Umgebung mit universellen Mustern. In der Phase der (2) \textit{entstehenden Bindung} (6 Wochen bis 6-8 Monate) reagiert das Kind bevorzugt auf vertraute Personen und entwickelt Erwartungen, wie seine Fürsorgerinnen und Fürsorger auf Bedürfnisse reagieren, und ein Gefühl, wie sehr es ihnen vertrauen kann. Während der Phase der (3) \textit{ausgeprägten Bindung} (zwischen 6-8 Monaten und 1.5-2 Jahren) zeigt das Kins ein selektives Bindungsverhalten und sucht die Nähe der Bezugsperson. Für die meisten Kinder dient die biologische Mutter als sichere Basis. In der (4) \textit{reziproken Beziehung} (ab 1.5-2 Jahren) ermöglichen die rapide ansteigenden kognitiven und sprachlichen Fähigkeiten dem Kind, die Gefühle, Ziele und Motive der Eltern zu verstehen. Es ist zu Empathie fähig.

Bindung soll nicht mit Abhängigkeit verwechselt werden \cite{Schmidt1996}. Bindung scheint die Autonomieentwicklung des Kindes zu fördern, beruht auf einer geglückten gegenseitigen Beziehung und ist somit ein Merkmal der Interaktion. Vier Typen von Bindung können beschrieben werden \cite{Resch1999}: In einer (1) \textit{sicheren Bindung (Typ B)} kann das Kind eine ausgewogene Balance zwischen der Suche nach Nähe und dem Explorationsverhalten im Beisein der Hauptbezugsperson aufrechthalten. Bei Trennung zeigt es Irritation, sucht nach Kontakt und lässt sich bei der Rückkehr der Bezugsperson leicht beruhigen. Bei der (2) \textit{unsicher-vermeidenden Bindung (Typ A)} zeigt das Kind im Beisein der Bezugsperson eine Pseudounabhängigkeit und scheint ihr gegenüber gleichgültig zu sein. Bei Trennung zeigt das Kind nur wenig Irritation, äussert bei Wiedervereinigung mit der Bezugsperson deutliche Ablehnung. Dieses äussere Desinteresse geht jedoch gemäss \citeA{Biringen1994} mit einer erhöhten inneren Erregung einher. Die (3) \textit{ambivalent-unsichere Bindung (Typ C)} ist gekennzeichnet durch eine besondere Abhängigkeit zur Bezugsperson. Bei Trennung ist das Kind sehr irritiert, zeigt eine Kombination von Näheverhältnis und Verweigerung und lässt sich bei der Wiedervereinigung nur schwer beruhigen. Beim (4) \textit{desorganisierten Bindungstyp (Typ D)} zeigt das Kind bei der Trennung keine konsistente Stressbewältigungsstrategie. Das Verhalten ist oft konfus oder sogar widersprüchlich \cite{Siegler2008}. Diese Bindungstypen wurden durch ein von \citeA{Ainsworth1978} entwickeltes Untersuchungsverfahren entwickelt: die \textit{Strange Situation Procedure}. Dabei wurden bei nicht klinischen Gruppen folgende Befunde erhoben: Etwa 57 \% der Kinder waren in einer sicheren Bindung, 26 \% zeigten ein unsicher-vermeidendes und 17 \% ein ambivalent-unsicheres Bindungsmuster. Das vierte Bindungsmuster (Typ D) wurde nur selten und vornehmlich in Hochrisikogruppen von Kindern gefunden \cite{Resch1999}.

Sicher gebundene Kinder zeigen im Kindergarten und in der Schule ein adäquateres Sozialverhalten, mehr Phantasie und positive Affekte beim Spiel, ein höheres Selbstwertgefühl, weniger depressive Symptome und eine längere Aufmerksamkeitsspanne \nohyphens{\cite{Dornes1993, Zeanah1994}}. Sicher gebundene Kinder scheinen im späteren Lebensalter offener und aufgeschlossener für neue Sozialkontakte mit Gleichaltrigen und Erwachsenen zu sein als vermeidend- oder ambivalent-unsichere Kinder \cite{Resch1999}. Zudem zeigen Kinder, die mit einem Jahr als sicher gebunden eingeschätzt werden, mit sechs Jahren signifikant weniger Psychopathologie als unsicher gebundene \cite{Lewis1984}.

\subparagraph{Die Entwicklung des Selbst und das Selbstwertgefühl}\label{par:EntwSelbst}
Bindungserfahrungen in den ersten Jahren beeinflussen die Entwicklung des Selbstgefühls. Die Vorstellung vom Selbst bei Kindern entsteht in den ersten Lebensjahren durch die Interaktionen mit anderen, ihnen wichtigen Bezugspersonen. Sie entwickelt sich bis ins Erwachsenenalter, wobei sie mit der emotionalen und kognitiven Entwicklung mit steigendem Alter komplexer wird \cite[S.~602 ff.]{Siegler2008}. Vielen Theorien zufolge entwickelt sich das Selbstgefühl bei Säuglingen und Kleinkindern aufgrund der Erkenntnis, dass ihre eigenen Handlungen Gegenstände, wie z. B. ein Mobile oder Menschen dazu veranlassen, auf vorhersehbare Weise zu reagieren \cite{Harter1998}. Kinder von Eltern, die sensibel auf die Signale ihrer Kinder reagieren und sie ermutigen, ihre Umgebung zu erkunden, sind gegenüber ihren Altersgenossen bezüglich der Entwicklung ihres Selbst häufig voraus \cite{Pipp1992}. 

Das \textit{Selbstwertgefühl} ist ein wichtiges Element des Selbstkonzepts und bezeichnet die allgemeine Bewertung des Selbst und die Gefühle, die durch diese Bewertung erzeugt werden \cite{Crocker2001}. Das Selbstwertgefühl ist insofern wichtig, da es ein Prädiktor für die Zufriedenheit der Menschen in ihrem Leben ist. Individuen mit einem tiefen Selbstwertgefühl neigen dazu, sich wertlos, deprimiert und hoffnungslos zu fühlen, im Gegensatz zu Menschen, die ein hohes Selbstwertgefühl besitzen und sich im Allgemeinen eher gut fühlen und hoffnungsvoll sind \cite{Harter1999}. Zu den wichtigsten Einflussfaktoren auf den Selbstwert von Kindern gehören die Anerkennung und Unterstützung, die sie von anderen erhalten \cite{Siegler2008}. Eltern, die ihrem Kind gegenüber mit Anerkennung und Interesse auftreten und die unterstützende und doch strenge Erziehungsmethoden anwenden, haben meistens auch Kinder mit hohem Selbstwertgefühl \cite{Feiring1996}. Hingegen bringen Eltern, die das Verhalten ihrer Kinder herabsetzen oder sie zurückweisen, ihnen tendenziell ein Gefühl der Wertlosigkeit bei \cite{Harter1999}. 

% ---------------------------------------
\subsubsection{Das Verhalten der Eltern}\label{sec:Medienverhalten}
Aus obigen Erläuterungen wird klar, dass das Verhalten der Eltern eine zentrale Rolle in der Eltern-Kind-Beziehung spielt. Basierend auf den zu Beginn dieser Arbeit gestellten Annahmen und der bisherigen Forschung fokussiert sich diese Arbeit auf das Medienverhalten der Eltern bezüglich der elterlichen Bindung, dem von den Eltern subjektiv erlebten Stress und dem daraus resultierenden subjektiven Wohlbefinden. Diese wissenschaftlichen Konstrukte werden im Folgenden erläutert und in der für diese Arbeit notwendigen Tiefe behandelt.

% ---------------------------------------
\paragraph{Bindungsstatus der Eltern}\label{sec:Bindungsstatus}
Dieses Kapitel behandelt die Bindungstheorie aus Sicht der Eltern. Die Grundlagen dazu wurde bereits im Absatz \textit{Bindung} im Abschnitt \textit{\nameref{sec:Entwicklungsaufgaben}} auf Seite \pageref{sec:Entwicklungsaufgaben} behandelt.

\citeA{Bowlby1973} postuliert, dass frühe Bindungserfahrungen in einem eigenständigen, repräsentativen System zusammengelagert werden, welches als \textit{Inneres Arbeitsmodell von Bindung} bezeichnet wird. Dieses Arbeitsmodell leitet die Handlungen der Eltern gegenüber ihren Kindern und beeinflusst deren Bindungssicherheit \cite{Siegler2008}. Diese Bindungsmodelle bei Erwachsenen basieren auf den Wahrnehmungen der eigenen Kindheitserfahrungen hinsichtlich der Beziehung zu ihren Eltern sowie der Wahrnehmung der Einflüsse dieser Beziehung auf das Erwachsenenalter \cite{Main1985}.

Die Bindungsmodelle der Eltern werden in der Regel in vier Bindungskategorien unterteilt: (1) autonome oder sichere, (2) abweisende, (3) verstrickte und (4) ungelöst-desorientierte Erwachsene \cite[S.~593 ff.]{Siegler2008}. Da diese Bindungsmuster das Ergebnis der reziproken Interaktion zwischen Kind und Bezugsperson darstellen, sind die frühen primären Bindungsmuster auf das Feinfühligkeitsverhalten der Bindungsperson zurückzuführen. Diese können bezüglich der Reaktionen der Bindungsperson auf das Kind wie folgt klassifiziert werden: (a) konsistente und den Bedürfnissen des Kindes entsprechende Reaktion, (b) inkonsistente und eher auf die eigenen Bedürfnisse entsprechende Reaktion und (c) konsistent abweisende und unfeinfühlige Reaktion auf das Verhalten des Kindes \cite{Schmidt2004}. 
Erwachsene, die als autonom oder sicher eingestuft werden, beschreiben ihre Vergangenheit in einer ausgeglichenen Weise und erinnern sich an positive wie negative Eigenschaften ihrer Eltern und ihre Beziehung zu ihnen. Autonome Erwachsene sprechen über ihre Vergangenheit in einer konsistenten und kohärenten Weise. Abweisende Erwachsene beschreiben oft, dass sie sich nicht an die Interaktionen mit ihren Eltern erinnern können oder spielen den Einfluss dieser Erfahrung herunter. Verstrickte Erwachsene sind intensiv auf ihre Eltern fokussiert und erinnern sich an eine Reihe von verwirrenden und wutgeladenen Bindungserfahrungen. Sie sind in ihren Bindungserinnerungen gefangen und ihnen ist keine zusammenhängende Beschreibung möglich. Ungelöst-desorganisierte Erwachsene leiden unter den Folgen posttraumatischer Erfahrungen von Verlust oder Missbrauch. Ihre Beschreibungen weisen Fehler auf und ergeben keinen Sinn \cite{Schmidt2004}.

Die Klassifikation der Eltern sagt ihr Einfühlungsvermögen gegenüber den eigenen Kindern und die Bindung ihrer Kinder an sie voraus \cite{Siegler2008}. Sichere Eltern sind in der Regel sensible und warmherzige Eltern und ihre Kinder sind in der Regel sicher gebunden \nohyphens{\cite{Magai2000, Steele1996}}. Entsprechend haben verstrickte und abweisende Eltern tendenziell unsicher gebundene Kinder. Diese allgemeinen Befunde wurden bei mehreren unterschiedlichen westlichen Kulturen bestätigt \cite{Hesse1999}. Obwohl dieser deutliche Zusammenhang zwischen elterlichen Bindungsmodellen und der Bindungssicherheit der Kinder besteht, ist der Grund für diesen Zusammenhang nicht klar \cite{Siegler2008}. Naheliegend ist, dass autonome Eltern sensibler auf ihre Kinder eingehen und dies zu einer sicheren Bindung bei den Kindern führt \cite{Pederson1998}. Anstelle der eigenen Bindungserfahrung der Eltern könnte die eigene persönliche Theorie über die Entwicklung der Kinder und Kindererziehung oder ihre Persönlichkeit  eine Rolle spielen \cite{Thompson1998}.

% ---------------------------------------
\paragraph{Stress}\label{sec:Stress}
Allgemein bedeutet Stress eine subjektiv unangenehm empfundene Situation, von der eine Person negativ beeinflusst wird \cite[S.~1500]{Stress2014}. Dabei wird unterschieden zwischen Distress, der als unangenehm empfundene Stress und Eustress, der als anregend empfundene Stress. Die zahlreichen Untersuchungen dazu können unter drei Perspektiven zusammengefasst werden \cite[S.~1500]{Stress2014}. In der (1) \textit{stimulusorientierten Sicht} werden Stressoren über bestimmte Reize, Situations- oder Bedingungsmerkmale operationalisiert. Untersuchungen haben dabei ergeben, dass Personen unterschiedlich auf Reize reagieren. In der (2) \textit{Arbeitsperspektive} werden bei Stress am Arbeitsplatz Faktoren, sogenannte Stressorenklassen identifiziert, die als Stressoren wahrgenommen werden. Das kann extremer Zeitdruck oder Monotonie sein. Bei der (3) \textit{transaktionalen Perspektive} steht die Inkongruenz zwischen den Anforderungen der Umwelt und den Ressourcen des Individuums im Zentrum. Dabei spielen Einschätzungen der eigenen Bewältigungsmöglichkeiten (Bewertungsprozesse) im Bezug auf das Ausmass des Stessempfindens eine entscheidende Rolle. 

Vom Adaptionssyndrom \cite{Selye1936} bis zur Lebensereignisforschung zieht sich eine rote Linie, derzufolge Stress als eine Reaktion des Organismus auf gewissermassen objektive äussere Stressoren konzeptualisiert wird \cite{Fliege2001}. Erhebungsmethodisch wurde versucht, diese Stressoren zu objektivieren. Dies erfolgte zunächst über die Messung physikalischer Einwirkungen wie Lärm, Licht etc. und über psychodiagnostische Versuche, bei denen versucht wurde, objektive Stressoren anhand kritischer Lebensereignisse zu definieren \cite{Holmes1980} oder über tägliche, geringfügige Belastungen und Ärgernisse zu erheben (\textit{engl. daily hassles}) \cite{Quast1986}. Das Stresskonzept wurde im gesundheitspsychologischen Zusammenhang weiterentwickelt und beeinflusst durch die Arbeiten zum Typ-A-Verhaltensmuster von \citeA{Friedman1974}, zur kognitiven Belastungs- und Bewältigungseinschätzung von \citeA{Lazarus1966} und zur Selbstwirksamkeitsüberzeugung von \citeA{Bandura1977}. Die subjektive Wahrnehmung, Bewertung und Weiterverarbeitung von Stressoren durch das Individuum wurde zunehmend in den Mittelpunkt gerückt und die Objektivierbarkeit und Allgemeingültigkeit von äusseren Stressoren in Frage gestellt \cite{Fliege2001}. Das von \citeA{Schulz1999} entwickelte Verfahren setzt das Stresserleben der Person in den Mittelpunkt. Das Augenmerk liegt dabei auf der Häufigkeit der retrospektiv erfragten Stresserfahrungen, welchen gemäss dem Vulnerabilitäts-Stress-Modell pathogenetische Wirkung zugeschrieben werden kann (chronischer Stess vs. vorübergehende Belastung).

Bezüglich der täglich ausgesetzten Informationsflut durch die Medien sprechen \citeA[S.~161 ff.]{Klingberg2008} von \textit{Infostress}. Dabei beziehen sie sich auf die begrenzten Möglichkeiten, die ein Individuum hat, Informationen aufzunehmen. Diese neuen Anforderungen können zu einer Überschreitung der geistigen Leistungsfähigkeit führen, was mit erhöhtem Stress einhergehen kann. \citeA{Klingberg2008} schliessen daraus, dass das Stressniveau vom Kontext und von der Deutung der Situation durch das Individuum abhängt. Stress habe demzufolge in hohem Masse mit der eigenen Einstellung zu tun, dem Kontrollgefühl, einer Situation gewachsen zu sein.

% ---------------------------------------
\paragraph{Subjektives Wohlbefinden}\label{sec:Swb}
Subjektives Wohlbefinden (\acrshort{swb}) wird in der psychologischen Forschung oft mit \textit{Glück} gleichgesetzt \cite{Eid2014}. Es lassen sich zwei generelle Perspektiven unterscheiden: das subjektive Wohlbefinden und das eudämonische Wohlbefinden. Das erstere bezeichnet Personen als glücklich, die eine häufige Lebenszufriedenheit aufweisen sowie häufig positive und selten negative Gefühle erleben. Im Gegensatz zu diesem hedonistischen Wohlbefinden zeichnet sich das eudämonische oder auch psychologische Wohlbefinden aus durch ein gelingendes Leben und menschliche Stärke. In beiden Richtungen wurden unterschiedliche Theorien und Bedingungen entwickelt. Dabei sind interindividuelle Unterschiede beider Richtungen nicht unabhängig voneinander, sondern hängen stark zusammen \cite{Eid2014}.

Bereits \citeA{Aristoteles1952} (384-323 v.Chr.) bezeichnete Glück als das höchste Gut, das der Mensch um seiner selbst Willen anstrebt. Obschon die Menschheit Glück seit jeher sucht, wurde dieses Thema von der Psychologie lange vernachlässigt, zusammen mit Freude, Heiterkeit und Wohlbefinden \nohyphens{\cite{Tugade2014, Gruber2014}}. Eine Ursache dafür beschreibt \citeA{Maddux2005} in der sogenannten \textit{Krankheitsideologie}, die vor allem in der Psychiatrie und der klinischen Psychologie anzutreffen sei. Nach wie vor ist von \enquote{Symptom} die Rede, von \enquote{Störung, Pathologie, Krankheit, Diagnose, Behandlung etc.} \cite[S.~14]{Maddux2005}. Im erstmals 1952 von der Amerikanischen Psychologenvereinigung herausgegebenen DSM wurden auf 86 Seiten 106 psychische Störungen beschrieben. Fünfzig Jahre später (1994) waren es 297 Störungen \cite{Bucher2009}. Dies verdeutlicht, dass sich vor allem die Psychiatrie hauptsächlich auf Krankheiten und Störungen fokussiert.

Es scheint, als ob Glück als Regel angeschaut wird. Glückliche Menschen haben keinen Anlass, professionelle Hilfe aufzusuchen, für Glück scheint es keinen Bedarf zu geben \cite[S.~14]{Veenhoven1991}. Zum Beispiel ist es für Eltern selbstverständlich, dass sich ihre Kinder normal entwickeln sowie gesund und glücklich sind. In diesem Fall steigt das Glück der Eltern nicht an. Es wird jedoch deutlich vermindert, wenn die Kinder Probleme bereiten \cite{Fingerman2012}. Negative Emotionen scheinen hervorstechender und schwerer zu regulieren als die für gegeben hingenommenen positiven Gefühle \cite{Charles2010}. 

Die Stunde der Glücksforschung begann, so \citeA{Vitterso2013}, in den 1960er Jahren, als \citeA{Wilson1967} Korrelate von Glück untersuchte. Da wurde der typisch glückliche Mensch als jung, gesund, gebildet, gut bezahlt, extravertiert, optimistisch, weiss und verheiratet definiert, egal ob männlich oder weiblich \cite{Bucher2009}. 1984 veröffentlichte Ed Diener \cite{Diener1984} seinen Aufsatz \textit{Subjective Well-being}, was als wichtiges Jahr für die Glückspsychologie angesehen wird und ihm den Namen \enquote{Jedi-Ritter} der Glücksforschung einbrachte \cite{Metzger2010}.

Die Positive Psychologie wurde von \citeA{Seligman2000} proklamiert und läutete ein neues Zeitalter der Glücksforschung ein \cite{Bucher2009}. Zwischen 1980 und 1985 erschienen 2152 Publikation zu Glück, Lebenszufriedenheit und Wohlbefinden. Zwischen 2000 und 2005 waren es bereits 35'069 \cite{Donaldson2014}. Positive Psychologie wurde als Wissenschaft des Glücks bezeichnet \cite{Carr2011} und beschäftigt sich mit den Tugenden und Stärken, die Glück erwiesenermassen erhöhen \cite{Peterson2004}.

Die UNO begann 2012 einen Weltglücksreport zu erstellen \cite{Helliwell2013}. Glücksforschung ist gemäss \citeA{Bucher2009} notwendig, weil mehr und mehr Menschen selbstquälerisch vor sich hin grübeln, obwohl es in der Ersten Welt noch nie so vielen Menschen materiell gut ging. Die Quote der sehr Glücklichen blieb über die Jahre konstant bei einem Drittel. Depressionen wurden zur Jahrtausendwende zehnmal häufiger diagnostiziert als vor fünfzig Jahren \cite{Seligman2009}. Dem hält \citeA{Layous2011} entgegen, dass glücksbegünstigende Strategien Depressionen nachhaltiger und kostengünstiger mindern als Psychopharmaka.

Eine vielfältig abgesicherte Erkenntnis ist, dass sich Glück lohnt und deshalb weiterhin psychologische Glücksforschung benötigt wird \cite{Lyubomirsky2005}. Glückliche Menschen leben gemäss \citeA{Danner2001} länger, haben ein leistungsfähigeres Immunsystem \cite{Barak2006}, sind effizienter bei der Arbeit und erfolgreicher im Beruf \cite{Achor2010}, lernen leichter, schneller und nachhaltiger \cite{Endres2014}, sind kreativer \cite{Baas2008} und haben vielfältigere Sozialbeziehungen \cite{Rodriguez2014}.


% ---------------------------------------
\subsection{Fragestellung und Hypothesen} \label{sec:Fragestellung}
Aus der oben dargestellten Theorie lässt sich ableiten, dass die Interaktion zwischen Eltern und Kindern immanent wichtig für die Entwicklung der Kinder ist. Dabei scheint jegliches Verhalten der Eltern, sei es durch das Vorzeigen in der Interaktion oder die Erledigung alltäglicher Handlungen, die Kinder zu prägen. Die Entwicklungsaufgaben der Kinder werden durch dieses elterliche Verhalten begünstigt oder erschwert. Durch den Einzug neuer Medien in die familiären Strukturen ist es von Interesse, diese genauer zu untersuchen und zu beleuchten. Wie in der Theorie beschrieben sind die Lern- und Entwicklungsaufgaben im ersten Jahr eines Kindes enorm vielseitig und bilden die Basis für die weitere Entwicklung. Somit gebührt dem Einfluss der Eltern in dieser sensiblen Phase ein besonderes Augenmerk. 

Auf Basis der geschilderten theoretischen Überlegungen möchte diese Studie einen möglichen Effekt zwischen dem Bindungsstil der Eltern und deren Medienverhalten während der Betreuung ihrer Kinder untersuchen. Die heutige Informationsflut stellt zusätzliche Anforderungen in Form von Stress an die Eltern. Um mögliche Unterschiede im Medienverhalten der Eltern zu untersuchen, soll der subjektiv erlebte Stress der Eltern als moderierende Variable einbezogen werden. Ob sich die Eltern durch das Medienverhalten während der Betreuung ihrer Kinder mehr Glück versprechen, soll mittels subjektiven Wohlbefinden (\acrshort{swb}) untersucht werden.

Basierend auf diesen Überlegungen lassen sich Fragestellung, die daraus abgeleiteten Hypothesen und die Grundhypothese formulieren:

\textbf{Fragestellung:}
Welchen Effekt haben der Bindungsstil und das aktuelle Stressempfinden der Eltern auf das im Beisein der Kinder praktizierte Medienverhalten? Kann zwischen diesem elterlichen Verhalten und deren subjektivem Wohlbefinden ein Zusammenhang hergestellt werden?

Für die Formulierung der Hypothesen wurde die Fragestellung aufgeteilt und in aufeinander aufbauende Hypothesen umgesetzt. 

\begin{itemize}
  \item[H1:] Eltern mit einem sicheren Bindungsstil nutzen Medien im Beisein ihrer Kinder weniger als Eltern, die einen unsicheren Bindungsstil haben.
  \item[H2:] Eltern, die ein hohes Ausmass an Stress empfinden, nutzen Medien im Beisein ihrer Kinder häufiger als Eltern, die ein niedriges Ausmass an Stress empfinden.
  \item [H3:] Eltern mit einer erhöhten Mediennutzung im Beisein ihrer Kinder weisen ein geringeres subjektives Wohlbefinden auf als Eltern mit einer geringeren Mediennutzung im Beisein ihrer Kinder.
  \item [H4:] Eltern mit einem sicheren Bindungsstil haben ein tieferes Stressempfinden und damit eine geringere Mediennutzung als Eltern, mit einem unsicheren Bindungsstil.
\end{itemize}

Aus diesen vier Hypothesen lässt sich zusammenfassend eine \textbf{Grundhypothese} bilden, welche die einzelnen Aussagen in einer Schlussfolgerung vereint: Eltern mit einem sicheren Bindungsstil nutzen Medien seltener, erleben weniger subjektiven Stress dabei und sind zufriedener als Eltern mit einem unsicheren Bindungsstil.